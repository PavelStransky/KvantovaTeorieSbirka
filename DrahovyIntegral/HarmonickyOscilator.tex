\subsection{Propagátor harmonického oscilátoru}
Cílem této úlohy je spočítat propagátor jednorozměrného harmonického oscilátoru daného Hamiltoniánem
\begin{equation}
\operator{H}=\frac{1}{2M}\operator{p}^{2}+\frac{1}{2}M\Omega^{2}\operator{x}^{2}.
\end{equation}
Budeme sledovat stejné kroky jako v případě propagátoru částice v homogenním poli.
Potenciál je kvadratický, propagátor bude tedy úměrný 
\begin{equation}
G(x_{\mathrm{f}}, t_{\mathrm{f}}; x_{\mathrm{i}}, t_{\mathrm{i}})\sim\e^{\frac{\im}{\hbar}S_{\mathrm{kl}}}
\end{equation}
(označili jsme $S_{\mathrm{kl}}\equiv S\left[x_{\mathrm{kl}}(t)\big|_{x_{\mathrm{kl}}(t_{\mathrm{i,f}})=x_{\mathrm{i,f}}}\right]$,
což je akce pro klasickou trajektorii s okrajovými podmínkami 
$x_{\mathrm{kl}}(t_{\mathrm{i}})=x_{\mathrm{i}}$, $x_{\mathrm{kl}}(t_{\mathrm{f}})=x_{\mathrm{f}}$).

\begin{enumerate}
\item Nalezněte klasickou trajektorii $x_{kl}(t)$ částice pohybující se v poli harmonického oscilátoru
při okrajových podmínkách $x_{\mathrm{kl}}(t_{\mathrm{i}})=x_{\mathrm{i}}$, $x_{\mathrm{kl}}(t_{\mathrm{f}})=x_{\mathrm{f}}$.

\item Nalezněte rychlosti $v_{\mathrm{i}}=\dot{x}_{\mathrm{kl}}(t_{\mathrm{i}})$, $v_{\mathrm{f}}=\dot{x}_{\mathrm{kl}}(t_{\mathrm{f}})$.

\item Rozložme libovolnou trajektorii $x(t)$ s okrajovými podmínkami $x(t_{\mathrm{i,f}})=x_{\mathrm{i,f}}$ na 
\begin{equation}
x(t)=x_{\mathrm{kl}}(t)+x_{\mathrm{q}}(t),
\end{equation}
kde $x_{\mathrm{kl}}$ splňuje výše uvedené okrajové podmínky a pro $x_{\mathrm{q}}$ platí $x_{\mathrm{q}}(t_{\mathrm{i,f}})=0$.
Nalezněte diferenciální operátor $\operator{A}$, který odpovídá matici $A$ Gaussovské integrace, 
a na prostoru funkcí $x_{\mathrm{q}}$ nalezněte jeho vlastní čísla.

\item Jelikož $\det{\frac{\operator{A}}{2\pi}}$ nekonverguje, regularizujte jej vydělením determinantem volné částice $\det{\frac{\operator{A}_{0}}{2\pi}}$,
kde $\operator{A}_{0}=\frac{\im}{\hbar}M\frac{\d^{2}}{\d t^{2}}$, a dopočítejte. 
Můžete využít vztah
\begin{equation}
\prod_{k=1}^{\infty}\left[1-\left(\frac{x}{k\pi}\right)^{2}\right]=\frac{\sin{x}}{x}.
\end{equation}

\item Ukažte, že porovnáním s propagátorem volné částice, kde má diferenciální operátor tvar $\operator{A}_{0}$, 
můžete propagátor harmonického oscilátoru vyjádřit ve tvaru
\begin{equation}
G(x_{\mathrm{f}}, t_{\mathrm{f}}; x_{\mathrm{i}}, t_{\mathrm{i}})=\sqrt{\frac{M}{2\im\pi\hbar\left(t_{\mathrm{f}}-t_{\mathrm{i}}\right)}}
\sqrt{\frac{\det{\frac{\operator{A}_{0}}{2\pi}}}{\det{\frac{\operator{A}}{2\pi}}}}\,
\e^{\frac{\im}{\hbar}S[x_{\mathrm{kl}}(t)]}\nonumber
\end{equation}
Dosaďte, dopočítejte akci klasické trajektorie a napište konečný výsledek.
\end{enumerate}

Díky regularizaci máme v rukou prostředek, který výrazně zjednodušuje výpočet dráhového integrálu. 
Nemusíme již uvažovat jeho diskrétní vyjádření. 

Dalším krokem je \emph{Van Vleckova formule}, která říká, 
že propagátor Gaussovského systému lze vyjádřit jen pomocí akce klasické trajektorie
\begin{equation}
G(\vector{x}_{\mathrm{f}}, t_{\mathrm{f}}; \vector{x}_{\mathrm{i}}, t_{\mathrm{i}})=
\sqrt{\det{\left(\frac{-1}{2\im\pi\hbar}\frac{\partial^{2}S_{\mathrm{kl}}}{\partial\vector{x}_{\mathrm{i}}\partial\vector{x}_{\mathrm{f}}}\right)}}\
\exp\left(\frac{\im}{\hbar}S_{\mathrm{kl}}\right).
\end{equation}
Ukažte, že Van Vleckova formule reprodukuje propagátor volné částice, propagátor částice v homogenním poli 
a propagátor částice v poli harmonického oscilátoru.
