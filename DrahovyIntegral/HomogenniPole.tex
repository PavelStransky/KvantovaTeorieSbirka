\subsection{Dráhový integrál částice v homogenním poli}
Mějme jednorozměrný případ částice pohybující se v homogenním poli.
Systém je popsaný Hamiltoniánem
\begin{equation}
\operator{H}=\frac{1}{2M}\operator{p}^{2}-F\operator{x}
\end{equation}

\begin{enumerate}
    \item Rozložme libovolnou trajektorii $x(t)$ s okrajovými podmínkami $x(t_{\mathrm{i,f}})=x_{\mathrm{i,f}}$ na 
    \begin{equation}
    x(t)=x_{\mathrm{kl}}(t)+x_{\mathrm{q}}(t),
    \end{equation}
    kde $x_{\mathrm{kl}}$ splňuje výše uvedené okrajové podmínky a pro $x_{\mathrm{q}}$ platí $x_{\mathrm{q}}(t_{\mathrm{i,f}})=0$.

    Nalezněte akci takovéto trajektorie pro případ obecného potenciálu $V(x)$.
    Využijte přitom toho integrace per partes a skutečnosti, že klasická trajektorie je řešením klasické pohybové rovnice.

    \item Nalezněte totéž pro případ homogenního pole a ukažte, že výraz pro trajektorii $x_{\mathrm{q}}(t)$ je stejný
    jako v případě volné částice.

    \item Nalezněte klasickou trajektorii $x_{kl}(t)$ částice pohybující se v homogenním poli
    při okrajových podmínkách $x_{\mathrm{kl}}(t_{\mathrm{i}})=x_{\mathrm{i}}$, $x_{\mathrm{kl}}(t_{\mathrm{f}})=x_{\mathrm{f}}$
    a spočítejte její akci.

    \item Napište výraz pro diferenciální operátor odpovidající matici $A$ z Gaussovského integrálu (\ref{GI}).

    Jelikož je tento operátor stejný jako v případě volné částice, bude normalizační faktor dráhového integrálu
    v obou případech stejný. Oba výrazy se budou lišit akcí v exponenciále.

    \item Vyjádřete finální tvar propagátoru.
\end{enumerate}