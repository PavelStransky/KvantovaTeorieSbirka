\subsection{Gaussovský integrál}
\begin{enumerate}
\item Nechť $\matrix{A}$ je reálná, symetrická, pozitivně definitní matice. 
Dokažte, že pak platí
\begin{equation}\label{GI}
\int\d^{d}\vector{x}\,\exp\left(-\frac{1}{2}\vector{x}^{T}\cdot\matrix{A}\cdot\vector{x}\right)=\left(\det{\frac{\matrix{A}}{2\pi}}\right)^{-\frac{1}{2}}
\end{equation}
($d$ je dimenze prostoru).

\item Nechť navíc $\vector{b}$ je libovolný vektor, $c$ libovolný skalár.
Označme $K(\vector{x})=\frac{1}{2}\vector{x}^{T}\cdot\matrix{A}\cdot\vector{x}+\vector{b}\cdot\vector{x}+c$.
V tomto případě se přesvědčte, že
\begin{equation}
\int\d^{d}\vector{x}\,\e^{-K(x)}=\left(\det{\frac{\matrix{A}}{2\pi}}\right)^{-\frac{1}{2}}\,\exp\left(\frac{1}{2}\vector{b}^{T}\cdot\matrix{A}^{-1}\cdot\vector{b}-c\right).
\end{equation}

\item Ukažte, že poslední uvedený integrál lze také vyjádřit takto:
\begin{equation}\label{GIstac}
\int\d^{d}\vector{x}\,\e^{-K(x)}=\left(\det{\frac{\matrix{A}}{2\pi}}\right)^{-\frac{1}{2}}\,\e^{-K_{\mathrm{stac}}},
\end{equation}
kde $K_{\mathrm{stac}}$ je stacionární bod výrazu $K(\vector{x})$, tj. bod, pro který $\nabla K(\vector{x})=0$.

Tento výsledek vlastně znamená, že počítáme-li propagátor systému s maximálně kvadratickou akcí, 
dá se vyjádřit jen pomocí akce klasické trajektorie (pro klasickou trajektorii je akce extremální).

\end{enumerate}

