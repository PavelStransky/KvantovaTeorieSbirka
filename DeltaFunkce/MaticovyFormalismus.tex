\subsection{Maticový formalismus}
Částice hmotnosti $M$ se pohybuje v jednorozměrném potenciálu složeném z $n$ různě silných $\delta$ funkcí,
\begin{equation}\label{PotentialSumDelta}
    V(x)=\sum_{j=1}^{n}c_{j}\,\delta(x-x_{j}),
\end{equation}
kde $c_{j}$ jsou jejich \uv{síly} a $x_{1}<x_{2}<\dotsb<x_{n}$ jejich polohy.

\begin{enumerate}
\item
    Obecnou vlnovou funkci pro částici v bodě $x$ zapište ve formě dvousložkového vektoru.
    
\item
    Nalezněte transformaci posunutí vlnové funkce o vzdálenost $a$ a zapište tuto transformaci ve formě matice.
    
\item
    Nalezněte transformaci vlnové funkce při průchodu $\delta$ funkcí v bodě $x_{j}$.
    
\item
    Nalezněte transformaci $\Xi$, která dává do vztahu vlnovou funkci před první jámou s vlnovou funkcí po poslední jámě.
    
\item
    Pro případ dvou stejně silných jam vzdálených od sebe $a$ najděte podmínku na to, aby vlnová funkce pro vázané stavy $E<0$ byla normalizovatelná.
\end{enumerate}
	
\begin{solution}
	\begin{enumerate}
	\item
		Až na jednotlivé body $x_{j}$ se jedná o vlnovou funkci volné částice
		\begin{equation}\label{eq:SumDeltaWaveFunction}
			\psi(x)
				=\underbrace{A_{+}\e^{\im k x}}_{\psi_{+}(x)}
					+\underbrace{A_{-}\e^{-\im k x}}_{\psi_{-}(x)},
		\end{equation}
		kde $k=\sqrt{2ME/\hbar^{2}}$ [pro vázané stavy $E<0$ je $k$ ryze imaginární, nebo je možné přejít k veličině $\kappa$~\eqref{eq:kappa}].
		Vlnová funkce se zapíše v kompaktním tvaru dvousložkového vektoru
		\begin{equation}
			\Psi(x)\equiv\makematrix{\psi_{+}(x) \\ \psi_{-}(x)}.
		\end{equation}
		
	\item
		Posunutá vlnová funkce je
		\begin{equation}
			\psi(x+a)
				=A_{+}\e^{\im kx+\im ka}+A_{-}\e^{-\im kx-\im ka}
				=\e^{+\im ka}\psi_{+}(x)+\e^{-\im ka}\psi_{-}(x),
		\end{equation}
		takže
		\begin{equation}
			\Psi(x+a)
				=\makematrix{\e^{+\im ka}\psi_{+}(x) \\ \e^{-\im ka}\psi_{-}(x)}
				=\underbrace{\makematrix{\e^{+\im ka} & 0 \\ 0 & \e^{-\im ka}}}_{\matrix{T}(a)}\Psi(x)\,.
		\end{equation}
		
	\item
		\trick{Sešívací podmínka~\eqref{eq:SewDerivativeDelta} pro vlnovou funkci~\eqref{eq:SumDeltaWaveFunction} je lineární, a proto ji lze rovněž vyjádřit maticovou transformací vektoru $\Psi(x)$.}
		Označí-li se vlnová funkce nalevo od $j$-té $\delta$ funkce jako $\Psi^{(L)}(x_{j})$ a vlnová funkce napravo od $\delta$ funkce jako $\Psi^{(R)}(x_{j})$, pak podmínka na spojitost a na skok v derivaci v bodě $x_{j}$ zní
		\begin{subequations}
			\begin{align}
				\psi_{+}^{(L)}+\psi_{-}^{(L)}
					&=\psi_{+}^{(R)}+\psi_{-}^{(R)},\\
				\im k\left(\psi_{+}^{(R)}-\psi_{-}^{(R)}+\psi_{+}^{(L)}-\psi_{-}^{(L)}\right)
					&=K_{j}\left(\psi_{+}^{(L,R)}+\psi_{-}^{(L,R)}\right).
			\end{align}
		\end{subequations}
		Z první rovnice plyne $\psi_{-}^{(R)}=\psi_{+}^{(L)}+\psi_{-}^{(L)}-\psi_{+}^{(R)}$, což po dosazení do druhé rovnice dá
		\begin{subequations}
			\begin{align}
				\psi_{+}^{(R)}
					&=\frac{K_{j}}{2\im k}\left(\psi_{+}^{(L)}+\psi_{-}^{(L)}\right)+\psi_{+}^{(L)}
					=\left(1+\frac{K_{j}}{2\im k}\right)\psi_{+}^{(L)}+\frac{K_{j}}{2\im k}\psi_{-}^{(L)},\\
				\psi_{+}^{(L)}
					&=-\frac{K_{j}}{2\im k}\left(\psi_{+}^{(L)}+\psi_{-}^{(L)}\right)+\psi_{-}^{(L)}
					=-\frac{K_{j}}{2\im k}\psi_{+}^{(L)}+\left(1-\frac{K_{j}}{2\im k}\right)\psi_{-}^{(L)}.
			\end{align}
		\end{subequations}
		To je lineární transformace, kterou lze zapsat jako
		\begin{equation}
			\Psi^{(R)}(x_{j})
				=\underbrace{\makematrix{1+\frac{K_{j}}{2\im k} & \frac{K_{j}}{2\im k} 
								\\ -\frac{K_{j}}{2\im k} & 1-\frac{K_{j}}{2\im k}}}_{\matrix{R}\left(K_{j}\right)}
					\Psi^{(L)}(x_{j}).
		\end{equation}
		
		\item
			Transformace $\Xi$ je složením transformací posunutí mezi $\delta$ funkcemi $\matrix{T}$ 
			a transformací přechodu přes $\delta$ funkce $\matrix{R}$:
			\begin{equation}\label{eq:SumDeltaTransformationMatrix}
				\Xi(x_{n};x_{1})
					=\matrix{R}(K_{n})\matrix{T}(x_{n}-x_{n-1})\matrix{R}(K_{n-1})\dotsb\matrix{R}(K_{2})\matrix{T}(x_{2}-x_{1})\matrix{R}(K_{1}).
			\end{equation}			
			
		\item
			Transformace~\eqref{eq:SumDeltaTransformationMatrix} má pro dvě $\delta$ funkce tvar
			\begin{align}
				\Xi(a)
					&=\matrix{R}(K)\matrix{T}(a)\matrix{R}(K)\nonumber\\
					&=\makematrix{1+\frac{K}{2\pi k} & \frac{K}{2\im k} 
						\\ -\frac{K}{2\im k} & 1-\frac{K}{2\im k}}
						\makematrix{\e^{\im ka} & 0 \\ 0 & \e^{-\im ka}}
						\makematrix{1+\frac{K}{2\pi k} & \frac{K}{2\im k} 
						\\ -\frac{K}{2\im k} & 1-\frac{K}{2\im k}}\nonumber\\
					&=\makematrix{\left(1+\frac{K}{2\im k}\right)^{2}\e^{\im ka}
								-\left(\frac{K}{2ik}\right)^{2}\e^{-\im ka} &
							\frac{K}{2\im k}\left[\left(1+\frac{K}{2\im k}\right)\e^{\im ka}
								+\left(1-\frac{K}{2\im k}\right)\e^{-\im ka}\right] \\
							-\frac{K}{2\im k}\left[\left(1+\frac{K}{2\im k}\right)\e^{\im ka}
								+\left(1-\frac{K}{2\im k}\right)\e^{-\im ka}\right] &
							-\left(\frac{K}{2ik}\right)^{2}\e^{\im ka}
							+\left(1-\frac{K}{2\im k}\right)^{2}\e^{\im ka}}.
			\end{align}
			Stav bude normalizovatelný, pokud $\psi_{+}^{(L)}=\psi_{-}^{(R)}=0$, kde $\psi_{+}^{(L)}$ zde značí část vlnové funkce s kladným znaménkem nalevo před levou $\delta$ funkcí a $\psi_{-}^{(R)}$ část vlnové funkce napravo za pravou $\delta$ funkcí potenciálu.\sfootnote{
				Pro vázané stavy je $k=\im\sqrt{-2ME/\hbar}=\im\kappa$.
			}
			Jelikož $\Psi^{(R)}=\Xi\Psi^{(L)}$, platí
			\begin{equation}
                \psi_{-}^{(R)}
                    =\Xi_{21}\psi_{+}^{(L)}+\Xi_{22}\psi_{-}^{(L)},
			\end{equation}
			a pokud se kvůli správnému asymptotickému chování má vynulovat $\psi_{+}^{(L)}$ a $\psi_{-}^{(R)}$, musí být $\Xi_{22}=0$, neboli (po zavedení $k=\im\kappa$)
			\begin{equation}
				-\left(\frac{K}{2\kappa}\right)^{2}\e^{-\kappa a}
					+\left(1+\frac{K}{2\kappa}\right)^{2}\e^{\kappa a}=0,
			\end{equation}
			\begin{subequations}
				\begin{align}
					K^{2}\e^{-2\kappa a}
						&=\left(K+2\kappa\right)^{2},\\
					\abs{K}\e^{-\kappa a}
						&=\pm\left(K+2\kappa\right).
				\end{align}					
			\end{subequations}
			Rovnice je identická s rovnicemi~\eqref{eq:DoubleDeltaEEven} a ~\eqref{eq:DoubleDeltaEOdd}.
			Jejím řešením bychom dostali energie vázaných stavů.
	\end{enumerate}
\end{solution}
