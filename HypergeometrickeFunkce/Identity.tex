\sec{Užitečné identity}
Pro hypergeometrickou funkci je známo tisíce identit, které lze najít například na stránkách 
\href{http://functions.wolfram.com/HypergeometricFunctions/Hypergeometric2F1/17/ShowAll.html}
{Wolfram} 
nebo \href{http://dlmf.nist.gov/15}{Digital Library of Mathematical Functions}.
Pro následující příklady jsou nejdůležitější identity tyto,
\begin{subequations}
    \begin{align}
        \2F1{a}{b}{c}{z}
            &=\frac{\Gamma(c)\Gamma(c-a-b)}{\Gamma(c-a)\Gamma(c-b)}
                \2F1{a}{b}{a+b-c+1}{1-z}\nonumber\\
        \label{eq:2F1Identity1}
            &\!\!\!\!\!\!\!\!\!\!\!\!\!\!\!\!\!\!\!\!
                +\frac{\Gamma(c)\Gamma(a+b-c)}{\Gamma(a)\Gamma(b)}\left(1-z\right)^{c-a-b}
                \2F1{c-a}{c-b}{c-a-b+1}{1-z},\quad c-a-b\notin\mathbb{Z},\\
        \2F1{a}{b}{c}{z}
            &=\frac{\Gamma(c)\Gamma(b-a)}{\Gamma(b)\Gamma(c-a)}\left(-z\right)^{-a}
                \2F1{a}{a-c+1}{a-b+1}{\frac{1}{z}}\nonumber\\
        \label{eq:2F1Identity3}
            &+\frac{\Gamma(c)\Gamma(a-b)}{\Gamma(a)\Gamma(c-b)}\left(-z\right)^{-b}
                \2F1{b}{b-c+1}{b-a+1}{\frac{1}{z}},\quad a-b\notin\mathbb{Z},\ z\notin(0,1),\\
        \label{eq:2F1Identity2}
        \2F1{a}{b}{c}{z}
            &=\left(1-z\right)^{-a}\2F1{a}{c-b}{c}{\frac{z}{z-1}},\quad z\notin(1,\infty),
    \end{align}
\end{subequations}
přičemž první dvě mohou sloužit k prodlužování řešení hypergeometrické funkce
za konvergenční oblast $\abs{z}<1$.
Druhá formule navíc určuje asymptotiku hypergeometrické funkce v nekonečnu,
\begin{equation}
    \label{eq:2F1Asymptotic}
    \2F1{a}{b}{c}{z\rightarrow\infty}\rightarrow
        \frac{\Gamma(c)\Gamma(b-a)}{\Gamma(b)\Gamma(c-a)}(-z)^{-a}+
        \frac{\Gamma(c)\Gamma(a-b)}{\Gamma(a)\Gamma(c-b)}(-z)^{-b}\quad a-b\notin\mathbb{Z}.
\end{equation}
