\subsection{Nekonečně hluboká nepravoúhlá jáma}
Určete energetické spektrum částice o hmotnosti $M$ nacházející se v jednorozměrném potenciálu
\begin{equation}
	V(x)=\frac{V_{0}}{\tan^{2}\frac{\pi x}{a}}\,,\qquad 0<x<a\,,
\end{equation}
kde parametr $V_{0}>0$ určuje strmost a $a>0$ šířku jámy.
Nalezněte energii základního stavu a normalizovanou vlnovou funkci.

\begin{solution}
	Cílem je převést Schrödingerovu rovnici pro vlnovou funkci $\psi=\psi(x)$
	\begin{equation}
		\label{eq:Tan2Schrodinger}
		-\frac{\hbar^{2}}{2M}\psi_{xx}+\left[V(x)-E\right]\psi=0
	\end{equation}
	(značíme zde $\psi_{x}\equiv\d\psi/\d x$) do tvaru pro hypergeometrickou funkci~\eqref{eq:HypergeometricEquation}.
	K tomu je třeba provést postupně několik substitucí.
	
	\begin{enumerate}
	\item
		Nejprve zavedeme bezrozměrnou proměnnou souřadnice
		\begin{equation}
			\label{eq:Tan2Xi}
			\xi\equiv\frac{\pi x}{a}
		\end{equation}
		a energie
		\begin{equation}
			\label{eq:Tan2Energy}
			\qquad\epsilon=\frac{\pi^{2}\hbar^{2}}{2Ma^{2}}\,,
			\qquad e\equiv\frac{E}{\epsilon}\,,
			\qquad v_{0}\equiv\frac{V_{0}}{\epsilon}\,,
			\qquad v(\xi)\equiv\frac{V(\xi)}{\epsilon}=\frac{v_{0}}{\tan^{2}\xi}\,,
		\end{equation}		
		čímž převedeme Schrödingerovu rovnici~\eqref{eq:Tan2Schrodinger} na tvar
		\begin{equation}
			\label{eq:Tan2SchrodingerXi}
			\psi_{\xi\xi}+\left[e-v(\xi)\right]\psi=0\,.
		\end{equation}
	
	\item
		Vlnovou funkci budeme hledat ve tvaru
		\begin{equation}
			\label{eq:Tan2u}
			\psi(\xi)=u(\xi)\sin^{\alpha}\xi\,,
		\end{equation}
		kde $\alpha$ bude prozatím volný parametr.
		Zavedeme zjednodušené označení
		\begin{equation}
			s\equiv\sin{\xi}\,,\qquad c\equiv\cos{\xi}\,.
		\end{equation}
		První a druhá derivace vlnové funkce pak jsou
		\begin{subequations}
			\begin{align}
				\psi_{\xi}
					&=u_{\xi}\,s^{\alpha}+\alpha\,u\,s^{\alpha-1}c\,,\\
				\psi_{\xi\xi}
					&=u_{\xi\xi}\,s^{\alpha}+2\alpha\,u_{\xi}\,s^{\alpha-1}c
						+\alpha(\alpha-1)\,u\,s^{\alpha-2}\underbrace{c^{2}}_{1-s^{2}}
						-\alpha\,u\,s^{\alpha}\nonumber\\
					&=u_{\xi\xi}\,s^{\alpha}+2\alpha\,u_{\xi}\,s^{\alpha-1}c
						+\alpha(\alpha-1)\,u\,s^{\alpha-2}
						-\alpha^{2}\,u\,s^{\alpha}\,.
			\end{align}				
		\end{subequations}
		Potenciál rozepíšeme jako
		\begin{equation}
			v(x)
				=v_{0}\frac{c^{2}}{s^{2}}
				=v_{0}\frac{1-s^{2}}{s^{2}}
				=\frac{v_{0}}{s^{2}}-v_{0}\,.
		\end{equation}
		Schrödingerova rovnice~\eqref{eq:Tan2SchrodingerXi} pak zní
		\begin{equation}
			u_{\xi\xi}\,s^{\alpha}
				+2\alpha\,u_{\xi}\,s^{\alpha-1}c
				-\alpha^{2}\,u\,s^{\alpha}
				+\underbrace{\left(e+v_{0}\right)}_{\nu^{2}}u\,s^{\alpha}
				+\left[\alpha(\alpha-1)-v_{0}\right]u\,s^{\alpha}
				-v_{0}\,u\,s^{\alpha-2}=0\,,
		\end{equation}
		a po vydělení výrazem $s^{\alpha}$ a po přerovnání členů dostaneme
		\begin{equation}
			u_{\xi\xi}
				+2\alpha\,u_{x}\,\frac{c}{s}
				+\left(\nu^{2}-\alpha^{2}\right)u
				+\left[\alpha(\alpha-1)-v_{0}\right]u\,s^{-2}=0\,.
		\end{equation}
		
		Konstanta $\alpha$ byla doposud libovolné číslo.
		Nyní ji zvolíme tak, aby vynulovala člen úměrný $s^{-2}$.
		Řešíme tedy rovnici
		\begin{equation}
			\alpha(\alpha-1)=v_{0}\,,
		\end{equation}
		jejíž kořeny jsou
		\begin{equation}
			\label{eq:Tan2Alpha}
			\alpha_{\pm}=\frac{1}{2}\left(1\pm\sqrt{4v_{0}+1}\right)\,.
		\end{equation}
		V následujícím výpočtu budeme uvažovat pouze kořen $\alpha\equiv\alpha_{-}<0$.
		
		V této fázi dostáváme tedy Schrödingerovu rovnici ve tvaru
		\begin{equation}
		\label{eq:Tan2SchrodingerUXi}
			u_{\xi\xi}
				+2\alpha\,u_{x}\,\frac{c}{s}
				+\left(\nu^{2}-\alpha^{2}\right)u=0\,.
		\end{equation}		
		
	\item
		Zavedeme novou nezávislou proměnnou\footnote{
			Správně bychom měli psát substituci ve tvaru
			\begin{equation}
				\sqrt{z}=\cos\xi\,,
			\end{equation}
			což je narozdíl od~\eqref{eq:Tan2z} prosté zobrazení převádějící $\xi\in(0,\pi)$
			na $\sqrt{z}\in(-1,1)$.
		}
		\begin{equation}
			\label{eq:Tan2z}
			z=\cos^{2}\xi=c^{2}\,,
		\end{equation}
		Jelikož
		\begin{subequations}
			\begin{align}
				u_{\xi}
					&\equiv\derivative{u}{\xi}=\derivative{u}{z}\derivative{z}{\xi}=u_{z}z_{\xi}\,,\\
				u_{\xi\xi}
					&\equiv\derivative[2]{u}{\xi}=\derivative[2]{u}{z}\left(\derivative{z}{\xi}\right)^{2}+\derivative{u}{z}\derivative[2]{z}{\xi}
						=u_{zz}z_{\xi}^{2}+u_{z}z_{\xi\xi}
			\end{align}				
		\end{subequations}
		a
		\begin{subequations}
			\begin{align}
				z_{\xi}&=-2sc\,,\\
				z_{\xi\xi}&=-2\left(c^{2}-s^{2}\right)\,,
			\end{align}				
		\end{subequations}
		dostaneme Schrödingerovu rovnici~\eqref{eq:Tan2SchrodingerUXi} do tvaru
		\begin{equation}
			4\,u_{zz}\underbrace{s^{2}}_{1-z}\underbrace{c^{2}}_{z}
				-2\,u_{z}\underbrace{\left(c^{2}-s^{2}\right)}_{2z-1}
				-4\alpha\,u_{z}\underbrace{c^{2}}_{z}
				+\left(\nu^{2}-\alpha^{2}\right)u=0\,,
		\end{equation}
		\begin{equation}
			\label{eq:Tan2SchrodingerFinal}
			z(1-z)u_{zz}
				+\left[\frac{1}{2}-\left(1+\alpha\right)z\right]u_{z}
				+\frac{1}{4}\left(\nu^{2}-\alpha^{2}\right)=0\,.
		\end{equation}
		Toto je rovnice pro hypergeometrické funkce~\eqref{eq:HypergeometricEquation} s hodnotami
		parametrů
		\begin{subequations}
			\begin{align}
				a+b&=\alpha\,,\\
				ab&=\frac{1}{4}\left(\nu^{2}-\alpha^{2}\right)\,,\\
				c&=\frac{1}{2}\,,
			\end{align}				
		\end{subequations}
		kde z prvních dvou rovnic obdržíme\footnote{
			Druhé možné řešení této soustavy rovnic je
			\begin{subequations}
				\begin{align}
					a&=\frac{1}{2}\left(\alpha+\nu\right)\,,\\
					b&=\frac{1}{2}\left(\alpha-\nu\right)\,,
				\end{align}					
			\end{subequations}
			které však s ohledem na symetrii hypergeometrických funkcí vůči záměně prvního
			a druhého argumentu dá totožné řešení diferenciální rovnice~\eqref{eq:Tan2SchrodingerFinal}.
		}
		\begin{subequations}
			\begin{align}
				a&=\frac{1}{2}\left(\alpha-\nu\right)\,,\\
				b&=\frac{1}{2}\left(\alpha+\nu\right)\,.
			\end{align}				
		\end{subequations}
	\end{enumerate}
	
	Dvě lineárně nezávislá řešení diferenciální rovnice~\eqref{eq:Tan2SchrodingerFinal} jsou
	dány hypergeometrickými funkcemi
	\begin{subequations}
		\begin{align}
			u_{1}(z)
				&=\2F1{\frac{1}{2}\left(\alpha-\nu\right)}{\frac{1}{2}\left(\alpha+\nu\right)}
					{\frac{1}{2}}{z}\,,\\
			u_{2}(z)
				&=\sqrt{z}\2F1{\frac{1}{2}(\alpha-\nu+1)}{\frac{1}{2}(\alpha+\nu+1)}{\frac{3}{2}}{z}\,.
		\end{align}			
	\end{subequations}
	
	Nyní budeme studovat konvergenci řešení.
	Požadujeme, aby v bodě $z=1$, který odpovídá $x=a$, byla vlnová funkce $\psi$ nulová.
	K tomu využijeme identitu~\eqref{eq:2F1Identity2} postupně na obě dvě lineárně nezávislá řešení.
	
	\begin{itemize}
	\item Řešení $u_{1}$:
		\begin{equation}
			u_{1}(z)
				=(1-z)^{\frac{1}{2}(\nu-\alpha)}
					\2F1{\frac{1}{2}(\alpha-\nu)}{\frac{1}{2}(1-\alpha-\nu)}{\frac{1}{2}}
						{\frac{z}{z-1}}\,.
		\end{equation}
		Po uvážení substitucí~\eqref{eq:Tan2u} a~\eqref{eq:Tan2z} 
		\begin{align}
			\psi_{1}
				&=s^{\alpha}s^{\nu-\alpha}
					\2F1{\frac{1}{2}(\alpha-\nu)}{\frac{1}{2}(1-\alpha-\nu)}
						{\frac{1}{2}}{\frac{c^{2}}{-s^{2}}}\nonumber\\
				&=s^{\nu}\sum_{m=0}^{\infty}(-1)^{m}\frac{c_{m}}{m!}\frac{c^{2m}}{s^{2m}}\,.
		\end{align}
		Pokud nemá řada konečný počet členů, diverguje v bodě $x=0$ a $x=a$.
		Abychom divergenci zabránili, musí být buď první, nebo druhý argument hypergeometrické
		funkce rovný nekladnému celému číslu.
		Pokud
		\begin{align}
			\frac{1}{2}(\alpha-\nu)&=-n\,,\ n\in\mathbb{N}_{0}
				&\Longrightarrow&&\nu&=2n+\alpha<2n
		\end{align}
		(zvolili jsme $\alpha<0$)	takže v tomto případě vlnová funkce konvergovat nebude
		(v konečné řadě bude vždy víc členů s funkcí $\sin\xi$ ve jmenovateli, 
		než kolik je mocnina v prefaktoru).
		Naopak pokud uvážíme druhý argument hypergeometrické funkce,
		\begin{align}
			\frac{1}{2}(1-\alpha-\nu)&=-n\,,\ n\in\mathbb{N}_{0}
				&\Longrightarrow&&\nu&=2n-\alpha+1>2n\,,
		\end{align}
		dostaneme dobře konvergující řešení.
		
		Dosadíme-li nyní za $\nu=\sqrt{e+v_{0}}$, můžeme vyjádřit energii jako
		\begin{equation}
			\label{eq:Tan2Odd}
			E_{n,1}
				=\epsilon\left[(2n+1)^{2}-2\alpha(2n+1)+\alpha\right]\,,\quad n\in\mathbb{N}_{0}\,,
		\end{equation}
		kde $\alpha$ je záporný kořen~\eqref{eq:Tan2Alpha}\footnote{
			Pokud bychom zvolili kladný kořen~\eqref{eq:Tan2Alpha}, $\alpha_{+}=1-\alpha_{-}$,
			dobře se chovající řešení by bylo dáno z prvního argumentu hypergeometrické funkce,
			avšak spektrum bychom dostali zcela stejné (a rovněž i vlnovou funkci).
		}
		a $\epsilon$ je energetická škála~\eqref{eq:Tan2Energy}.
		Odpovídající nenormovaná vlnová funkce v proměnné $x$ zní
		\begin{equation}
			\psi_{n,1}(x)=\sin^{\alpha}\frac{\pi x}{a}
				\2F1{\alpha-n-\frac{1}{2}}{n+\frac{1}{2}}{\frac{1}{2}}
				{\cos^{2}\frac{\pi x}{a}}\,.
		\end{equation}
	
	\item Řešení $u_{2}$:
	
		Postupujeme analogicky jako u $u_{1}$.
		Aplikací identity~\eqref{eq:2F1Identity2} postupně dostaneme
		\begin{equation}
			u_{2}(z)=\sqrt{z}(1-z)^{\frac{1}{2}(\nu-\alpha-1)}
				\2F1{\frac{1}{2}(\alpha-\nu+1)}{1-\frac{1}{2}(\alpha+\nu)}{\frac{3}{2}}
				{\frac{z}{z-1}}\,,
		\end{equation}
		\begin{align}
			\psi_{2}
				&=s^{\alpha}cs^{\nu-\alpha-1}
					\2F1{\frac{1}{2}(\alpha-\nu+1)}{1-\frac{1}{2}(\alpha+\nu)}{\frac{3}{2}}
					{\frac{c^{2}}{-s^{2}}}\nonumber\\
				&=cs^{\nu-1}\sum_{m=0}^{\infty}(-1)^{m}\frac{c_{m}}{m!}\frac{c^{2m}}{s^{2m}}\,,
		\end{align}
		kde ukončení řady pomocí druhého parametru hypergeometrické funkce vede k řešení
		s požadovanými okrajovými podmínkami,
		\begin{align}
			1-\frac{1}{2}(\alpha+\nu)&=-n\,,\ n\in\mathbb{N}_{0}
			&\Longrightarrow&&\nu=2n+2-\alpha\,,
		\end{align}
		což vede na kvantovací podmínku pro energie
		\begin{equation}
			\label{eq:Tan2Even}
			E_{n,2}=\epsilon\left[(2n+2)^{2}-2\alpha(2n+2)+\alpha\right]\,,\quad n\in\mathbb{N}_{0}
		\end{equation}
		a vlnové funkce ve tvaru
		\begin{equation}
		\psi_{n,2}(x)=\sin^{\alpha}\frac{\pi x}{a}\cos\frac{\pi x}{a}
			\2F1{\alpha-n-\frac{1}{2}}{n+\frac{3}{2}}{\frac{3}{2}}{\cos^{2}\frac{\pi x}{a}}\,.
		\end{equation}
	\end{itemize}
	
	Řešení~\eqref{eq:Tan2Odd} a~\eqref{eq:Tan2Even} lze zkombinovat dohromady do
	\begin{equation}
		E_{m}=\epsilon\left[m^{2}-2\alpha m+\alpha\right]\,,\quad m\in\mathbb{N}\,,
	\end{equation}
	přičemž pro $m$ liché je vlnová funkce ve tvaru $\psi_{n=(m-1)/2,1}(x)$ a pro $m$ sudé
	$\psi_{n=m/2-1,2}(x)$.
	
	Energie základního stavu je
	\begin{equation}
		E_{1}=\frac{\pi^{2}\hbar^{2}}{4Ma^{2}}
			\left(1-\sqrt{\frac{8Ma^{2}V_{0}}{\pi^{2}\hbar^{2}}+1}\right)
	\end{equation}
	a odpovídající vlnová funkce
	\begin{align}
		\psi_{1}(x)\equiv\psi_{0,1}(x)
			&=N_{1}\sin^{\alpha}\frac{\pi x}{a}
				\underbrace{\2F1{\alpha-\frac{1}{2}}{\frac{1}{2}}{\frac{1}{2}}{z}}
					_{(1-z)^{\frac{1}{2}-\alpha}}\nonumber\\
			&=N_{1}\sin^{1-\alpha}\frac{\pi x}{a}\,,
	\end{align}
	kde jsme využili znalosti~\eqref{eq:2F1Fraction}.
	Normalizační faktor $N_{1}$ získáme integrací
	\begin{equation}
		N_{1}
			=\frac{1}{\sqrt{\int_{0}^{a}\sin^{2-2\alpha}\frac{\pi x}{a}\d x}}
			=\sqrt{\frac{\pi}{a}\frac{\Gamma(1-\alpha)}
				{\Gamma\left(\frac{1}{2}\right)\Gamma\left(\frac{3}{2}-\alpha\right)}}\,.
	\end{equation}
	
	\begin{note}
		Pokud $V_{0}\rightarrow0$, dostaneme $\alpha\rightarrow0$ a výraz pro energie se zjednoduší
		na
		\begin{equation}
			E_{m}=\frac{\pi^{2}\hbar^{2}}{2Ma^{2}}m^{2}\,,\quad m\in\mathbb{N}\,,
		\end{equation}
		což jsou energie pravoúhlé jámy.
		
		Naopak pro $V_{0}\rightarrow\infty$ je $\alpha\rightarrow-\infty$ a v tom případě dostaneme
		\begin{equation}
			E_{m}=\hbar\Omega\left(m-\frac{1}{2}\right)\,,
				\quad\Omega\equiv\frac{\pi}{a}\sqrt{\frac{2V_{0}}{M}}\,,
		\end{equation}
		energie harmonického oscilátoru aproximující dno jámy.
	\end{note}
\end{solution}
