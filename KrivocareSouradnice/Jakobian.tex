\sec{Jakobián}
	Přechod mezi jednotlivými souřadnými systémy $\vector{x}$ a $\vector{q}$ je dán~\emph{Jacobiho maticí}\index{matice!Jacobiho}
	\begin{equation}
		\matrix{J}=\left(\partialderivative{\vector{x}}{\vector{q}}\right)
			=\left(\partialderivative{\left(x^{1},x^{2},\dotsc,x^{d}\right)}
            {\left(q^{1},q^{2},\dotsc,q^{d}\right)}\right)
            =\makematrix{
                \partialderivative{x^{1}}{q^{1}} & \partialderivative{x^{1}}{q^{2}} & \partialderivative{x^{1}}{q^{3}} & \\
                \partialderivative{x^{2}}{q^{1}} & \partialderivative{x^{2}}{q^{2}} & \partialderivative{x^{2}}{q^{3}} & \ldots \\
                \partialderivative{x^{3}}{q^{1}} & \partialderivative{x^{3}}{q^{2}} & \partialderivative{x^{3}}{q^{3}} & \\
                & \vdots & & \ddots
            }
	\end{equation}
	či ve složkách
	\begin{equation}
		\label{eq:JacobiMatrix}
		J_{i}^{j}=\partialderivative{x^{j}}{q^{i}}\equiv n_{(i)}^{j}\,.
	\end{equation}
	(Jacobiho matice má tedy ve sloupcích kovariantní bázové vektory $\vector{n}_{(i)}$.)
	Podmínka invertovatelnosti~\eqref{eq:xq} souvisí s nesingularitou Jacobiho matice,
	\begin{equation}
		\det{\matrix{J}}\neq 0
	\end{equation}
	($\det{\matrix{J}}$ se nazývá \emph{Jakobián}\index{Jakobián}).
