\sec{Divergence}
Divergenci v křivočarých souřadnicích odvodíme z kovariantní derivace:
\begin{equation}
	\divergence{\vector{F}}
		={F^{k}}_{;k}
		=\partialderivative{F^{k}}{q^{k}}+\Gamma^{k}_{kj}F^{j}.
\end{equation}
Zúžený Christoffelův symbol je
\begin{equation}
	\Gamma^{k}_{kj}
		=\frac{1}{2}g^{kl}\left(\partialderivative{g_{kl}}{q^{j}}
			+\underbrace{\partialderivative{g_{jl}}{q^{k}}-\partialderivative{g_{kj}}{q^{l}}}_{0}\right)
		=\frac{1}{2}g^{kl}\partialderivative{g_{kl}}{q^{j}}
		=\frac{1}{2}\trace{\matrix{g}^{-1}\partialderivative{\matrix{g}}{q^{j}}}
\end{equation}
(druhý a třetí člen závorky jsou antisymetrické vůči záměně $k\leftrightarrow l$, 
takže při násobení symetrickou maticí $g^{kl}$ dají nulu)
a použití vztahu~\eqref{eq:LogDet} dá
\begin{equation}
	\Gamma^{k}_{kj}
		=\frac{1}{2}\partialderivative{}{q^{j}}\ln{\det{\matrix{g}}}
		=\frac{1}{2\det{\matrix{g}}}\partialderivative{}{q^{j}}\det{\matrix{g}}
		\left(=\partialderivative{}{q^{j}}\ln{\sqrt{\det{\matrix{g}}}}\right),
\end{equation}
takže
\begin{equation}
	\label{eq:Divergence}
	\divergence{\vector{F}}
		=\partialderivative{F^{k}}{q^{k}}+\left(\frac{1}{2\det{\matrix{g}}}\partialderivative{}{q^{j}}\det{\matrix{g}}\right)F^{j}
		=\frac{1}{\sqrt{\det{\matrix{g}}}}\partialderivative{}{q^{k}}\sqrt{\det{\matrix{g}}}\,F^{k}\,,
\end{equation}
přičemž poslední rovnost se dokáže rozderivováním součinu 
(jak $\det{\matrix{g}}$, tak $F^{k}$ závisejí na $\vector{q}$).
