\subsection{Sférické souřadnice}
	\label{sec:SphericalCoordinates}
	Pro sférické souřadnice $\left(q^{1},q^{2},q^{3}\right)\equiv(r,\theta,\phi)$ definované běžným způsobem jako
	\begin{align}
		x&=r\sin\theta\cos\phi\nonumber\\
		y&=r\sin\theta\sin\phi\\
		z&=r\cos\theta\nonumber
	\end{align}
	vypočítejte:
	\begin{enumerate}
	\item 
		Jacobiho matici $\matrix{J}\equiv\frac{\partial(x,y,z)}{\partial(r,\theta,\phi)}$.

	\item 
		Determinant Jacobiho matice $\det\matrix{J}$.

	\item 
		Metrický tenzor $\matrix{g}$.

	\item 
		Inverzní metrický tenzor $\matrix{g}^{-1}$.

	\item 
		Determinant metrického tenzoru $\det\matrix{g}$.

	\item
		Lagranžián a Hamiltonián klasického systému bez potenciálu.

	\item 
		Laplaceův operátor $\Delta$.
	\end{enumerate}
	
\begin{solution}
	\begin{enumerate}
	\item
		Jacobiho matice podle definičního vztahu~\eqref{eq:JacobiMatrix}:
		\begin{equation}
			\matrix{J}=\makematrix{
                \sin{\theta}\cos{\phi} & r\cos{\theta}\cos{\phi} & -r\sin{\theta}\sin{\phi} \\
                \sin{\theta}\sin{\phi} & r\cos{\theta}\sin{\phi} & r\sin{\theta}\cos{\phi} \\
                \cos{\theta}  & -r\sin{\theta} & 0}.
		\end{equation}
		
	\item
		Jakobián
		\begin{align}
			\det{\matrix{J}}
				&=r^{2}\sin{\theta}\cos^{2}{\theta}\cos^{2}{\phi}+r^{2}\sin^{3}\theta\sin^{2}\phi\nonumber\\
				&+r^{2}\sin{\theta}\cos^{2}{\theta}\sin^{2}{\phi}+r^{2}\sin^{3}\theta\cos^{2}\phi\nonumber\\
				&=r^{2}\sin{\theta}\cos^{2}\theta+r^{2}\sin^{3}\theta\nonumber\\
				&=r^{2}\sin{\theta}.
		\end{align}
	
	\item
		Složky metrického tenzoru na základě~\eqref{eq:MetricTensorJacobi}:
		\begin{subequations}
			\begin{align}
				g_{11}
					&=\sin^{2}\theta\cos^{2}\phi+\sin^{2}\theta\cos^{2}\phi+\cos^{2}\theta
					 =\sin^{2}\theta+\cos^{2}\theta
					 =1,\\
				g_{12}
					&=r\sin\theta\cos\theta\cos^{2}\phi+r\sin\theta\cos\theta\sin^{2}\phi
						-r\sin\theta\cos\theta
					 =0,\\
				g_{13}
					&=-r\sin^{2}\theta\sin\phi\cos\phi+r\sin^{2}\theta\sin\phi\cos\phi
					 =0,\\
				g_{21}
					&=g_{12}
					 =0,\\
				g_{22}
					&=r^{2}\cos^{2}\theta\cos^{2}\phi+r^{2}\cos^{2}\theta\sin^{2}\phi+r^{2}\sin^{2}\theta
					 =r^{2},\\
				g_{23}
					&=r^{2}\sin\theta\cos\theta\sin\phi\cos\phi-r^{2}\sin\theta\cos\theta\sin\phi\cos\phi
					 =0,\\
				g_{31}
					&=g_{13}
					 =0,\\
				g_{32}
					&=g_{23}
					 =0,\\
				g_{33}
					&=r^{2}\sin^{2}\theta\sin^{2}\phi+r^{2}\sin^{2}\theta\cos^{2}\phi
					 =r^{2}\sin^{2}\theta.
			\end{align}				
		\end{subequations}
		V maticovém zápisu
		\begin{equation}
			\matrix{g}=\makematrix{
                1 & 0 & 0 \\
				0 & r^{2} & 0 \\
				0 & 0 & r^{2}\sin^{2}\theta}.
		\end{equation}
		
	\item
		Metrický tenzor je diagonální\sfootnote{Je-li metrický tenzor diagonální, 
		znamená to, že zobecněné souřadnice jsou v každém místě prostoru ortogonální.},
		takže inverzní matice je dána prostou inverzí diagonálních elementů,
		\begin{equation}
			\matrix{g}^{-1}=\makematrix{
                1 & 0 & 0 \\
				0 & \frac{1}{r^{2}} & 0 \\
				0 & 0 & \frac{1}{r^{2}\sin^{2}\theta}}.
		\end{equation}
	
	\item
		Determinant diagonální matice = součin členů na diagonále
		\begin{equation}
			\det{\matrix{g}}=r^{4}\sin^{2}\theta\,.
		\end{equation}
	
	\item
		Lagranžián je dán vztahem~\eqref{eq:LagrangianCurvilinear}
		\begin{equation}
			L=\frac{1}{2}M\left(\dot{r}^{2}+\dot{\theta}^{2}r^{2}
				+\dot{\phi}^{2}r^{2}\sin^{2}\theta\right)
		\end{equation}
		a Hamiltonián~\eqref{eq:HamiltonianCurvilinear}
		\begin{equation}
			H=\frac{1}{2M}\left(p_{r}^{2}+\frac{1}{r^{2}}p_{\theta}^{2}
				+\frac{1}{r^{2}\sin^{2}\theta}p_{\phi}^{2}\right).
		\end{equation}
	
	\item
		Laplaceův operátor je dán vztah~\eqref{eq:Laplace}:
		\begin{align}
			\Delta
				&=\frac{1}{r^{2}\sin\theta}\left[\partialderivative{}{r}r^{2}\sin\theta\partialderivative{}{r}
					+\partialderivative{}{\theta}r^{2}\sin\theta\frac{1}{r^{2}}\partialderivative{}{\theta}
					+\partialderivative{}{\phi}r^{2}\sin\theta\frac{1}{r^{2}\sin^{2}\theta}\partialderivative{}{\phi}\right]\nonumber\\
				&=\frac{1}{r^{2}}\partialderivative{}{r}r^{2}\partialderivative{}{r}
					+\frac{1}{r^{2}\sin\theta}\partialderivative{}{\theta}\sin\theta\partialderivative{}{\theta}
					+\frac{1}{r^{2}\sin^{2}\theta}\partialderivative[2]{}{\phi}.
        \end{align}	
        
    \item 
        Srovnání posledních dvou výrazů vede na vyjádření komponent hybnosti~\cite{Essen1978},
		\begin{subequations}
			\begin{align}
				p_{r}&=-\im\hbar\frac{1}{r}\partialderivative{}{r}r,\\
				p_{\theta}&=-\im\hbar\left(\partialderivative{}{\theta}+\frac{1}{2}\cotg{\theta}\right),\\
				p_{\phi}&=-\im\hbar\partialderivative{}{\phi}.
			\end{align}
		\end{subequations}
    \end{enumerate}
\end{solution}