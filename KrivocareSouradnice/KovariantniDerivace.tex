\sec{Kovariantní derivace}
Je zadáno vektorové pole v křivočarých souřadnicích $\vector{F}=\vector{F}(\vector{q})=F^{k}(\vector{q})\vector{n}_{(k)}$.
Derivování podle $i$-té složky vede na
\begin{equation}
	\label{eq:CovariantVector}
	\partialderivative{\vector{F}}{q^{i}}
		=\partialderivative{}{q^{i}}\left(F^{k}\vector{n}_{(k)}\right)
		=\partialderivative{F^{k}}{q^{i}}\vector{n}_{(k)}
			+\underbrace{\partialderivative{\vector{n}_{(k)}}{q^{i}}F^{k}}_{\partialderivative{\vector{n}_{(j)}}{q^{i}}F^{j}}
		=\partialderivative{F^{k}}{q^{i}}\vector{n}_{(k)}+\Gamma_{ij}^{k}F^{j}\vector{n}_{(k)},
\end{equation}
kde poslední rovnost plyne z toho, že $\partial\vector{n}_{(j)}/\partial q^{i}$ je vektor, a dá se tudíž vyjádřit v bázi $\vector{n}_{(k)}$ pomocí koeficientů $\Gamma_{ij}^{k}$,
\begin{equation}
	\label{eq:ChristoffelSymboln}
	\partialderivative{\vector{n}_{(j)}}{q^{i}}=\Gamma_{ij}^{k}\vector{n}_{(k)}.
\end{equation}	
Koeficienty $\Gamma_{ij}^{k}$ se nazývají \emph{Christoffelovy symboly}\index{symboly!Christoffelovy} (2. druhu).
Jsou symetrické vůči záměně dolních indexů\footnote{
	Je vidět z~\eqref{eq:Vectorn} a~\eqref{eq:ChristoffelSymboln}:
	\begin{equation}
		\partialderivative{\vector{n}_{(j)}}{q^{i}}
			=\frac{\partial^{2}\vector{x}}{\partial q_{j}\partial q^{i}}
			=\partialderivative{\vector{n}_{(i)}}{q^{j}}.
	\end{equation}
}
\begin{equation}
	\Gamma_{ij}^{k}=\Gamma_{ji}^{k}
\end{equation}
a dají se vyjádřit pomocí derivací metrického tenzoru\footnote{
	Odvození spočívá v derivování~\eqref{eq:MetricTensor} s různě označenými indexy.
}
\begin{equation}
	\label{eq:ChristoffelSymbol}
	\boxed{\Gamma_{ij}^{k}
	=\frac{1}{2}g^{kl}\left(\partialderivative{g_{il}}{q^{j}}+\partialderivative{g_{jl}}{q^{i}}-\partialderivative{g_{ij}}{q^{l}}\right)}.
\end{equation}

Rovnice~\eqref{eq:CovariantVector} zapsaná ve složkách zní
\begin{equation}
	\label{eq:CovariantDerivative}
	\boxed{{F^{k}}_{;i}=\partialderivative{F^{k}}{q^{i}}+\Gamma_{ij}^{k}F^{j}}
\end{equation}
a nazývá se \emph{kovariantní derivace}\footnote{
	Kovariantní derivace podle $i$-té složky se značí středníkem následovaným indexem $i$.
	Obyčejná derivace se značí čárkou následovanou indexem $i$.
	Rovnici~\eqref{eq:CovariantDerivative} lze tedy zapsat jako
	\begin{equation}
		{F^{k}}_{;i}={F^{k}}_{,i}+\Gamma^{k}_{ij}F^{j}.
	\end{equation}
}.
