\sec{Lagranžián a Hamiltonián}
Lagranžián se v obecných křivočarých souřadnicích zapisuje jako
\begin{equation}
	\label{eq:LagrangianCurvilinear}
		L(\dot{\vector{q}},\vector{q})=T(\dot{\vector{q}},\vector{q})-V(\vector{q})
			=\frac{1}{2}Mg_{kl}(\vector{q})\dot{q}^{k}\dot{q}^{l}-V(\vector{q}),
\end{equation}
kde předpokládáme, že kinetický člen systému lze zapsat jako kvadratickou formu 
v rychlostech $\dot{\vector{q}}$.
K Hamiltoniánu přejdeme zavedením zobecněných hybností
\begin{equation}
	p_{k}=\partialderivative{L}{\dot{q}^{k}}=Mg_{kl}\dot{q}^{l},
\end{equation}
takže
\begin{equation}
	\label{eq:HamiltonianCurvilinear}
	H(\vector{p},\vector{q})
		=T(\vector{p},\vector{q})+V(\vector{q})
		=\frac{1}{2M}g^{kl}(\vector{q})p_{k}p_{l}+V(\vector{q}).
\end{equation}
Nuance zavedení kvantového Hamiltoniánu jsou diskutovány v práci~\cite{Podolsky1928}.
