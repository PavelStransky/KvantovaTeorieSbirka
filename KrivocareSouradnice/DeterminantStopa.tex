\subsection{Determinant a stopa}
\begin{enumerate}
\item
	Dokažte, že pro diagonalizovatelnou matici $\matrix{A}$ platí vztah
	\begin{equation}
		\label{eq:DetExp}
		\det\e^{\matrix{A}}=\e^{\trace\matrix{A}}.
	\end{equation}
	
\item
	Na základě předchozího vztahu ukažte, že pokud je matice $\matrix{B}$ funkcí 
	zobecněných souřadnic, $\matrix{B}=\matrix{B}(\vector{q})$, platí
	\begin{equation}
		\label{eq:LogDet}
		\partialderivative{}{q^{i}}\ln\det\matrix{B}(\vector{q})=\trace\matrix{B}^{-1}(\vector{q})\partialderivative{}{q^{i}}\matrix{B}(\vector{q}),
	\end{equation}
	kde $\matrix{B}^{-1}(\vector{q})$ je matice inverzní k matici $\matrix{B}(\vector{q})$.
\end{enumerate}

\begin{solution}
	\begin{enumerate}
		\item
			Matici $\matrix{A}$ lze zdiagonalizovat a vyjádřit ve tvaru
			\begin{equation}
				\matrix{A}=\matrix{C}\matrix{D}\matrix{C}^{-1},
			\end{equation}
			kde $\matrix{D}$ je diagonální matice.
			Dosazením do~\eqref{eq:DetExp} dostaneme na levé a pravé straně
			\begin{align}
				\det{\e^{\matrix{A}}}
					&=\det{\e^{\matrix{C}\matrix{D}\matrix{C}^{-1}}}
					 =\det{\matrix{C}\e^{\matrix{D}}\matrix{C}^{-1}}
					 =\det{\matrix{C}}\det\e^{\matrix{D}}\frac{1}{\det{\matrix{C}}}\nonumber\\
					&=\det{\matrix{D}}=\prod_{j}\e^{D_{jj}},\\
				\e^{\trace{\matrix{A}}}
					&=\e^{\trace{\matrix{C}\matrix{D}\matrix{C}^{-1}}}
					 =\e^{\trace{\matrix{D}}}
					 =\e^{\sum_{j}D_{jj}}
					 =\prod_{j}\e^{D_{jj}},
			\end{align}
			čímž je~\eqref{eq:DetExp} dokázáno.
		
		\item
			Označme $\matrix{B}(\vector{q})=\e^{\matrix{A}(\vector{q})}$, 
			takže $\matrix{A}(\vector{q})=\ln{\matrix{B}(\vector{q})}$.
			Dosadíme-li do~\eqref{eq:DetExp}, dostaneme
			\begin{equation}
				\det{\matrix{B}(\vector{q})}=\e^{\trace\ln{\matrix{B}(\vector{q}})},
			\end{equation}
			po zlogaritmování
			\begin{equation}
				\ln\det{\matrix{B}(\vector{q})}=\trace\ln{\matrix{B}(\vector{q})}
			\end{equation}
			a zderivování
			\begin{equation}
				\partialderivative{}{q^{i}}\ln\det{\matrix{B}(\vector{q})}=\trace{\partialderivative{}{q^{i}}\ln{B(\vector{q})}}=\trace{\frac{1}{B(\vector{q})}\partialderivative{}{q^{i}}B(\vector{q})}.
			\end{equation}
			Tím je důkaz hotov.
	\end{enumerate}
\end{solution}
