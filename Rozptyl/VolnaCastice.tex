\subsection*{Stacionární stavy volné částice s ostrou hodnotou impulsmomentu}
Vlnová funkce volné částice s velikostí vlnového vektoru $k$ se vyjádří ve sférických souřadnicích pomocí radiální vlnové funkce a kulových funkcí
\begin{equation}
    \psi_{klm}(r,\theta,\phi)
        =\braket{\vector{r}}{k\,l\,m}
        =R_{klm}(r)Y_{lm}(\theta,\phi),
\end{equation}
což je řešení Schrödingerovy rovnice
\begin{equation}
    -\frac{\hbar^{2}}{2M}\Delta\psi_{klm}(r,\theta,\phi)
        =E\psi_{klm}(r,\theta,\phi),
\end{equation}
kde Laplaceův operátor ve sférických souřadnicích je
\begin{align}
    \Delta
        &=\frac{1}{r^{2}}\frac{\partial}{\partial r}r^{2}\frac{\partial}{\partial r}
            +\frac{1}{r^{2}\sin{\theta}}\frac{\partial}{\partial\theta}\sin{\theta}\frac{\partial}{\partial\theta}
            +\frac{1}{r^{2}\sin^{2}\theta}\frac{\partial^{2}}{\partial\phi^{2}}\nonumber\\
        &=\frac{\partial^{2}}{\partial r^{2}}+\frac{2}{r}\frac{\partial}{\partial r}
            -\frac{\vector{L}^{2}}{\hbar^{2}r^{2}}\,,
\end{align}
přičemž $E=\hbar^{2}k^{2}/2M$.
Po dosazení $\vector{L}^{2}\psi_{klm}(r,\theta,\phi)=\hbar^{2}l(l+1)\psi_{klm}(r,\theta,\phi)$
\begin{align}
    \left[\frac{\partial^{2}}{\partial r^{2}}+\frac{2}{r}\frac{\partial}{\partial r}
        +\left(k^{2}-\frac{l(l+1)}{r^{2}}\right)\right]R_{kl}(r)
        &=0\nonumber\\
        &\downarrow z=kr\nonumber\\
    \left[\frac{\partial^{2}}{\partial z^{2}}+\frac{2}{z}\frac{\partial}{\partial z}
        +\left(1-\frac{l(l+1)}{z^{2}}\right)\right]R_{kl}(z)
        &=0,
\end{align}
což je přesně Helmholtzova rovnice pro sférické Besselovy funkce \eqref{eq:Helmholtz}.
Obecné řešení pro radiální část tedy zní
\begin{equation}\label{eq:FreeParticleRadialWF}
    R_{kl}(r)
        =a_{l}(k)j_{l}(kr)+b_{l}(k)n_{l}(kr).
\end{equation}
Sférická Neumannova funkce podle asymptotiky \eqref{eq:SphericalBessel0} diverguje pro $r=0$,
ve výsledném řešení se tudíž nebude vyskytovat.
Užitím relací ortogonality \eqref{eq:SphericalBesselOrtogonality} 
dostáváme normovanou radiální část vlnové funkce volné částice
\begin{equation}
    \important{
    R_{kl}(r)			
        =\sqrt{\frac{2}{\pi}}\,k j_{l}(kr)
    }.
    \label{eq:FreeParticleRadialWFBessel}
\end{equation}
