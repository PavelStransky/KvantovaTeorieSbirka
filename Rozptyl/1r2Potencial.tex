\subsection{Rozptyl na $\frac{1}{r^{2}}$ potenciálu}
Uvažujte rozptyl částice o hmotnosti $M$ na potenciálu
\begin{equation}
    \label{eq:1r2PotentialV}
	V(r)=\frac{v}{r^{2}},
\end{equation}
kde parametr $v$ může být kladný (pro odpudivou sílu) nebo záporný (pro sílu přitažlivou).

\begin{enumerate}
	\item 
		Řešením Schrödingerovy rovnice nalezněte vlnovou funkci pro energii $E>0$.
		
	\item
		Spočítejte fázové posunutí $\delta_{l}(k)$ $l$-té parciální vlny 
		a načrtněte jeho závislost na $k$ (nebo na energii $E$).
		
	\item
		Nalezněte totální účinný průřez pro $l$-tou parciální vlnu $\sigma_{l}(k)$.
		Diskutujte fyzikální příčinu skutečnosti, že pro $k\rightarrow0$ účinný průřez diverguje.
\end{enumerate}


\begin{solution}
	\begin{enumerate}
		\item 
		Potenciál je sféricky symetrický, úhlová část vlnové funkce je tedy dána kulovými funkcemi.

		Schrödingerova rovnice pro radiální část vlnové funkce je
		\begin{equation}\label{eq:1r2SchrodingerR}
			\left[\derivative[2]{}{r}+\frac{2}{r}\derivative{}{r}+k^{2}-\frac{l(l+1)}{r^{2}}-\frac{2Mv}{\hbar^{2}r^{2}}\right]R_{kl}(r)=0,
		\end{equation}
		kde $k^{2}=2ME/\hbar^{2}$. 
		Pokud $v=0$, jedná se o rovnici pro volnou částici ve sférických souřadnicích, jejíž obecné řešení je dáno vztahem~\eqref{eq:FreeParticleRadialWFBessel}.
		Pro nenulové $v$ se zavede substituce $\lambda=\lambda(l)$ daná rovnicí
		\begin{equation}\label{eq:Kvadr}
			\lambda(\lambda+1)=l(l+1)+\frac{2Mv}{\hbar^{2}},
		\end{equation}
		čímž rovnice~\ref{eq:1r2SchrodingerR} formálně přejde na rovnici pro volnou částici
		\begin{equation}
			\left[\derivative[2]{}{r}+\frac{2}{r}\derivative{}{r}+k^{2}-\frac{\lambda(\lambda+1)}{r^{2}}\right]R_{kl}(r)=0,
		\end{equation}
		jejíž obecné řešení je dáno lineární kombinací sférické Besselovy a Neumannovy funkce
		\begin{equation}
			R_{kl}(r)=a_{\lambda}(k)j_{\lambda}(kr)+b_{\lambda}(k)n_{\lambda}(kr)
		\end{equation}
	
		Kvadratická rovnice~\eqref{eq:Kvadr} má dvě řešení
		\begin{equation}
			\lambda_{\pm}=-\frac{1}{2}\pm\frac{1}{2}\sqrt{1+4l(l+1)+\frac{8Mv}{\hbar^{2}}}.
		\end{equation}
		To, že nyní $\lambda$ není přirozené číslo (obecně může být dokonce komplexní) vůbec nevadí, Besselova funkce to snese. 
	
		Nadále se omezíme na případ, kdy $\lambda_{\pm}\in\mathbb{R}$ (to nastává vždy v případě odpudivého potenciálu $v>0$ a pro nepříliš silné přitažlivé potenciály).
		Jelikož platí, že
		\begin{equation}
			\lambda_{+}+\lambda_{-}=-1,
		\end{equation}
		je díky identitě~\eqref{eq:BesselNeumann} jedno, jestli se vybere $\lambda_{+}$ a sférickou Besselovu funkci, nebo $\lambda_{-}$ a sférickou Neumannovu funkci.
		Zvolme kořen $\lambda_{+}$.

		\emph{Asymptotika v bodě $r=0$} (vlnová funkce zde nesmí divergovat, aby byla normalizovatelná k $\delta$ funkci) navíc vyžaduje, aby $\lambda\geq0$, tj.
		\begin{equation}
			\frac{\hbar^{2}}{2M}l(l+1)\geq-v,
		\end{equation}        
		Normalizovaná radiální část vlnové funkce je pak
		\begin{equation}\label{eq:1r2PotencialR}
			\important{
				R_{kl}(r)=\sqrt{\frac{2}{\pi}}kj_{\lambda}(kr),\quad\lambda=-\frac{1}{2}+\frac{1}{2}\sqrt{1+4l(l+1)+\frac{8Mv}{\hbar^{2}}}
			}.
		\end{equation}

	\item
		Asymptotika radiální části vlnové funkce (sférické Besselovy funkce) pro $kr\rightarrow\infty$ je v případě volné částice~\eqref{eq:SphericalBesselInfinity}
		\begin{equation}
			j_{l}(kr)\sim\frac{\sin\left(kr-l\frac{\pi}{2}\right)}{kr}.
		\end{equation}
		Analogicky pro částici v potenciálu~\eqref{eq:1r2PotentialV}
		\begin{equation}
			j_{\lambda}(kr)
				\sim\frac{\sin\left(kr-\lambda\frac{\pi}{2}\right)}{kr}
				=\frac{\sin\left(kr-l\frac{\pi}{2}+\delta_{l}\right)}{kr},
		\end{equation}
		kde
		\begin{equation}
			\delta_{l}=(l-\lambda)\frac{\pi}{2}
		\end{equation}
		je hledaný fázový posuv ($\lambda$ je dáno v rovnici~\eqref{eq:1r2PotencialR}).
		Z výrazu je ihned vidět, že $\delta_l$ nezávisí na $k$, a tedy ani na energii.
		To souvisí se skutečností, že systém je škálově invariantní vůči přeškálování souřadnice a energie
		\begin{equation}
			r\mapsto ar, \quad k\mapsto k/a
		\end{equation}
		(vlnová funkce závisí jen na součinu $kr$).

		Fázový posuv je také možné počítat v 1. Bornově aproximaci.
		Výpočet je však celkem zdlouhavý a vede jen přibližné řešení.  
		Navíc je zde diskutabilní platnost Bornovy aproximace (důvody plynou z diskuze v Poznámkách níže).

	\item
		Stačí jen dosadit do vzorečku
		\begin{equation}
			\sigma_{l}(k)=\frac{4\pi}{k^{2}}(2l+1)\sin^{2}\delta_l.
		\end{equation}
\end{enumerate}
\end{solution}

\begin{note}
	Ač potenciál~\eqref{eq:1r2PotentialV} vypadá celkem nevinně, je to skrytá bestie.
	Problematické není jeho chování v $r\rightarrow\infty$ (potenciál není dlouhodosahový\footnote{
		Dlouhodosahový je Coulombický potenciál, všechny rychleji klesající již jsou krátkodosahové.
		Krátkodosahovost potenciálu~\eqref{eq:1r2PotentialV} je patrná z toho, že částice ovlivněná tímto potenciálem se asymptoticky chová jako volná částice. 
		To v případě Coulombického potenciálu není pravda.
	}), nýbrž chování v bodě $r=0$.
	Už v klasické fyzice je chování potenciálu divoké: pro přitažlivý potenciál $v<0$ částice s energií $E<0$ vždy spadne do středu, přičemž spadne do něj dokonce za konečný čas, avšak vykoná při tom nekonečně mnoho otáček okolo středu a dopadne s nekonečnou rychlostí (spočítejte si to, je to poučné).
	Potenciál totiž svou silou překoná i centrifugální bariéru.
	V kvantové fyzice je Hamiltonián pro $v<0$ a $E<0$ nesamosdružený.
	Kdybyste přesto počítali jeho spektrum, dospěli byste k výsledku, že energie systému není omezená zdola (základní stav by ležel v $-\infty$).

	Patologie potenciálu se projeví i v rozptylu, a to právě tím, že pro malé energie účinný průřez diverguje.
	Divergence je způsobena chováním potenciálu v počátku.
	A to jsme tiše pominuli případ, kdy $\lambda$ je komplexní číslo, což odpovídá případu, kdy centrifugální bariéra nestačí a částice se přes ni dostane k singularitě rozptylového centra.

	Chceme-li s potenciálem~\eqref{eq:1r2PotentialV} pracovat realisticky, je potřeba ho nějakým způsobem \uv{regulari\-zo\-vat}.
	To je možné například tak, že částici nedovolíme přiblížit se k počátku a pro $r<\rho$ potenciál nahradíme bariérou $V(r)=\infty$.

	Potenciál~\eqref{eq:1r2PotentialV} popisuje například interakci bodového náboje s dipólem.
	Dipól však není bodová částice a je tedy jasné, že pro $r\approx0$ model přestává platit a je potřeba uvažovat složitější tvar interakce.

	O matematických vlastnostech potenciálu $\frac{1}{r^{2}}$ se dočtete více například v článcích~\cite{Coon2002} či~\cite{Frank1971}.
\end{note}
