\subsection*{Sférické Besselovy funkce}
Sférické Besselovy funkce jsou dvě linárně nezávislá řešení diferenciální rovnice 2. řádu 
(někdy nazývané Helmholtzova rovnice)
\begin{equation}\label{eq:Helmholtz}
    \important{
        \left[\frac{\d^{2}}{\d z^{2}}+\frac{2}{z}\frac{\d}{\d z}+
                \left(1-\frac{l(l+1)}{z^{2}}\right)\right]
                \begin{array}{c}
                    j_{l}(z)\\
                    n_{l}(z)
                \end{array}
            =0
    }.
\end{equation}
\begin{itemize}
\item
    $j_{l}(z)$ se nazývá \emph{sférická Besselova funkce} nebo
    \emph{sférická Besselova funkce 1. druhu}.
    
\item
    $n_{l}(z)$ se nazývá \emph{sférická Neumannova funkce} nebo
    \emph{sférická Besselova funkce 2. druhu}.
    
\item
    Definují se také \emph{sférické Hankelovy funkce 1. a 2. druhu} vztahy
    \begin{subequations}
        \begin{align}
            h_{l}^{(1)}(z)
                &\equiv j_{l}(z)+\im n_{l}(z),\\
            h_{l}^{(2)}(z)
                &\equiv j_{l}(z)-\im n_{l}(z).
        \end{align}					
    \end{subequations}
\end{itemize}
$l$ je parametr (převážně celočíselný nezáporný, což budeme předpokládat v dalších výrazech, ale obecně může být reálný).

\note Sférické Besselovy funkce mají vztah k Besselovým (cylindrickým) funkcím 
\begin{subequations}
    \begin{align}
        j_{l}(z)&=\sqrt{\frac{\pi}{2z}}J_{l+\frac{1}{2}}(z),\\
        n_{l}(z)&=\sqrt{\frac{\pi}{2z}}N_{l+\frac{1}{2}}(z).
    \end{align}
\end{subequations}

\subsubsection*{Symetrie}
    \begin{subequations}
        \begin{align}
            j_{l}(z)&=\minus{l}j_{l}(-z),\\
            n_{l}(z)&=\minus{l+1}n_{l}(-z),\\
            h_{l}^{(1,2)}(z)&=\minus{l}h_{l}^{(1,2)}(-z).
        \end{align}				
    \end{subequations}

\subsection*{Vztah Besselových a Neumannových funkcí}
    \begin{subequations}
        \begin{align}
            l_{l}(z)&=\minus{l}n_{-l-1}(z)\\
            n_{l}(z)&=\minus{l+1}j_{-l-1}(z).
        \end{align}
        \label{eq:BesselNeumann}
    \end{subequations}

\subsubsection*{Vyjádření pomocí řady}
    \begin{subequations}
        \begin{align}
            j_{l}(z)
                &=z^{l}\sum_{n=0}^{\infty}\frac{1}{n!\left(2l+2n+1\right)!!}\left(-\frac{z^{2}}{2}\right)^{n}\\
            n_{l}(z)
                &=-\frac{1}{z^{l+1}}\left\{
                    \sum_{n=0}^{l-1}\frac{\left(2l-2n-1\right)!!}{n!}\left(-\frac{z^{2}}{2}\right)^{n}
                    +\frac{1}{l!}\left(\frac{z^{2}}{2}\right)^{l}
                    +\minus{l}\sum_{n=l+1}^{\infty}\frac{1}{n!\left(2n-2l-1\right)!!}\left(-\frac{z^{2}}{2}\right)^{n}
                    \right\}
        \end{align}        
    \end{subequations}

\subsubsection*{Vyjádření pomocí goniometrických funkcí}
    \begin{subequations}
        \begin{align}
            j_{l}(z)
                &=(-z)^{l}\left(\frac{1}{z}\frac{\d}{\d z}\right)^{l}\frac{\sin{z}}{z}\\					
            n_{l}(z)
                &=-(-z)^{l}\left(\frac{1}{z}\frac{\d}{\d z}\right)^{l}\frac{\cos{z}}{z}
        \end{align}        
    \end{subequations}

\subsubsection*{Asymptotika $z\rightarrow0$}
    \begin{equation}\label{eq:SphericalBessel0}
        \important{
            \begin{aligned}
                j_{l}(z)
                    &\xrightarrow{z\rightarrow0}\frac{z^{l}}{(2l+1)!!}\\
                n_{l}(z)
                    &\xrightarrow{z\rightarrow0}
                        \begin{cases}
                            -\frac{1}{z} & \text{pokud } l=0\\
                            -\frac{(2l-1)!!}{z^{l+1}} & \text{pokud } l>0
                        \end{cases}
            \end{aligned}
        }
    \end{equation}			
    Za předpokladu $l\geq0$ divergují sférické Neumannovy funkce pro $z\rightarrow0$.

\subsubsection*{Asymptotika $z\rightarrow\infty$}
    \begin{equation}\label{eq:SphericalBesselInfinity}
        \important{
            \begin{aligned}
                j_{l}(z)
                    \xrightarrow{z\rightarrow\infty}\frac{\sin\left(z-l\frac{\pi}{2}\right)}{z}\\
                n_{l}(z)
                    \xrightarrow{z\rightarrow\infty}-\frac{\cos\left(z-l\frac{\pi}{2}\right)}{z}
            \end{aligned}
        }
    \end{equation}

\subsubsection*{Relace ortogonality}
\begin{equation}\label{eq:SphericalBesselOrtogonality}
    \int_{0}^{\infty}j_{l}(kr)j_{l}(k'r)r^{2}\d r
        =\frac{\pi}{2k^{2}}\delta(k-k')
\end{equation}

\subsubsection*{Rozklad exponenciály}
\begin{align}
    \e^{\im\vector{k}\cdot\vector{r}}
        &=4\pi\sum_{l=0}^{\infty}\sum_{m=-l}^{l}\im^{l}j_{l}(kr)
            Y_{lm}^{*}\left(\frac{\vector{k}}{k}\right)
            Y_{lm}\left(\frac{\vector{r}}{r}\right)=\nonumber\\
        &=\sum_{l=0}^{\infty}\im^{l}(2l+1)j_{l}(kr)P_{l}(\cos{\theta}),
\end{align}
kde $P_{l}$ jsou Legenedreovy polynomy a souřadná soustava je orientována tak, že osa $z$ je rovnoběžná s vektorem $\vector{r}$,
\begin{equation}
    \vector{k}\cdot\vector{r}=kr\cos{\theta}.
\end{equation}		

\subsubsection*{Explicitní vyjádření nejnižších Besselových funkcí}
    \begin{subequations}
        \begin{align}
            j_{0}(z)
                &=\frac{\sin{z}}{z}
            &n_{0}(z)
                &=-\frac{\cos{z}}{z}\\
            j_{1}(z)
                &=\frac{\sin{z}}{z^{2}}-\frac{\cos{z}}{z}
            &n_{1}(z)
                &=-\frac{\cos{z}}{z^{2}}-\frac{\sin{z}}{z}\\
            j_{2}(z)
                &=\left(\frac{3}{z^{3}}-\frac{1}{z}\right)\sin{z}-\frac{3}{z^{2}}\cos{z}
            &n_{2}(z)
                &=-\left(\frac{3}{z^{3}}-\frac{1}{z}\right)\cos{z}-\frac{3}{z^{2}}\sin{z}
        \end{align}            
    \end{subequations}