\subsection{Magnetický moment}
	Dva nezávislé impulsmomenty $\vectoroperator{L}$ (například orbitální moment hybnosti) a $\vectoroperator{S}$ (vnitřní spin systému) splňující $\commutator{\vectoroperator{L}}{\vectoroperator{S}}=0$ se složí na celkový impulsmoment
	\begin{equation}
		\vectoroperator{J}=\vectoroperator{L}+\vectoroperator{S}.
	\end{equation}
	Stavem $\ket{(ls)jm}$ označme vlastní vektory operátorů $\vectoroperator{L}^{2}$, $\vectoroperator{S}^{2}$, $\vectoroperator{J}^{2}$, $\operator{J}_{3}$:
    \begin{subequations}
        \begin{align}
            \vectoroperator{L}^{2}\ket{(ls)jm}&=l(l+1)\ket{(ls)jm},\\
            \vectoroperator{S}^{2}\ket{(ls)jm}&=s(s+1)\ket{(ls)jm},\\
            \vectoroperator{J}^{2}\ket{(ls)jm}&=j(j+1)\ket{(ls)jm},\\
            \operator{J}_{3}\ket{(ls)jm}&=m\ket{(ls)jm}.
        \end{align}            
        \label{eq:LSEigenvalues}
    \end{subequations}
	Definujme operátor magnetického momentu\footnote{
		Magnetický moment je vyjádřen v jednotkách Bohrova (jaderného) magnetonu
		\begin{equation}
			\mu_{0}=\frac{e\hbar}{2M},
		\end{equation}
		kde $e$ je elementární náboj, $M$ je hmotnost elektronu (nukleonu).
		Uvedený výraz platí v jednotkách SI, v Gaussovských elektromagnetických jednotkách se objevuje ještě rychlost světla $c$ ve jmenovateli.
	}
	\begin{equation}
		\vectoroperator{\mu}=g_{L}\vectoroperator{L}+g_{S}\vectoroperator{S},
	\end{equation}
	přičemž $g_{L}$, $g_{S}$ jsou reálné parametry, které se nazývají gyromagnetické faktory ($g$-faktory).\footnote{
		Jejich číselné hodnoty jsou
		\begin{align}
			g_{\mathrm{elektron}}&=-2.00231930419922\approx2 \\
			g_{\mathrm{mion}}&=-2.0023318414\approx2 \\
			g_{\mathrm{neutron}}&=-3.82608545\\
			g_{\mathrm{proton}}&=5.585694702
		\end{align}
		(znaménka $g_{L,S}$ a $\mu_{0}$ bývají občas definována obráceně).
	}

	Spočítejte diagonální maticový element\footnote{
		Veličina s největší projekcí $j=m$ se nazývá \emph{magnetický moment částice},
		\begin{equation}
			\mu\equiv\matrixelement{(ls)jj}{\mu_{z}}{(ls)jj}.
		\end{equation}
	}
	\begin{equation}
		\matrixelement{(ls)jm}{\vectoroperator{\mu}}{(ls)jm}.
	\end{equation}

\begin{solution}
	Předně z výběrových pravidel pro projekci impulsmomentu Wigner-Eckartova teorému vyplývá, že
	\begin{equation}
		\matrixelement{(ls)jm}{\operator{\mu}_{x}}{(ls)jm}=\matrixelement{(ls)jm}{\operator{\mu}_{y}}{(ls)jm}=0,
	\end{equation}
	neboť $\operator{\mu}_{x,y}$ jsou zapsané ve sférických komponentách pomocí lineární kombinace $\tensoroperatorcomponent{\mu}{1}{\pm1}$
	a $m\pm1\neq m$.

	K výpočtu maticového elementu $\operator{\mu}_{z}$ se využije projekční teorém~\eqref{eq:ProjectionTheorem}:
	\begin{align}
		&\matrixelement{(ls)jm}{\operator{\mu}_{z}}{(ls)jm}=\nonumber\\
		&\qquad=\matrixelement{(ls)jm}{\frac{\vectoroperator{J}\cdot\vectoroperator{\mu}}{\vectoroperator{J}^{2}}\,\operator{J}_{z}}{(ls)jm}=\nonumber\\
		&\qquad=\frac{m}{j(j+1)}\,\matrixelement{(ls)jm}{\vectoroperator{J}\cdot\vectoroperator{\mu}}{(ls)jm}
	\end{align}
	Dále
	\begin{align}
		\vectoroperator{J}\cdot\vectoroperator{\mu}&=(\vectoroperator{L}+\vectoroperator{S})\cdot(g_{L}\vectoroperator{L}+g_{S}\vectoroperator{S})=\nonumber\\
		&=g_{L}\vectoroperator{L}^{2}+g_{S}\vectoroperator{S}^{2}+(g_{L}+g_{S})\,\vectoroperator{L}\cdot\vectoroperator{S}
	\end{align}
	a k vyjádření $\vectoroperator{L}\cdot\vectoroperator{S}$ využijeme standardní \trick{trik (spin-orbitální vazba)}
	\begin{equation}
		\vectoroperator{J}^{2}=(\vectoroperator{L}+\vectoroperator{S})\cdot(\vectoroperator{L}+\vectoroperator{S})=\vectoroperator{L}^{2}+\vectoroperator{S}^{2}+2\,\vectoroperator{L}\cdot\vectoroperator{S},
	\end{equation}
	takže
	\begin{equation}
		\vectoroperator{L}\cdot\vectoroperator{S}=\frac{1}{2}(\vectoroperator{J}^{2}-\vectoroperator{L}^{2}-\vectoroperator{S}^{2}).
	\end{equation}
	Po dosazení a využití relací~\eqref{eq:LSEigenvalues} vychází
	\begin{align}
		\vectoroperator{J}\cdot\vectoroperator{\mu}
			&=g_{L}\vectoroperator{L}^{2}+g_{S}\vectoroperator{S}^{2}+\frac{1}{2}\left(g_{L}+g_{S}\right)\left(\vectoroperator{J}^{2}-\vectoroperator{L}^{2}-\vectoroperator{S}^{2}\right)\nonumber\\
			&=\frac{1}{2}\left(g_{L}+g_{S}\right)\vectoroperator{J}^{2}+\frac{1}{2}\left(g_{L}\vectoroperator{L}^{2}+g_{S}\vectoroperator{S}^{2}-g_{L}\vectoroperator{S}^{2}-g_{S}\vectoroperator{L}^{2}\right)\nonumber\\
			&=\frac{1}{2}\left(g_{L}+g_{S}\right)\vectoroperator{J}^{2}+\frac{1}{2}\left(g_{L}-g_{S}\right)\left(\vectoroperator{L}^{2}-\vectoroperator{S}^{2}\right),
	\end{align}
	a tedy konečný výsledek je
	\begin{equation}
		\boxed{
			\matrixelement{(ls)jm}{\operator{\mu}_{z}}{(ls)jm}
			=\frac{m}{2}\left\{g_{L}+g_{S}+\frac{g_{L}-g_{S}}{j(j+1)}\left[l\left(l+1\right)-s\left(s+1\right)\right]\right\}\equiv g_{J}m,
		}
	\end{equation}
	kde $g_{J}$ se nazývá \emph{Landéův $g$-faktor}.\index{faktor!Landéův}
\end{solution}
		