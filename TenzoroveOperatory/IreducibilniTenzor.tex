\sec{Ireducibilní tenzorový operátor}
	Složky $\tensoroperatorcomponent{T}{\lambda}{\mu}$ libovolného \emph{ireducibilního}\footnote{
		Ireducibilního proto, že se transformuje podle příslušné 
		ireducibilní reprezentace grupy $\group{SO}(3)$ pomocí Wignerových $D$-funkcí
		\begin{subequations}
			\begin{align}
				\operator{T}_{\mu}^{(\lambda)'}
					&=\sum_{\mu'}D_{\mu'\mu}^{\lambda}(\phi,\theta,\psi)\tensoroperatorcomponent{T}{\lambda}{\mu'},\\
				D_{\mu'\mu}^{\lambda}(\phi,\theta,\psi)
					&\equiv\matrixelement{\lambda\mu'}{\operator{\mathcal{R}}_{3}(\psi)\operator{\mathcal{R}}_{2}(\theta)
						\operator{\mathcal{R}}_{3}(\phi)}{\lambda\mu}
					=\e^{-\im(\mu'\psi+\mu\phi)}d^{\lambda}_{\mu'\mu}(\theta),
			\end{align}				
		\end{subequations}
		narozdíl např. od tenzoru vzniklého vzniklého dyadickým součinem dvou vektorů, 
		který se transformuje běžnými rotačními maticemi
		\begin{equation}
			\operator{T}_{j'k'l'\dotsc}=\sum_{jkl\dotsc}\operator{R}_{j'j}(\psi,\theta,\phi)
				\operator{R}_{k'k}(\psi,\theta,\phi)\operator{R}_{l'l}(\psi,\theta,\phi)\operator{T}_{jkl\dotsc}.
		\end{equation}
		Symboly $\psi,\theta,\phi$ označují Eulerovy úhly.
	} 
	\emph{tenzorového operátoru}\footnote{
		Nazývá se také zkráceně sférický tenzor.
	}\index{operátor!ireducibilní tenzorový}
	$\tensoroperator{T}{\lambda}$, $\lambda=0,1,\dotsc$\footnote{
		Formálně lze zavést i ireducibilní tenzorový operátor poločíselného řádu 
		$\lambda=\frac{1}{2},\frac{3}{2},\dotsc$.
		Jeho střední hodnota by však byla dvojznačná, a proto nemůže odpovídat žádné pozorovatelné.
	}, $\mu=-\lambda,\dotsc,\lambda$
	splňují komutační relace
	\begin{subequations}
		\begin{align}
			\commutator{\operator{J}_{3}}{\tensoroperatorcomponent{T}{\lambda}{\mu}}
				&=\mu\tensoroperatorcomponent{T}{\lambda}{\mu}\\
			\commutator{\operator{J}_{\pm}}{\tensoroperatorcomponent{T}{\lambda}{\mu}}
				&=\alpha^{(\pm)}(\lambda,\mu)\tensoroperatorcomponent{T}{\lambda}{\mu\pm1},
		\end{align}			
		\label{eq:IrreducibleTensor}
	\end{subequations}
	kde $\vectoroperator{J}$ je operátor impulsmomentu, $\operator{J}_{\pm}=\operator{J}_{1}\pm\im\operator{J}_{2}$ a
	\begin{equation}
		\alpha^{(\pm)}(\lambda,\mu)
			\equiv\sqrt{\lambda(\lambda+1)-\mu(\mu\pm1)}.
	\end{equation}