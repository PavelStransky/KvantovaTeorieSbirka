\sec{Wigner-Eckartův teorém}
	\begin{equation}
		\label{eq:WignerEckart}
		\boxed{
			\begin{aligned}
			&\matrixelement{a,J\,M}{\tensoroperatorcomponent{T}{\lambda}{\mu}}{b,j\,m}=\\
			&\qquad=\frac{\minus{J+\lambda-j}}{\sqrt{2J+1}}
				\clebsch{\lambda}{\mu}{j}{m}{J}{M}\,\reducedmatrixelement{a,J}{\tensoroperator{T}{\lambda}}{b,j}\\
			&\qquad=\minus{J-M}\threej{J}{\lambda}{j}{-M}{\mu}{m}\reducedmatrixelement{a,J}{\tensoroperator{T}{\lambda}}{b,j}
			\end{aligned}
		},
	\end{equation}
	přičemž:
	\begin{itemize}
	\item
		Zlomek $\frac{(-1)^{J+\lambda-j}}{\sqrt{2J+1}}$ je jen záležitost konvence.
		Zde je použita stejná konvence jako např. v knize J.~Formánka~\cite{Formanek2004}.	
	
	\item 
		$\reducedmatrixelement{a,J}{\tensoroperator{T}{\lambda}}{b,j}$ je \emph{redukovaný maticový element}.\index{maticový element!redukovaný}

	\item 
		$a,b$ označují další vlastní čísla (může jich být i více) operátoru (operátorů) $\operator{A}$,	které spolu s impulsmomentem $\vectoroperator{J}$ tvoří úplnou množinu pozorovatelných:
		\begin{equation}
			\commutator{\operator{A}}{\vectoroperator{J}}=0.
		\end{equation}

	\item 
		Mezi $3j$ symbolem\index{symbol!3j} a Clebsch-Gordanovými koeficienty platí relace
        \begin{subequations}
            \begin{align}
                &\threej{j_{1}}{j_{2}}{j_{3}}{m_{1}}{m_{2}}{m_{3}}=\nonumber\\
                &\qquad=\minus{j_{2}-j_{3}-m_{1}}
                    \frac{\clebsch{j_{2}}{m_{2}}{j_{3}}{m_{3}}{j_{1}}{-m_{1}}}{\sqrt{2j_{1}+1}}\\
                &\qquad=\minus{j_{3}-j_{1}-m_{2}}
                    \frac{\clebsch{j_{3}}{m_{3}}{j_{1}}{m_{1}}{j_{2}}{-m_{2}}}{\sqrt{2j_{2}+1}}\\
                &\qquad=\minus{j_{1}-j_{2}-m_{3}}
                    \frac{\clebsch{j_{1}}{m_{1}}{j_{2}}{m_{2}}{j_{3}}{-m_{3}}}{\sqrt{2j_{3}+1}},
            \end{align}                
            \label{eq:3j}
        \end{subequations}
		přičemž uvedené tři rovnosti plynou ze symetrie $3j$ symbolů:
        \begin{subequations}
            \begin{equation}
                \label{eq:Permutation3j}
                \threej{j_{\sigma_{1}}}{j_{\sigma_{2}}}{j_{\sigma_{3}}}
                    {m_{\sigma_{1}}}{m_{\sigma_{2}}}{m_{\sigma_{3}}}=
                \begin{cases}
                    \minus{j_{1}+j_{2}+j_{3}}\threej{j_{1}}{j_{2}}{j_{3}}{m_{1}}{m_{2}}{m_{3}} 
                    & \sign\sigma=-1 \\
                    \threej{j_{1}}{j_{2}}{j_{3}}{m_{1}}{m_{2}}{m_{3}} & \sign\sigma=1
                \end{cases}
            \end{equation}
            Další symetrie $3j$ symbolů:
            \begin{equation}
                \label{eq:Minus3j}
                \threej{j_{1}}{j_{2}}{j_{3}}{-m_{1}}{-m_{2}}{-m_{3}}
                    =\minus{j_{1}+j_{2}+j_{3}}\threej{j_{1}}{j_{2}}{j_{3}}{m_{1}}{m_{2}}{m_{3}}.
            \end{equation}                
        \end{subequations}
	\end{itemize}

	Wigner-Eckartův teorém je užitečný zejména proto, že k výpočtu všech $n=(2J+1)(2\lambda+1)(2j+1)$ maticových elementů stačí znát jediný z nich,	zbytek se dopočte pomocí běžně tabulovaných Clebsch-Gordanových koeficientů.
	Z oněch $n$ elementů jsou navíc všechny, které nesplňují výběrová pravidla pro hodnoty 
	$J,M,\lambda,\mu,j,m$, tj.
	\begin{equation}
	\label{eq:WignerEckartSelectionRules}
		\boxed{
		\begin{aligned}
			m+\mu&=M\\
			|j-\lambda|\leq&J\leq j+\lambda\qquad\text{(trojúhelníková nerovnost)}
		\end{aligned}
		},
	\end{equation}
	nulové.
