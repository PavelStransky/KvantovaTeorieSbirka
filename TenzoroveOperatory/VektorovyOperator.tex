\subsection{Vektorový operátor jako ireducibilní tenzor}
Ukažte, že libovolnému vektorovém operátoru $\vectoroperator{V}$ s kartézskými složkami $(\operator{V}_{1},\operator{V}_{2},\operator{V}_{3})$ se dá přiřadit ireducibilní tenzorový operátor 1. řádu pomocí předpisu
\begin{equation}\label{eq:VectorTensor}
    \boxed{
        \begin{aligned}
            \tensoroperatorcomponent{V}{1}{-1}
                &=\frac{1}{\sqrt{2}}\left(\operator{V}_{1}-\im\operator{V}_{2}\right)\\
            \tensoroperatorcomponent{V}{1}{0}
                &=\operator{V}_{3}\\
            \tensoroperatorcomponent{V}{1}{1}
                &=-\frac{1}{\sqrt{2}}\left(\operator{V}_{1}+\im\operator{V}_{2}\right)
        \end{aligned}
    }
\end{equation}
($\tensoroperatorcomponent{V}{1}{\mu}$ jsou sférické složky vektoru).

\begin{solution}
    Libovolný vektorový operátor $\vectoroperator{V}$ splňuje komutační relace s operátorem momentu
    hybnosti\footnote{Jsou to například operátory $\vectoroperator{X}$, $\vectoroperator{P}$, $\vectoroperator{L}$, $\dotsc$.}
    \begin{equation}
        \commutator{\operator{V}_{j}}{\operator{J}_{k}}=\im\epsilon_{jkl}V_{l}.
    \end{equation}	
    Důkaz transformačních vztahů~\eqref{eq:VectorTensor} se provede přímým dosazením do vztahů~\eqref{eq:IrreducibleTensor}:
    \begin{align*}
        \commutator{\operator{J}_{3}}{\tensoroperatorcomponent{V}{1}{-1}}
            &=-\commutator{\frac{1}{\sqrt{2}}\left(\operator{V}_{x}-\im\operator{V}_{y}\right)}{\operator{J}_{3}}
            =-\frac{1}{\sqrt{2}}\left(-\im\operator{V}_{y}+\operator{V}_{x}\right)
            =-\tensoroperatorcomponent{V}{1}{-1},\\
        \commutator{\operator{J}_{3}}{\tensoroperatorcomponent{V}{1}{0}}
            &=-\commutator{\operator{V}_{3}}{\operator{J}_{3}}=0,\\
        \commutator{\operator{J}_{3}}{\tensoroperatorcomponent{V}{1}{1}}
            &=\commutator{\frac{1}{\sqrt{2}}\left(\operator{V}_{x}+\im\operator{V}_{y}\right)}{\operator{J}_{3}}
            =\frac{1}{\sqrt{2}}\left(-\im\operator{V}_{y}-\operator{V}_{x}\right)
            =+\tensoroperatorcomponent{V}{1}{1},
    \end{align*}
    \begin{align*}
        \commutator{\operator{J}_{\pm}}{\tensoroperatorcomponent{V}{1}{-1}}
            &=-\frac{1}{\sqrt{2}}\commutator{\operator{V}_{1}-\im\operator{V}_{2}}{\operator{J}_{1}\pm\im\operator{J}_{2}}
            =-\frac{1}{\sqrt{2}}\left(\mp\operator{V}_{3}-\operator{V}_{3}\right)
            =\begin{cases}
                \alpha^{+}(1,-1)\tensoroperatorcomponent{V}{1}{0} \\ 0
                \end{cases},\\
        \commutator{\operator{J}_{\pm}}{\tensoroperatorcomponent{V}{1}{0}}
            &=-\commutator{\operator{V}_{3}}{\operator{J}_{1}\pm\im\operator{J}_{2}}
            =\pm\operator{J}_{1}-\im\operator{J}_{2}
            =\alpha^{\pm}(1,0)\tensoroperatorcomponent{V}{1}{\pm1},\\
        \commutator{\operator{J}_{\pm}}{\tensoroperatorcomponent{V}{1}{1}}
            &=\frac{1}{\sqrt{2}}\commutator{\operator{V}_{1}+\im\operator{V}_{2}}{\operator{J}_{1}\pm\im\operator{J}_{2}}
            =\frac{1}{\sqrt{2}}\left(\mp\operator{V}_{3}+\operator{V}_{3}\right)
            =\begin{cases}
                0 \\ \alpha^{-}(1,1)\tensoroperatorcomponent{V}{1}{0}
                \end{cases}.
    \end{align*}
\end{solution}