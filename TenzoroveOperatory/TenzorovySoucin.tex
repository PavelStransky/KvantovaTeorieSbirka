\sec{Tenzorový součin}
	Tenzorový součin dvou ireducibilních tenzorových operátorů 
	\begin{equation}
		\tensoroperator{W}{\lambda}=[\tensoroperator{U}{\lambda_{1}}\otimes\tensoroperator{V}{\lambda_{2}}]^{(\lambda)}
	\end{equation}
	je definován vztahem pro komponenty
	\begin{equation}
		\boxed{
			\tensoroperatorcomponent{W}{\lambda}{\mu}
				=\sum_{\mu_{1},\mu_{2}}
					\clebsch{\lambda_{1}}{\mu_{1}}{\lambda_{2}}{\mu_{2}}{\lambda}{\mu}
					\tensoroperatorcomponent{U}{\lambda_{1}}{\mu_{1}}\tensoroperatorcomponent{V}{\lambda_{2}}{\mu_{2}}
		}.
	\end{equation}
	Pro tenzor nultého řádu pak vyplývá
    \begin{align}
        \tensoroperatorcomponent{W}{0}{0}
            &=\sum_{\mu_{1},\mu_{2}}\clebsch{\lambda_{1}}{\mu_{1}}{\lambda_{2}}{\mu_{2}}{0}{0}
                \tensoroperatorcomponent{U}{\lambda_{1}}{\mu_{1}}\tensoroperatorcomponent{V}{\lambda_{2}}{\mu_{2}}\nonumber\\
            &=\frac{\minus{\lambda_{1}}}{\sqrt{2\lambda_{1}+1}}\sum_{\mu_{1}}\minus{\mu_{1}}
                \tensoroperatorcomponent{U}{\lambda_{1}}{\mu_{1}}\tensoroperatorcomponent{V}{\lambda_{1}}{-\mu_{1}}
    \end{align}
	a na základě této rovnosti se definuje \emph{skalární součin}\index{součin!skalární pro tenzorové operátory} ireducibilních tenzorových operátorů
	\begin{equation}
		\label{eq:TensorOperatorScalar}
		\boxed{
			\tensoroperator{U}{\lambda}\cdot\tensoroperator{V}{\lambda}
				\equiv\minus{-\lambda}\sqrt{2\lambda+1}\,\tensoroperatorcomponent{W}{0}{0}
				=\sum_{\mu}\minus{\mu}\tensoroperatorcomponent{U}{\lambda}{\mu}\tensoroperatorcomponent{V}{\lambda}{-\mu}
		}\,.
	\end{equation}
