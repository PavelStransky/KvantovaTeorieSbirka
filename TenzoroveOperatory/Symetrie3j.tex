\subsection{Využití symetrií $3j$ symbolů}
Pomocí symetrií $3j$ symbolů a znalosti Clebsch-Gordanova koeficientu 
\begin{equation}
    \clebsch{j}{m}{0}{0}{J}{M}=\delta_{jJ}\delta_{mM}
\end{equation}
(plyne z volby $\ket{j_{1}\,j_{1}}\ket{0\,0}=\ket{j_{1}\,j_{1}\,0\,j_{1}}$, 
což je vlastně Condon-Shortleyova fázová konvence) spočítejte Clebsch-Gordanovy koeficienty
\begin{equation}
    \clebsch{0}{0}{j}{m}{J}{M}\quad\mathrm{a}\quad\clebsch{j_{1}}{m_{1}}{j_{2}}{m_{2}}{0}{0}.
\end{equation}

\begin{solution}
	Speciální případ vztahu mezi Clebsch-Gordanovými koeficienty a $3j$ symboly~\eqref{eq:3j} dá
	\begin{align}
		\threej{j}{0}{J}{m}{0}{-M}=\frac{(-1)^{j+M}}{\sqrt{2J+1}}\clebsch{j}{m}{0}{0}{J}{M},
	\end{align}
	přičemž z výběrových pravidel pro skládání momentů hybnosti~\eqref{eq:AngularMomentumSelectionRules} navíc plyne, že jediné nenulové koeficienty jsou ty s $j=J$ a $m=-M$.
	Z permutačních symetrií $3j$ symbolů~\eqref{eq:Permutation3j} pak vyplývá
	\begin{equation}
		\threej{j}{0}{J}{m}{0}{-M}
			=(-1)^{j+J}\threej{0}{j}{J}{0}{m}{-M}
			=\threej{J}{j}{0}{-M}{m}{0},
	\end{equation}
	což zpětně převedeno na Clebsch-Gordanovy koeficienty definičním vztahem~\eqref{eq:3j} je
	\begin{align*}
		(-1)^{j+J}\threej{0}{j}{J}{0}{m}{-M}
			&=\frac{(-1)^{J+M}}{\sqrt{2J+1}}\clebsch{0}{0}{j}{m}{J}{M}\,,\\
		\threej{J}{j}{0}{-M}{m}{0}
			&=(-1)^{J-j}\clebsch{J}{-M}{j}{m}{0}{0},
	\end{align*}
	a tedy
    \begin{subequations}
        \begin{align}
            \clebsch{0}{0}{j}{m}{J}{M}
                &=\clebsch{j}{m}{0}{0}{J}{M}=\delta_{jJ}\delta_{mM}\,,\\
            \clebsch{J}{-M}{j}{m}{0}{0}
                &=(-1)^{J-j}\frac{(-1)^{j+M}}{\sqrt{2J+1}}\clebsch{0}{0}{j}{m}{J}{M}
                 =\frac{(-1)^{J+M}}{\sqrt{2J+1}}\delta_{jJ}\delta_{mM}\,.
        \end{align}            
    \end{subequations}
	Druhý vztah po přeznačení
	$J\mapsto j_{1}$, $-M\mapsto m_{1}$, $j\mapsto j_{2}$, $m\mapsto m_{2}$ 
	vede na hledaný Clebsch-Gordanův koeficient
	\begin{equation}
		\label{eq:ClebschGordan00}
		\boxed{
			\clebsch{j_{1}}{m_{1}}{j_{2}}{m_{2}}{0}{0}
				=\frac{(-1)^{j_{1}-m_{1}}}{\sqrt{2j_{1}+1}}\delta_{j_{1}j_{2}}\delta_{m_{1}\,-m_{2}}
		}.
	\end{equation}
\end{solution}