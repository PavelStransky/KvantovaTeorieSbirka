Pro složené	soustavy nerozlišitelných částic je výhodný popis pomocí kreačních 
a anihilačních operátorů $\operatorconjugate{a}_{k}$, $\operator{a}_{k}$, které působí na \emph{Fockově prostoru}
\begin{equation}
    \hilbert{F}
        =\hilbert{H}^{(0)}\oplus\hilbert{H}^{(1)}\oplus\hilbert{H}^{(2)}\oplus\dotsb
\end{equation}
($\hilbert{H}^{(n)}$ označuje Hilbertův prostor soustavy $n$ částic, 
$\hilbert{H}^{(0)}$ obsahuje pouze jeden stav $\ket{0}$, který se běžně nazývá \emph{vakuum}).
Normované bázové vektory prostoru $\hilbert{H}^{(n)}$ budeme značit
\begin{equation}
    \ket{N_{1},N_{2},\dotsc;N},
        \qquad\text{kde}\quad\sum_{k=1}^{\infty}N_{k}=N
\end{equation}
je celkový počet částic ($N_{1}$ je počet částic v jednočásticovém stavu (orbitalu) $\ket{\phi_{k}}$),
a dají se vytvořit pomocí kreačních operátorů
\begin{equation}
    \ket{N_{1},N_{2},\dotsc;N}
        =\frac{1}{\sqrt{N_{1}!N_{2}!\dotsm}}\left(\operatorconjugate{a}_{1}\right)^{N_{1}}\left(\operatorconjugate{a}_{2}\right)^{N_{2}}\dotsm\ket{0}
\end{equation}	
Schematicky můžeme tedy psát
\begin{equation}
    \hilbert{F}
        =\ket{0}\otimes\operatorconjugate{a}_{j}\ket{0}\otimes\operatorconjugate{a}_{j}\operatorconjugate{a}_{k}\ket{0}\otimes\dotsb
\end{equation}	
Kreační operátory přidávají částici, anihilační ubírají:
\begin{subequations}
    \begin{align}
        \operatorconjugate{a}_{k}\ket{N_{1},\dotsc,N_{k},\dotsc;N}
            &=\sqrt{N_{k}+1}\,\ket{N_{1},\dotsc,N_{k}+1,\dots;N+1}\\
        \operator{a}_{k}\ket{N_{1},\dotsc,N_{k},\dotsc;N}
            &=\sqrt{N_{k}}\,\ket{N_{1},\dotsc,N_{k}-1,\dots;N-1}
    \end{align}        
\end{subequations}
(odmocninové koeficienty plynou z normalizace vektorů).
Působení anihilačního operátoru na vakuum dá $0$:
\begin{equation}
    \operator{a}_{k}\ket{0}=0
\end{equation}

Vlnové funkce soustavy částic musí být symetrické vůči záměně libovolných dvou nerozlišitených bosonů
(částic s celočíselným spinem) a antisymetrické vůči záměně dvou nerozlišitelných fermionů
(částic s poločíselným spinem).
Toho lze docílit tím, že kreační operátory splňují komutační (bosony) nebo antikomutační (fermiony) relace.

Bosonové kreační a anihilační operátory označíme $\operatorconjugate{b}_{k}$, $\operator{b}_{k}$.
Komutační relace mezi nimi zní
\begin{align}\label{eq:BosonCommutator}
        \commutator{\operator{b}_{j}}{\operatorconjugate{b}_{k}}
            &=\delta_{jk}\,,
        &\commutator{\operator{b}_{j}}{\operator{b}_{k}}
            &=\commutator{\operatorconjugate{b}_{j}}{\operatorconjugate{b}_{k}}
            =0\,.
\end{align}
Operátor počtu částic ve stavu $\ket{\phi_{k}}$ a operátor celkového počtu částic jsou
\begin{subequations}
    \begin{align}
        \operator{N}_{k}
            &=\operatorconjugate{a}_{k}\operator{a}_{k}\,,\\
        \operator{N}
            &=\sum_{k}\operatorconjugate{a}_{k}\operator{a}_{k}\,.
    \end{align}    
\end{subequations}
