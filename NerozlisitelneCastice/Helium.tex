\subsection{Atom helia}\label{sec:Helium}
Hamiltonián popisující atom helia zní
\begin{subequations}
    \begin{align}
        \operator{H}&=\operator{H}_{0}+\operator{H}_{\mathrm{I}},\\
        \operator{H}_{0}&=\frac{\vectoroperator{p}_{1}^{2}}{2M}+\frac{\vectoroperator{p}_{2}^{2}}{2M}-\frac{2\gamma}{\operator{r}_{1}}-\frac{2\gamma}{\operator{r}_{2}},\\
        \operator{H}_{\mathrm{I}}&=\frac{\gamma}{\abs{\vector{r}_{1}-\vector{r}_{2}}},
    \end{align}
    \end{subequations}
kde $\operator{H}_{0}$ popisuje interakci jednotlivých elektronů v obalu s atomovým jádrem a $\operator{H}_{I}$ elektrostatické odpuzování mezi oběma elektrony.
Elektrony v obalu jsou nerozlišitelné fermiony.
Jádro je bráno za bodovou částici, spin-orbitální interakci a jaderné pohyby zanedbáváme. 

\begin{enumerate}
    \item
        Určete základní stav a první excitovaný stav atomu helia, pokud zanedbáme vzájemnou interakci obou elektronů.
    \item
        Za předpokladu, že $\operator{H_{I}}$ lze chápat jako poruchu, spočítejte opravu k energii základního stavu.
    \item
        Spočítejte korekci k degenerovanému excitovanému 1s2s stavu a určete vzdálenost rozštěpených hladin, které je způsobeno odpuzováním elektronů.
    \item 
        Helium vložíme do homogenního magnetického pole $\vector{B}=(0,0,B)$ mířícího ve směru osy $z$.
        Spočítejte, o kolik se změní enegie základního stavu. 
        Jaké znaménko bude mít tato změna a co to fyzikálně znamená?
    \item
        Předpokládejte, že v čase $t=0$ je jeden elektron atomu helia ve stavu 1s se spinem mířícím nahoru a druhý elektron ve stavu 2s se spinem mířícím dolů.
        Jaký bude stav systému v čase $t>0$?
        Odpuzování elektronů berte jako malou poruchu a spočítejte čas $T$, ve kterém se oba spiny \uv{prohodí}.
        Diskutujte platnost tohoto výpočtu.
\end{enumerate}


