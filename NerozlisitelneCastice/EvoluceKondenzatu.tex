\subsection{Evoluce bosonového kondenzátu}
	Soustava $N$ bosonů se nachází ve stavu kondenzátu
	\begin{equation}
		\ket{\psi_{0};N}\equiv\frac{1}{\sqrt{N!}}\left(\operatorconjugate{b}_{0}\right)^{N}\ket{0}.
	\end{equation}
	a systém je popsán jednočásticovým Hamiltoniánem
	\begin{equation}
		\operator{H}=\mathcal{E}\left(\operatorconjugate{B}\operator{b}_{0}+\operatorconjugate{b}_{0}\operator{B}\right),
	\end{equation}
	ve kterém $\mathcal{E}$ je reálný parametr udávající škálu energie a
	\begin{equation}
		\operatorconjugate{B}\equiv\sum_{k>0}z_{k}\operatorconjugate{b}_{k}
	\end{equation}
	(komplexní parametry $z_{k}$ nyní nemusí být narozdíl od předchozího příkladu normovány).
	Operátory $\operator{b}_{k}$, $\operatorconjugate{b}_{k}$ splňují bosonové komutační relace \eqref{eq:BosonCommutator}.
	
	Nalezněte časový vývoj stavu $\ket{\psi_{0}(t);N}$.

\begin{solution}
	Hledaný stav je časovou evolucí počátečního stavu,
	\begin{equation}
		\ket{\psi_{0}(t);N}=\operator{U}(t)\ket{\psi_{0};N},
	\end{equation}
	kde
	\begin{equation}
		\operator{U}(t)=\e^{-\frac{\im}{\hbar}\operator{H}t}.
	\end{equation}
	Začneme s jedním bosonem v kondenzátu a postupně budeme přidávat další.
	\begin{itemize}
	\item
		$N=1$:
		\begin{equation}
			\ket{\psi_{0}(t);1}=\e^{\overbrace{-\frac{\im}{\hbar}\mathcal{E}t}^{\alpha}\overbrace{\left(\operatorconjugate{B}\operator{b}_{0}+\operatorconjugate{b}_{0}\operator{B}\right)}^{\operator{X}}}\operatorconjugate{b}_{0}\ket{0}.
		\end{equation}
		Podle BCH formule~\eqref{eq:BCH} vyjádřené v operátorech $\operator{X},\operatorconjugate{b}_{0}$,
		\begin{align*}
			\e^{\alpha\operator{X}}\operatorconjugate{b}_{0}\e^{-\alpha\operator{X}}
				&=\sum_{k=0}^{\infty}\frac{\alpha^{k}}{k!}\operator{K}_{n}, &
			\operator{K}_{0}&=\operatorconjugate{b}_{0}\,,\\
			&&\operator{K}_{k+1}&=\commutator{\operator{X}}{\operator{K}_{n}}\,,
		\end{align*}
		vychází
		\begin{equation}
			\ket{\psi_{0}(t);1}
				=\underbrace{\left(\operatorconjugate{b}_{0}
					+\alpha\commutator{\operator{X}}{\operatorconjugate{b}_{0}}
					+\frac{\alpha^{2}}{2!}\commutator{\operator{X}}{\commutator{\operator{X}}{\operatorconjugate{b}_{0}}}
					+\frac{\alpha^{3}}{3!}\dotsb\right)}_{\operator{Z}}\underbrace{\e^{\alpha\operator{X}}\ket{0}}_{\ket{0}}.
		\end{equation}
		Jednotlivé členy přispějí následujícím způsobem:
        \begin{subequations}
            \begin{align}
                \commutator{\operator{X}}{\operatorconjugate{b}_{0}}
                    &=\commutator{\operatorconjugate{B}\operator{b}_{0}+\operatorconjugate{b}_{0}\operator{B}}{\operatorconjugate{b}_{0}}
                     =\commutator{\operatorconjugate{B}\operator{b}_{0}}{\operatorconjugate{b}_{0}}
                     =\operatorconjugate{B}\commutator{\operator{b}_{0}}{\operatorconjugate{b}_{0}}=\operatorconjugate{B}\,,\\
                \commutator{\operator{X}}{\commutator{\operator{X}}{\operatorconjugate{b}_{0}}}
                    &=\commutator{\operatorconjugate{B}\operator{b}_{0}+\operatorconjugate{b}_{0}\operator{B}}{\operatorconjugate{B}}
                     =\operatorconjugate{b}_{0}\commutator{\operator{B}}{\operatorconjugate{B}}=\abs{\vector{z}}^{2}\operatorconjugate{b}_{0}\,,\\
                \commutator{\operator{X}}{\commutator{\operator{X}}{\commutator{\operator{X}}{\operatorconjugate{b}_{0}}}}
                    &=\commutator{\operatorconjugate{B}\operator{b}_{0}+\operatorconjugate{b}_{0}\operator{B}}{\abs{\vector{z}}^{2}\operatorconjugate{b}_{0}}=\abs{\vector{z}}^{2}\operatorconjugate{B}\,,\\
                \dotsb
            \end{align}                
        \end{subequations}
		Po dosazení tedy vychází
		\begin{align}
			\ket{\psi_{0}(t);1}
				&=\left(\operatorconjugate{b}_{0}+\alpha\operatorconjugate{B}+\alpha^{2}\frac{\abs{\vector{z}}^{2}}{2!}\operatorconjugate{b}_{0}+\alpha^{3}\frac{\abs{\vector{z}}^{2}}{3!}\operatorconjugate{B}+\alpha^{4}\frac{\abs{\vector{z}}^{4}}{4!}\operatorconjugate{b}_{0}+\dotsb\right)\ket{0}\nonumber\\
				&=\Bigg[\underbrace{\left(1+\alpha^{2}\frac{\abs{\vector{z}}^{2}}{2!}+\alpha^{4}\frac{\abs{\vector{z}}^{4}}{4!}+\dotsb\right)}_{\cosh{\alpha\abs{\vector{z}}}}\operatorconjugate{b}_{0}					
					+\underbrace{\left(\alpha+\alpha^{3}\frac{\abs{\vector{z}}^{2}}{3!}+\alpha^{5}\frac{\abs{\vector{z}}^{4}}{4!}+\dotsb\right)}_{\frac{1}{\abs{\vector{z}}}\sinh{\alpha\abs{\vector{z}}}}\operatorconjugate{B}\Bigg]\ket{0}\nonumber\\
				&=\left[\operatorconjugate{b}_{0}\,\cosh{\alpha\abs{\vector{z}}}+\operatorconjugate{B}\,\frac{\sinh{\alpha\abs{\vector{z}}}}{\abs{\vector{z}}}\right]\ket{0}.
		\end{align}
			
	\item
		$N=2$:
		\begin{align}
			\ket{\psi_{0}(t);2}
				&=\e^{\alpha\operator{X}}\frac{\left(\operatorconjugate{b}_{0}\right)^{2}}{2!}\ket{0}=\frac{\operator{Z}}{2}\e^{\alpha\operator{X}}\operatorconjugate{b}_{0}\ket{0}=\frac{\operator{Z}^{2}}{2}\e^{\alpha\operator{X}}\ket{0}=\frac{\operator{Z}^{2}}{2}\ket{0}\nonumber\\
				&=\frac{1}{2}\left[\operatorconjugate{b}_{0}\,\cosh{\alpha\abs{\vector{z}}}+\operatorconjugate{B}\,\frac{\sinh{\alpha\abs{\vector{z}}}}{\abs{\vector{z}}}\right]^{2}\ket{0}\,.
		\end{align}
			
	\item
		$N$ obecné: Pomocí BCH formule se komutuje $N$-krát.
        Indukcí vychází
		\begin{align}
			\ket{\psi_{0}(t);N}
				&=\frac{1}{\sqrt{N!}}\left[\operatorconjugate{b}_{0}\,\cosh{\alpha\abs{\vector{z}}}+\operatorconjugate{B}\,\frac{\sinh{\alpha\abs{\vector{z}}}}{\abs{\vector{z}}}\right]^{N}\ket{0}\nonumber\\
				&=\frac{1}{\sqrt{N!}}\left[\operatorconjugate{b}_{0}\,\cos{\frac{\mathcal{E}t\abs{\vector{z}}}{\hbar}}+\im\frac{\operatorconjugate{B}}{\abs{\vector{z}}}\,\sin{\frac{\mathcal{E}t\abs{\vector{z}}}{\hbar}}\right]^{N}\ket{0}.
		\end{align}
		Stavový vektor tedy rotuje v komplexní rovině s frekvencí
		\begin{equation}
			\omega\equiv\frac{\mathcal{E}}{\hbar}\abs{\vector{z}}.
		\end{equation}		
	\end{itemize}
\end{solution}

\begin{note}
	O kondenzátorových koherentních stavech a jejich aplikacích se lze dočíst více v~článku~\cite{Caprio2005}.
\end{note}