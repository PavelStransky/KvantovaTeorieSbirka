\subsection{Rozptyl dvou částic se spinem}
Interakce dvou \emph{rozlišitelných} částic o hmotnosti $M$ se spinem $\frac{1}{2}$ je popsána potenciálem
\begin{equation}
    V=V_{0}(r)+V_{\ti{s}}(r)\,\vector{\sigma}^{(1)}\cdot\vector{\sigma}^{(2)}
\end{equation}
kde $r$ je vzájemná vzdálenost obou částic,
\begin{subequations}
    \begin{align}
        \vector{\sigma}^{(1)}
            &\equiv\vector{\sigma}\otimes\matrix{1},\\
        \vector{\sigma}^{(2)}
            &\equiv\matrix{1}\otimes\vector{\sigma}
    \end{align}        
\end{subequations}
jsou operátory na Hilbertově prostoru $\hilbert{H}=\hilbert{H}^{(1)}\otimes\hilbert{H}^{(2)}$ a	$\vector{\sigma}=(\sigma_{1},\sigma_{2},\sigma_{3})$ jsou Pauliho matice.
Potenciály $V_{0}(r)$, $V_{\ti{s}}(r)$ nejsou blíže specifikovány, předpokládejte pouze, že interakce jsou krátkodosahové, takže integrály
\begin{subequations}
    \begin{align}
        v_{0}(\vector{q})
            =\int\e^{\im\vector{q}\cdot\vector{r}}V_{0}(r)\d^{3}\vector{r}=\frac{4\pi}{q}\int_{0}^{\infty}rV_{0}(r)\sin{qr}\,\d r,\\
        v_{\ti{s}}(\vector{q})
            =\int\e^{\im\vector{q}\cdot\vector{r}}V_{\ti{s}}(r)\d^{3}\vector{r}=\frac{4\pi}{q}\int_{0}^{\infty}rV_{s}(r)\sin{qr}\,\d r,
    \end{align}        
\end{subequations}
kde $\vector{q}=\vector{k}-\vector{k'}$, existují a jsou konečné.

\begin{enumerate}
    \item
        V rámci 1. Bornovy aproximace vyjádřete amplitudy rozptylu 
        \begin{align*}
            &f_{\uparrow\uparrow\longrightarrow\uparrow\uparrow}(\theta)
            &&f_{\downarrow\downarrow\longrightarrow\downarrow\downarrow}(\theta)\\
            &f_{\uparrow\downarrow\longrightarrow\uparrow\downarrow}(\theta)
            &&f_{\uparrow\downarrow\longrightarrow\downarrow\uparrow}(\theta)            
        \end{align*}
        a příslušné diferenciální účinné průřezy $\frac{\d\sigma}{\d\Omega}$.
        $\theta$ označuje úhel mezi vektory $\vector{k}$ a $\vector{k'}$ a šipky udávají orientace jednotlivých spinů před interakcí a po interakci (poslední amplituda rozptylu udává případ, kdy si jednotlivé částice prohodí spin).

    \item 
        Jak se změní diferenciální účinné průřezy, pokud obě částice budou~\emph{nerozlišitelné fermiony}?

    \item
        Uvažujte speciální případ Gaussovské interakce 
        \begin{equation}
            V_{0}(r)=V_{s}(r)=v\e^{-\mu r^2}
        \end{equation}
        Nalezněte, pro jaké úhly $\theta$ budou diferenciální účinné průřezy 
        \begin{align}
            &\left.\derivative{\sigma}{\Omega}\right|_{\uparrow\uparrow\longrightarrow\uparrow\uparrow}
            &&\left.\derivative{\sigma}{\Omega}\right|_{\uparrow\downarrow\longrightarrow\uparrow\downarrow}
            &&\left.\derivative{\sigma}{\Omega}\right|_{\uparrow\downarrow\longrightarrow\downarrow\uparrow}
        \end{align}
        maximální, a to jak v případě rozlišitelných částic, tak v případě částic nerozlišitelných.
\end{enumerate}

