\subsection{Gaussovská porucha fermionového systému}
\label{sec:GaussovskaPoruchaFermiony}
	Zadání je stejné jako v předchozím příkladu \ref{sec:IdenticalBosons}, avšak nyní jsou částice fermiony se spinem $\frac{1}{2}$.
	Spočítejte v prvním řádu poruchové teorie opravu k energii základního stavu pro singletní i tripletní spinový stav.
	
\begin{solution}
	Vlnová funkce dvou stejných fermionů je
	\begin{equation}
		\psi^{\ti{F}}(x_{1},x_{2})
			=\phi^{\ti{F}}(x_{1},x_{2})\Sigma_{S\xi}
	\end{equation}
	kde $\phi^{\ti{F}}(x_{1},x_{2})$ je prostorová část, $\Sigma_{S\xi}$ část spinová.
	Dva spiny o velikosti $\frac{1}{2}$ se složí buď na celkový spin $S=1$ --- tripletní stav ---, který je symetrický vůči záměně částic $1\leftrightarrow2$, nebo na spin $S=0$ --- singletní
	stav ---, který je vůči záměně antisymetrický.
	Vlnová funkce fermionového systému musí být antisymetrická.
	Z toho vyplývá, že její prostorová část musí být
	\begin{itemize}
		\item \emph{symetrická} pro singletní stav
		\item \emph{antisymetrická} pro tripletní stav.
	\end{itemize}
	Prostorová část vlnové funkce pro singletní stav tudíž vypadá stejně jako v případě bosonů~\eqref{eq:IdenticalBosonsWaveFunction},
	\begin{equation}
		\phi^{\ti{F}}_{0,S=0}(x_{1},x_{2})
			=\phi_{0}(x_{1})\phi_{0}(x_{2})
			=\sqrt{\frac{M\Omega}{2\pi\hbar}}\e^{-\frac{M\Omega}{2\hbar}\left(x_{1}^{2}+x_{2}^{2}\right)},
	\end{equation}
	a tím pádem také oprava k energii vyjde stejně:
	\begin{equation}
		E_{0,S=0}^{\ti{F}(1)}
			=\frac{v}{2}\sqrt{\frac{M\Omega}{M\Omega+2\hbar\alpha}}
	\end{equation}
	
	U prostorové části vlnové funkce stavu tripletního si již nelze vystačit s jednočásticovou vlnovou funkcí $\psi_{0}$.
	Antisymetrizovat se dá pouze součin rozdílných jednočásticových vlnových funkcí.
    Jedna z částic se tedy musí nacházet v prvním vzbuzeném energetickém stavu $\phi_{1}(x)$,
	\begin{equation}
		\phi^{\ti{F}}_{0,S=1}(x_{1},x_{2})
			=\frac{1}{\sqrt{2}}\left[\phi_{0}(x_{1})\phi_{1}(x_{2})-\phi_{1}(x_{1})\phi_{0}(x_{2})\right],
	\end{equation}
	přičemž stav $\phi_{1}(x)$ se určí napříkad aplikováním posunovacího operátoru $\operatorconjugate{a}$~\eqref{eq:ShiftOperatorToPX} na vlnovou funkci $\phi_{0}(x)$:
	\begin{align}
		\phi_{1}(x)
			&=\sqrt{\frac{M\Omega}{2\hbar}}\left(x-\frac{\hbar}{M\Omega}\frac{\partial}{\partial x}\right)
				\sqrt[4]{\frac{M\Omega}{2\pi\hbar}}\e^{-\frac{M\Omega}{2\hbar}x^{2}}=\nonumber\\
			&=\phi_{0}(x)\sqrt{\frac{M\Omega}{2\hbar}}\left(x+\frac{\hbar}{M\Omega}\frac{M\Omega}{\hbar}x\right)\nonumber\\
			&=x\phi_{0}(x)\sqrt{\frac{2M\Omega}{\hbar}}.
	\end{align}
	Celková vlnová funkce tedy má tvar
	\begin{align}
		\phi^{\ti{F}}_{0,S=1}(x_{1},x_{2})
			&=\frac{1}{\sqrt{2}}\left[\phi_{0}(x_{1})x_{2}\phi_{0}(x_{2})\sqrt{\frac{2M\Omega}{\hbar}}-x_{1}\phi_{0}(x_{1})\sqrt{\frac{2M\Omega}{\hbar}}\phi_{0}(x_{2})\right]\nonumber\\
			&=\sqrt{\frac{M\Omega}{\hbar}}\left(x_{2}-x_{1}\right)\phi_{0}(x_{1})\phi_{0}(x_{2}).
	\end{align}
	Neporušená hodnota energie je v tomto stavu
	\begin{equation}
		E_{0,S=1}^{\ti{F}(0)}
			=\hbar\omega\left(0+\frac{1}{2}\right)+\hbar\omega\left(1+\frac{1}{2}\right)
			=2\hbar\omega
	\end{equation}
	a příspěvek 1. řádu poruchové teorie zní
	\begin{align}
		E_{0,S=1}^{\ti{F}(1)}
			&=\iint\psi^{\ti{F}*}_{0,S=1}(x_{1},x_{2})\operator{H}_{\ti{I}}\psi_{0,S=1}^{\ti{F}}(x_{1},x_{2})\d x_{1}\d x_{2}=\nonumber\\
			&=v\frac{M\Omega}{\hbar}\frac{M\Omega}{2\pi\hbar}
				\iint\left(x_{1}-x_{2}\right)^{2}\e^{-\frac{M\Omega}{\hbar}\left(x_{1}^{2}+x_{2}^{2}\right)}
				\e^{-\alpha\left(x_{1}-x_{2}\right)^{2}}\d x_{1}\d x_{2}=\nonumber\\
			&=\frac{v}{2\pi}\left(\frac{M\Omega}{\hbar}\right)^{2}
				\iint\left(x_{1}-x_{2}\right)^{2}\e^{-\frac{M\Omega}{\hbar}\left(x_{1}^{2}+x_{2}^{2}\right)}
				\e^{-\alpha\left(x_{1}-x_{2}\right)^{2}}\d x_{1}\d x_{2}=\nonumber\\
			&=\frac{v}{2\pi}\left(\frac{M\Omega}{\hbar}\right)^{2}
				\int\e^{-\frac{2M\Omega}{\hbar}X^{2}}\d X
				\int x^{2}\e^{-\left(\frac{M\Omega}{2\hbar}+\alpha\right)x^{2}}\d x=\nonumber\\
			&=\frac{v}{2\pi}\left(\frac{M\Omega}{\hbar}\right)^{2}
				\sqrt{\frac{\pi\hbar}{2M\Omega}}\,
				\frac{1}{2}\frac{\pi}{\frac{M\Omega}{2\hbar}+\alpha}
				\sqrt{\frac{\pi}{\frac{M\Omega}{2\hbar}+\alpha}}=\nonumber\\
			&=\frac{v}{2}\left(\frac{M\Omega}{M\Omega+2\hbar\alpha}\right)^{3/2}.
	\end{align}
\end{solution}
