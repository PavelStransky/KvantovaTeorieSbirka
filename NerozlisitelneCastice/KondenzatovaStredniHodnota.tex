\subsection{Kondenzátová střední hodnota}
	Je zadán bosonový $N$-částicový kondenzát\footnote{
		Tento příklad je vyřešen a diskutován v učebnici~\cite{Cejnar2013}.
	}
	\begin{equation}
		\ket{B;N}
			\equiv\frac{1}{\sqrt{N!}}\left(\operatorconjugate{B}\right)^{N}\ket{0}
	\end{equation}
	kde operátor $\operatorconjugate{B}$ je vyjádřen lineární kombinací
	\begin{equation}
		\operatorconjugate{B}
			\equiv\sum_{k}z_{k}\operatorconjugate{b}_{k}
	\end{equation}
	a $\operatorconjugate{b}_{k}$ bosonové operátory, splňující komutační relace \eqref{eq:BosonCommutator}.
	Komplexní čísla $z_{k}$ jsou normována:
	\begin{equation}
		\abss{z}\equiv\sum_{k}z_{k}^{*}z_{k}=1
	\end{equation}

	Spočítejte střední hodnotu $\matrixelement{B;N}{\operator{H}}{B;N}$, kde operátor $\operator{H}$ je složen z jednočásticové a dvoučásticové části
	\begin{equation}
		\operator{H}
			=\sum_{ij}\epsilon_{ij}\operatorconjugate{b}_{i}\operator{b}_{j}
				+\frac{1}{2}\sum_{klmn}v_{klmn}\operatorconjugate{b}_{k}\operatorconjugate{b}_{l}\operator{b}_{n}\operator{b}_{m},
	\end{equation}
	přičemž $\epsilon_{ij}$, $v_{klmn}$ jsou komplexní čísla.
	
	\begin{solution}
		Pokud se prokomutují všechny anihilační operátory napravo 
		skrz kreační operátory kondenzátu, dá jejich působení na stav vakua nulový příspěvek.
		
		Komutace skrz jeden operátor $\operatorconjugate{B}$ vede na
		\begin{equation}
			\operator{K}_{1}
				\equiv\commutator{\operator{b}_{j}}{\operatorconjugate{B}}
				=\sum_{k}z_{k}\commutator{\operator{b}_{j}}{\operatorconjugate{b}_{k}}
				=\sum_{k}z_{k}\delta_{jk}
				=z_{j}.
		\end{equation}
		Indukcí se pak dostane
		\begin{align}
			\operator{K}_{N}
				&\equiv\commutator{\operator{b}_{j}}{(\operatorconjugate{B})^{N}}=\nonumber\\
				&=\commutator{\operator{b}_{j}}{\operatorconjugate{B}}(\operatorconjugate{B})^{N-1}+\operatorconjugate{B}\commutator{\operator{b}_{j}}{(\operatorconjugate{B})^{N-1}}=\nonumber\\
				&=z_{j}(\operatorconjugate{B})^{N-1}+\operatorconjugate{B}\operator{K}_{N-1},
		\end{align}
		neboli
		\begin{align*}
			\operator{K}_{2}
				&=z_{j}\operatorconjugate{B}+\operatorconjugate{B}z_{j}=2z_{j}\operatorconjugate{B}\\
			\operator{K}_{3}
				&=z_{j}(\operatorconjugate{B})^{2}+\operatorconjugate{B}(2z_{j}\operatorconjugate{B})=3z_{j}(\operatorconjugate{B})^{2}\\
				&\vdots\\
			\operatorconjugate{K}_{N}
				&=Nz_{j}(\operatorconjugate{B})^{N-1}.
		\end{align*}
		Působení anihilačního operátoru $\operator{b}_{j}$ na kondenzát dává
		\begin{align}
			\operator{b}_{j}\ket{B;N}
				&=\frac{1}{\sqrt{N!}}\,\operator{b}_{j}(\operatorconjugate{B})^{N}\ket{0}=\\
				&=\frac{1}{\sqrt{N!}}\,Nz_{j}(\operatorconjugate{B})^{N-1}\ket{0}+\frac{1}{\sqrt{N!}}(\operatorconjugate{B})^{N}\operator{b}_{j}\ket{0}=\\
				&=\frac{N}{\sqrt{N}}\,z_{j}\frac{1}{\sqrt{(N-1)!}}(\operatorconjugate{B})^{N-1}\ket{0}=\\
				&=z_{j}\sqrt{N}\,\ket{B;N-1}.
		\end{align}
	
		Střední hodnota Hamiltoniánu v kondenzátovém stavu tedy je
		\begin{equation}
			\matrixelement{B;N}{\operator{H}}{B;N}
				=N\sum_{ij}\epsilon_{ij}z_{i}^{*}z_{j}
					+\frac{N(N-1)}{2}\sum_{klmn}v_{klmn}z_{k}^{*}z_{l}^{*}z_{m}z_{n}
		\end{equation}
		Podobně lze postupovat při výpočtu kondenzátové střední hodnoty vícečásticových operátorů.
	\end{solution}
