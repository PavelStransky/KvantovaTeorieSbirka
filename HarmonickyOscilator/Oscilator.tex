\subsection{Jednorozměrný harmonický oscilátor}\label{sec:HarmonicOscillator}\index{harmonický oscilátor}
Harmonický oscilátor je popsán Hamiltoniánem
\begin{equation}
    \important{
        \operator{H}
            =\frac{1}{2M}\operator{p}^{2}+\frac{1}{2}M\Omega^{2}\operator{x}^{2},
    }
    \label{eq:HOHamiltonianXY}
\end{equation}
kde $M$ je hmotnost kmitající částice, $\Omega=\sqrt{k/M}$ je úhlová frekvence oscilátoru, $\operator{x}$ je operátor souřadnice a $\operator{p}$ operátor k němu přidružené hybnosti.
Oba tyto samosdružené operátory splňují kanonický komutační vztah\index{komutační relace!kanonické}
\begin{equation}
    \commutator{\operator{x}}{\operator{p}}
        =\im\hbar.
\end{equation}

V harmonickém oscilátoru lze vhodně nadefinovat operátory $\operator{a}$, $\conjugate{\operator{a}}$, a tím ho převést na algebraický systém, který jsme vyřešili v předchozím příkladu~\ref{sec:ShiftOperators}.
Hledejte $\operator{a}$ ve tvaru
\begin{equation}
    \operator{a}
        =\alpha{\operator{x}}+\beta{\operator{p}},\qquad 
        \alpha,\beta\in\mathbb{C}.
    \label{eq:HOA}
\end{equation}

\begin{enumerate}
\item 
    Nalezněte hodnoty konstant $\alpha$, $\beta$
    a zapište Hamiltonián $\operator{H}$ pomocí operátorů $\operator{a}$, $\conjugate{\operator{a}}$.

\item 
    Nalezněte spektrum (vlastní energie a vlastní vektory) Hamiltoniánu.

\item 
    Vyjádřete operátory hybnosti $\operator{p}$ a souřadnice $\operator{x}$ pomocí operátorů $\operator{a}$, $\conjugate{\operator{a}}$.

\item Spočítejte střední hodnoty 
    \begin{equation}
        \matrixelement{n}{\operator{x}}{n},\quad
        \matrixelement{n}{\operator{p}}{n},\quad
        \matrixelement{n}{\operator{x}^{2}}{n},\quad 
        \matrixelement{n}{\operator{p}^{2}}{n},
    \end{equation}
    kde $\ket{n}$ je vlastní stav Hamiltoniánu.

\item 
    Spočítejte střední hodnoty 
    \begin{equation}
        \matrixelement{n}{\operator{T}}{n},\qquad
        \matrixelement{n}{\operator{V}}{n},
    \end{equation}		
    kde $\operator{T}$ a $\operator{V}$ jsou operátory kinetické, resp. potenciální energie oscilátoru, 
    a srovnejte s hodnotou energie ve stavu $\ket{n}$ (viriálový teorém).\index{teorém!viriálový}

\item 
    Ověřte platnost relací neurčitosti mezi polohou a hybností.

\item 
    Pomocí posunovacích operátorů vyjádřených v $x$-reprezentaci nalezněte vlnové funkce Harmonického oscilátoru.
\end{enumerate}

\begin{solution}
	\begin{enumerate}
	\item 
        Dosazení lineární kombinace souřadnic a hybností~\eqref{eq:HOA} do definice operátoru $\operator{n}$ dá
		\begin{align}
			\operator{n}=\conjugate{\operator{a}}\operator{a}
				&=\left(\alpha^{*}\operator{x}+\beta^{*}\operator{p}\right)
					\left(\alpha\operator{x}+\beta\operator{p}\right)\nonumber\\
				&=\abs{\alpha}^{2}\operator{x}^{2}
					+\alpha^{*}\beta\underbrace{\operator{x}\operator{p}}_{\operator{p}\operator{x}+\im\hbar}
					+\alpha\beta^{*}\operator{p}\operator{x}+\abs{\beta}^{2}\operator{p}^{2}\nonumber\\
				&=\abs{\alpha}^{2}\operator{x}^{2}+\left(\alpha^{*}\beta+\alpha\beta^{*}\right)\operator{p}\operator{x}
					+\im\hbar\alpha^{*}\beta+\abs{\beta}^{2}\operator{p}^{2}.
				\label{eq:HONAlphaBeta}
		\end{align}
		Srovnáním tohoto výrazu s Hamiltoniánem~\eqref{eq:HOHamiltonianXY} je patrné, že až na konstantní člen lze Hamiltonián harmonického oscilátoru zkonstruovat z operátoru $\operator{n}$, pokud vymizí členy mísící souřadnici a hybnost, tj. pokud
		\begin{equation}
            \alpha^{*}\beta+\alpha\beta^{*}=0
            \qquad
            \Longleftrightarrow
            \qquad
            \real\alpha^{*}\beta=0.
		\end{equation}
        Bez újmy na obecnosti lze zvolit $\alpha$ reálné a $\beta$ ryze imaginární.
        Pak
		\begin{equation}
            \operator{H}
                =\gamma\operator{n}+\delta,
		\end{equation}
		kde $\gamma$ a $\delta$ jsou již reálná čísla.
			
		Dodatečná podmínka plyne z komutačních relací~\eqref{eq:ShiftOperatorCommutator},
		\begin{align}
			1=\commutator{\operator{a}}{\conjugate{\operator{a}}}
				&=\left(\alpha\operator{x}+\beta\operator{p}\right)\left(\alpha^{*}\operator{x}+\beta^{*}\operator{p}\right)
					-\left(\alpha^{*}\operator{x}+\beta^{*}\operator{p}\right)\left(\alpha\operator{x}+\beta\operator{p}\right)\nonumber\\
				&=\left(\alpha\beta^{*}
					-\alpha^{*}\beta\right)\underbrace{\operator{x}\operator{p}}_{\operator{p}\operator{x}+\im\hbar}
					+\left(\alpha^{*}\beta-\alpha\beta^{*}\right)\operator{p}\operator{x}\nonumber\\
				&=\left(\alpha\beta^{*}-\alpha^{*}\beta\right)\im\hbar\nonumber\\
				&=2\alpha\beta^{*}\im\hbar,
		\end{align}
		takže musí platit
		\begin{equation}\label{eq:HOAlphaBeta*}
			\alpha\beta^{*}=\frac{1}{2\im\hbar}.
		\end{equation}
		
		Srovnáním příslušných členů výrazu~\eqref{eq:HONAlphaBeta} s Hamiltoniánem~\eqref{eq:HOHamiltonianXY}
        a díky podmínce~\eqref{eq:HOAlphaBeta*} lze přiřadit parametrům hodnoty
        \begin{subequations}
            \begin{align}
                \alpha
                    &=\frac{1}{\sqrt{\gamma}}\sqrt{\frac{1}{2}M\Omega^{2}},\\
                \beta
                    &=\frac{\im}{\sqrt{\gamma}}\sqrt{\frac{1}{2M}},\\
                \gamma
                    &=\hbar\Omega,\\
                \delta
                    &=\frac{\gamma}{2}=\frac{\hbar\Omega}{2}.			
            \end{align}
			\label{eq:HOAlphaBeta}
        \end{subequations}
		Výsledkem je vyjádření Hamiltoniánu harmonického oscilátoru ve tvaru
		\begin{equation}
			\important{\operator{H}=\hbar\Omega\left(\conjugate{\operator{a}}\operator{a}+\frac{1}{2}\right)}.
            \label{eq:HOHamiltonianAA+}
    	\end{equation}
		
	\item
		Spektrum Hamiltoniánu lze určit na základě znalosti spektra operátoru $\operator{n}$~\eqref{eq:EigenstateNumberOperator}:
		\begin{align}
            \operator{H}\ket{E_{n}}
                &=E_{n}\ket{E_{n}}\nonumber\\
			\hbar\Omega\left(\conjugate{\operator{a}}\operator{a}\ket{E_{n}}+\frac{1}{2}\ket{E_{n}}\right)
				&=E_{n}\ket{E_{n}}\nonumber\\
            \hbar\Omega\left(n+\frac{1}{2}\right)\ket{n}
                &=E_{n}\ket{n},
		\end{align}
		takže
		\begin{equation}
            \important{E_{n}
                =\hbar\Omega\left(n+\frac{1}{2}\right),\quad n\in\mathbb{N}_{0}
            }
            \label{eq:HOEnergy}
    	\end{equation}
		a vlastní vektory jsou totožné s vlastními vektory operátoru $\operator{n}$, $\ket{E_{n}}\equiv\ket{n}$.
	
	\item
		Dosazení $\alpha$, $\beta$ a $\gamma$ z~\eqref{eq:HOAlphaBeta} do~\eqref{eq:HOA}, vede na vztahy\sfootnote{
			Lze navíc zavést veličiny rozměru souřadnice a hybnosti
			\begin{equation}
                x_{0}
                    \equiv\sqrt{\frac{\hbar}{M\Omega}},\qquad
                p_{0}
                    \equiv\frac{x_{0}}{\hbar}=\sqrt{\hbar M\Omega}
			\end{equation}
			a vztahy~\eqref{eq:ShiftOperatorToPX} přepsat do bezrozměrného tvaru
			\begin{equation}
                \operator{a}
                    =\frac{1}{\sqrt{2}}\left(\frac{\operator{x}}{x_{0}}+\im\frac{\operator{p}}{p_{0}}\right).
			\end{equation}
		}
		\begin{subequations}
			\begin{align}
				\operator{a}
					=\frac{\operator{A}}{\sqrt{\hbar\Omega}}
					&=\frac{1}{\sqrt{\hbar\Omega}}\left(\sqrt{\frac{1}{2}M\Omega^{2}}\operator{x}
						+\im\sqrt{\frac{1}{2M}}\operator{p}\right)\nonumber\\
					&=\sqrt{\frac{M\Omega}{2\hbar}}\left(\operator{x}+\frac{\im}{M\Omega}\operator{p}\right),\\
				\conjugate{\operator{a}}
					&=\sqrt{\frac{M\Omega}{2\hbar}}\left(\operator{x}-\frac{\im}{M\Omega}\operator{p}\right).
			\end{align}			
			\label{eq:ShiftOperatorToPX}
		\end{subequations}
		Sečtení a odečtení vede k inverzní transformaci od posunovacích operátorů k operátorům souřadníce a hybnosti,
        \begin{subequations}
            \begin{empheq}[box=\fbox]{align}
                \operator{x}
                    &=\sqrt{\frac{\hbar}{2M\Omega}}\left(\conjugate{\operator{a}}+\operator{a}\right),\\
                \operator{p}
                    &=\im\sqrt{\frac{\hbar M\Omega}{2}}\left(\conjugate{\operator{a}}-\operator{a}\right).
			\end{empheq}
			\label{eq:PXToShiftOperator}		
		\end{subequations}
	
	\item
		K výpočtu středních hodnot operátorů $\operator{x}$, $\operator{p}$ se využije jejich vyjádření pomocí posunovacích operátorů~\eqref{eq:PXToShiftOperator} a poté vztahy~\eqref{eq:AN}:
		\begin{align}
			\matrixelement{n}{\operator{x}}{n}
				&=\sqrt{\frac{\hbar}{2M\Omega}}\matrixelement{n}{\conjugate{\operator{a}}+\operator{a}}{n}\nonumber\\
				&=\sqrt{\frac{\hbar}{2M\Omega}}\left(\sqrt{n+1}\braket{n}{n+1}
					+\sqrt{n}\braket{n}{n-1}\right)\nonumber\\
				&=0,
				\label{eq:MatrixElementNXN}
		\end{align}
		jelikož vlastní vektory $\ket{n-1}$, $\ket{n}$ a $\ket{n+1}$ jsou na sebe kolmé.
        Stejně tak vychází
		\begin{equation}
			\label{eq:MatrixElementNPN}
			\matrixelement{n}{\operator{p}}{n}=0.
		\end{equation}
		Lze odpozorovat \trick{pravidlo}, že střední hodnota $\matrixelement{n}{f(r,s;\operator{a},\conjugate{\operator{a}})}{n}$, kde $f(r,s;\operator{a},\conjugate{\operator{a}})$ je funkce součinu $r$ operátorů $\operator{a}$ a $s$ operátorů $\conjugate{\operator{a}}$ v libovolném pořadí, je nenulová pouze tehdy, pokud $r=s$.
		Obecněji maticový element $\matrixelement{m}{f(r,s;\operator{a},\conjugate{\operator{a}})}{n}$ je nenulový, pokud $m+r=n+s$.
	
		Pro střední hodnoty kvadrátů operátorů souřadnice a hybnosti platí
		\begin{align}
			\matrixelement{n}{\operator{x}^{2}}{n}
				&=\frac{\hbar}{2M\Omega}\matrixelement{n}{\left(\conjugate{\operator{a}}+\operator{a}\right)^{2}}{n}\nonumber\\
				&=\frac{\hbar}{2M\Omega}\matrixelement{n}{\underbrace{\left(\conjugate{\operator{a}}\right)^{2}}_{0}+\operator{a}\conjugate{\operator{a}}
					+\conjugate{\operator{a}}\operator{a}+\underbrace{\operator{a}^{2}}_{0}}{n}\nonumber\\
				&=\frac{\hbar}{2M\Omega}\matrixelement{n}{\sqrt{n+1}\sqrt{n+1}+\sqrt{n}\sqrt{n}}{n}\nonumber\\
				&=\frac{\hbar}{2M\Omega}(2n+1),
				\label{eq:MatrixElementNX2N}\\
			\matrixelement{n}{\operator{p}^{2}}{n}
				&=-\frac{\hbar M\Omega}{2}\matrixelement{n}{\left(\conjugate{\operator{a}}-\operator{a}\right)^{2}}{n}\nonumber\\
				&=-\frac{\hbar M\Omega}{2}\matrixelement{n}{\underbrace{\left(\conjugate{\operator{a}}\right)^{2}}_{0}-\operator{a}\conjugate{\operator{a}}
					-\conjugate{\operator{a}}\operator{a}+\underbrace{\left(\conjugate{\operator{a}}\right)^{2}}_{0}}{n}\nonumber\\
				&=\frac{\hbar M\Omega}{2}(2n+1).
				\label{eq:MatrixElementNP2N}
		\end{align}
	
	\item
		Operátor kinetické energie je
		\begin{equation}
            \operator{T}
                =\frac{1}{2M}\operator{p}^{2},
		\end{equation}
		jehož střední hodnota vychází po dosazení~\eqref{eq:MatrixElementNP2N}
		\begin{equation}\label{eq:MatrixElementT}
			\matrixelement{n}{\operator{T}}{n}
				=\frac{1}{2M}\frac{\hbar M\Omega}{2}(2n+1)
				=\frac{1}{2}\hbar\Omega\left(n+\frac{1}{2}\right)
				=\frac{E_{n}}{2}.
		\end{equation}
		Podobně potenciál
		\begin{equation}
            \operator{V}
                =\frac{1}{2}M\Omega^{2}\operator{x}^{2}
		\end{equation}
		po dosazení~\eqref{eq:MatrixElementNX2N} vychází
		\begin{equation}\label{eq:MatrixElementV}
			\matrixelement{n}{\operator{V}}{n}
				=\frac{1}{2}M\Omega^{2}\frac{\hbar}{2M\Omega}(2n+1)
				=\frac{1}{2}\hbar\Omega\left(n+\frac{1}{2}\right)
				=\frac{E_{n}}{2}.
		\end{equation}
	
		Viriálový teorém\index{teorém!viriálový} udává vztah mezi střední hodnotou operátoru kinetické energie 
		a operátoru potenciálu pro libovolný stav $\ket{\psi}$,\sfootnote{
			Výrazu $\derivative{}{\operator{x}}\operator{V}(\operator{x})$ je třeba rozumět ve smyslu $\derivative{}{x}V(x)|_{x=\operator{x}}$.\index{derivace!operátoru}
		} tedy nikoliv jen pro vlastní stav Hamiltoniánu:
		\begin{equation}
			2\matrixelement{\psi}{\operator{T}}{\psi}
				=\matrixelement{\psi}{\operator{x}\derivative{}{\operator{x}}\operator{V}(\operator{x})}{\psi}.
		\end{equation}
		To se v případě, že $\operator{V}(\operator{x})$ je homogenní funkce stupně $s$, dále zjednoduší na
		\begin{equation}
			2\matrixelement{\psi}{\operator{T}}{\psi}=s\matrixelement{\psi}{\operator{V}}{\psi}.
		\end{equation}
        V případě harmonického oscilátoru je $s=2$. 
        Pro střední hodnoty kinetické energie~\eqref{eq:MatrixElementT} a potenciálu~\eqref{eq:MatrixElementV} je viriálový teorém splněn.
		
	\item
		Relace neurčitosti\index{relace neurčitosti} znějí
		\begin{equation}
			\uncertainty{n}{x}\uncertainty{n}{p}\geq\frac{\hbar^{2}}{4},
		\end{equation}
		kde
		\begin{subequations}
			\begin{align}
				\uncertainty{n}{x}
					&=\matrixelement{n}{\operator{x}^{2}}{n}-\matrixelement{n}{\operator{x}}{n}^{2},\\
					\uncertainty{n}{p}
					&=\matrixelement{n}{\operator{p}^{2}}{n}-\matrixelement{n}{\operator{p}}{n}^{2}.
			\end{align}
		\end{subequations}
		Po dosazení vypočtených středních hodnot~\eqref{eq:MatrixElementNXN}, \eqref{eq:MatrixElementNPN}, \eqref{eq:MatrixElementNX2N} a \eqref{eq:MatrixElementNP2N} do relací neurčitosti vychází
		\begin{align}
			\uncertainty{n}{x}\uncertainty{n}{p}
				&=\matrixelement{n}{\operator{x}^{2}}{n}\matrixelement{n}{\operator{p}^{2}}{n}\nonumber\\
				&=\frac{\hbar}{2M\Omega}(2n+1)\frac{\hbar M\Omega}{2}(2n+1)\nonumber\\
				&=\frac{\hbar^{2}}{4}(2n+1)^{2}.
		\end{align}
		Jelikož $n\geq0$, relace neurčitosti jsou splněny.
        Stav s nejmenší možnou neurčitostí je základní stav $n=0$.
        
    \item 
        Operátory $\operator{a}$ a $\conjugate{\operator{a}}$ v $x$-reprezentaci se získají ze vztahů~\eqref{eq:ShiftOperatorToPX} přechodem $\operator{x}\mapsto x$ a $\operator{p}\mapsto-\im\hbar\,\d/\d x$,
        \begin{equation}\label{ShiftOperatorXRepresentation}
            \operator{a}
                =\sqrt{\frac{M\Omega}{2\hbar}}\left(x+\frac{\hbar}{M\Omega}\derivative{}{x}\right).
        \end{equation}

        Pro základní stav harmonického oscilátoru platí
        \begin{equation}
            \operator{a}\ket{0}=0
            \xrightarrow{x-\text{reprezentace}}
            \operator{a}\psi_{0}(x)=0,
        \end{equation}
        což vede v $x$-reprezentaci na obyčejnou diferenciální rovnici 1. řádu
        \begin{equation}
            \left(x+\frac{\hbar}{M\Omega}\derivative{}{x}\right)\psi_{0}(x)=0.
        \end{equation}
        Její řešení získané separací proměnných zní
        \begin{equation}
            \psi_{0}(x)
                =N\e^{-\frac{M\Omega}{2\hbar}x^{2}}.
        \end{equation}
        Normalizační konstanta $N$ vychází z podmínky
        \begin{equation}
            \int_{-\infty}^{\infty}\abs{\psi_{0}(x)}^{2}\d x=1
            \Longrightarrow
            N=\sqrt[4]{\frac{M\Omega}{\pi\hbar}}.
        \end{equation}
        Normalizovaná vlnová funkce základního stavu harmonického oscilátoru je tedy
        \begin{equation}
            \psi_{0}(x)
                =\sqrt[4]{\frac{M\Omega}{\pi\hbar}}\e^{-\frac{M\Omega}{2\hbar}x^{2}}.
        \end{equation}

        Vlnové funkce vzbuzených stavů se nagenergují ze základního stavu působením operátoru $\conjugate{\operator{a}}$ podle vztahu~\eqref{eq:EigenstateNumberOperator},
        \begin{align}
            \psi_{n}(x)
                =\frac{1}{\sqrt{n!}}\left(\frac{M\Omega}{2\hbar}\right)^{\frac{n}{2}}\left(x-\frac{\hbar}{M\Omega}\derivative{}{x}\right)^{n}\psi_{0}(x)
                =\frac{1}{\sqrt{2^{n}n!}}\sqrt[4]{\frac{M\Omega}{\pi\hbar}}\e^{-\frac{M\Omega}{2\hbar}x^{2}}H_{n}\left(\sqrt{\frac{M\Omega}{\hbar}}x\right),
        \end{align}
        kde $H_{n}(\xi)$ je Hermitův polynom.\index{polynom!Hermitův}
    \end{enumerate}    
\end{solution}	
