\subsection[Vlastnosti posunovacích operátorů]{Spektrum operátoru $\operator{n}=\operatorconjugate{a}\operator{a}$}\index{operátor!počtu}\label{sec:ShiftOperators}
Je zadán operátor $\operator{a}$ a operátor $\operatorconjugate{a}$ k němu sdružený, které mezi sebou splňují komutační relace\footnote{
    Vztahu~\eqref{eq:ShiftOperatorCommutator} je třeba rozumět ve smyslu $\commutator{\operator{a}}{\conjugate{\operator{a}}}=\operator{1}$, kde $\operator{1}$ je operátor identity, podobně jako je tomu například u komutačních relací samosdružených operátorů souřadnice a hybnosti $\commutator{\operator{x}}{\operator{p}}=\im\hbar\operator{1}$, které se běžně zkráceně zapisují $\commutator{\operator{x}}{\operator{p}}=\im\hbar$.
}\footnote{
    Obecněji lze uvažovat komutační relace
    \begin{equation}
        \commutator{\operator{A}}{\conjugate{\operator{A}}}
            =m,\qquad 
        m\in\mathbb{R}^{+}.
    \end{equation}
    Ty přejdou na~\eqref{eq:ShiftOperatorCommutator} přeškálováním $\operator{a}=\operator{A}/\sqrt{m}$.		
}
\begin{equation}
    \commutator{\operator{a}}{\conjugate{\operator{a}}}
        =1.
    \label{eq:ShiftOperatorCommutator}
\end{equation}
Definujme operátor 
\begin{equation}
    \operator{n}
        \equiv\conjugate{\operator{a}}\operator{a}.
    \label{eq:NumberOperator}
\end{equation}

\begin{enumerate}
\item 
    Ukažte, že operátor $\operator{n}$ je samosdružený a pozitivně definitní.

\item
    Nalezněte, čemu se rovnají komutátory $\commutator{\operator{n}}{\operator{a}^{k}}$ a $\commutator{\operator{n}}{\left(\conjugate{\operator{a}}\right)^{k}}$ pro $k\in{\mathbb N}$.

\item
    Ukažte, čemu se rovná $\conjugate{\operator{a}}\ket{n}$, $\operator{a}\ket{n}$, kde $\ket{n}$ je vlastní vektor operátoru $\operator{n}$ příslušející vlastní hodnotě $n$.

\item
    Nalezněte všechny vlastní hodnoty $n$ operátoru $\operator{n}$.
    
\item 
    Nalezněte normalizované vlastní vektory operátoru $\operator{n}$.
    
\item 
    Nalezněte tvar operátorů $\operator{a}$ a $\conjugate{\operator{a}}$ v maticové realizaci.
\end{enumerate}

\begin{solution}
	\begin{enumerate}	
	\item 
		Samosdruženost plyne z identit\index{operátor!samosdružený}
		\begin{equation}
			\operatorconjugate{n}
				=\conjugate{\left(\operatorconjugate{a}\operator{a}\right)}
				=\conjugate{\operator{a}}\conjugate{\left(\operatorconjugate{a}\right)}
				=\operatorconjugate{a}\operator{a}=\operator{n}.
		\end{equation}
        \trick{Díky samosdruženosti existuje spektrální rozklad} operátoru $\operator{n}$,
		\begin{equation}
            \operator{n}\ket{n}=n\ket{n},
            \label{eq:NumberOperatorSpectrum}
		\end{equation}
        kde $n$ je reálné číslo a $\ket{n}$ odpovídající normalizovaný vlastní vektor, $\braket{n}{n}=1$.
        
        Pozitivita operátoru $\operator{n}$ znamená, že všechny jeho vlastní hodnoty jsou nezáporné.\index{operátor!pozitivně definitní}
		Platí
		\begin{equation}
			\matrixelement{n}{\operator{n}}{n}
				=n\braket{n}{n}
				=n
		\end{equation}
		a zároveň
		\begin{equation}
			\matrixelement{n}{\operator{n}}{n}
				=\matrixelement{n}{\conjugate{\operator{a}}\operator{a}}{n}
				=\abss{\operator{a}\ket{n}}
				\geq0,
		\end{equation}
		takže opravdu
		\begin{equation}
			n\geq0.
		\end{equation}
	
	\item
		Pro $k=1$ je
		\begin{equation}
			\commutator{\operator{n}}{\operator{a}}
				=\commutator{\conjugate{\operator{a}}\operator{a}}{\operator{a}}
				=\conjugate{\operator{a}}\underbrace{\commutator{\operator{a}}{\operator{a}}}_{0}
					+\underbrace{\commutator{\conjugate{\operator{a}}}{\operator{a}}}_{-1}\operator{a}
				=-\operator{a},
			\label{eq:NACommutator}
		\end{equation}
		kde druhá rovnost platí díky rozvoji komutátoru~\eqref{eq:CommutatorAB,C} a poslední rovnost vyplývá ze vztahu~\eqref{eq:ShiftOperatorCommutator}.
		Induktivní opakování tohoto postupu vede na
		\begin{align}
			\commutator{\operator{n}}{\operator{a}^{k}}
				&=\commutator{\operator{n}}{\operator{a}^{k-1}\operator{a}}
				 =\operator{a}^{k-1}\underbrace{\commutator{\operator{n}}{\operator{a}}}_{-\operator{a}}
					+\commutator{\operator{n}}{\operator{a}^{k-1}}\operator{a}\nonumber\\
				&=-\operator{a}^{k}+\commutator{\operator{n}}{\operator{a}^{k-2}\operator{a}}\operator{a}
				 =-\operator{a}^{k}+\operator{a}^{k-2}\underbrace{\commutator{\operator{n}}{\operator{a}}}_{-\operator{a}}\operator{a}
					+\commutator{\operator{n}}{\operator{a}^{k-2}}\operator{a}^{2}\nonumber\\
				&=-2\operator{a}^{k}+\commutator{\operator{n}}{\operator{a}^{k-3}\operator{a}}\operator{a}^{2}=\dotsb=-k\operator{a}^{k}.
		\end{align}
		Zcela analogicky se ukáže, že
		\begin{equation}
            \commutator{\operator{n}}{\left(\conjugate{\operator{a}}\right)^{k}}
                =k\left(\conjugate{\operator{a}}\right)^{k}.
		\end{equation}
	
	\item
        Vlastní hodnoty a vlastní vektory operátoru $\operator{n}$ splňují rovnici~\eqref{eq:NumberOperatorSpectrum}, kde $n\geq0$.
        Za předpokladu $n\neq0$ a díky vztahu~\eqref{eq:NACommutator} platí
		\begin{equation}
            \operator{n}\operator{a}\ket{n}
                =\left(\operator{a}\operator{n}-\operator{a}\operator{n}+\operator{n}{\operator{a}}\right)\ket{n}
				=\left(\operator{a}\operator{n}+\commutator{\operator{n}}{\operator{a}}\right)\ket{n}
				=\left(\operator{a}\operator{n}-\operator{a}\right)\ket{n}
				=\left(n-1\right)\operator{a}\ket{n}.
		\end{equation}
        Analogicky pak
		\begin{equation}
            \operator{n}\conjugate{\operator{a}}\ket{n}
                =\left(n+1\right)\conjugate{\operator{a}}\ket{n}.
		\end{equation}
		Vektory $\operator{a}\ket{n}$ a $\conjugate{\operator{a}}\ket{n}$ jsou tedy oba vlastními stavy operátoru 
		$\operator{n}$ příslušejícími vlastním hodnotám $n-1$, respektive $n+1$.
		Norma těchto vektorů je
        \begin{subequations}
            \begin{align}
                \abs{\matrixelement{n}{\conjugate{\operator{a}}\operator{a}}{n}}
                    &=\abs{\matrixelement{n}{\operator{n}}{n}}=n\abs{\braket{n}{n}}=n,\\
                \abs{\matrixelement{n}{\operator{a}\conjugate{\operator{a}}}{n}}
                    &=\abs{\matrixelement{n}{\conjugate{\operator{a}}\operator{a}+1}{n}}=n+1.
            \end{align}                
        \end{subequations}
        Normované vlastní vektory operátoru $\operator{n}$ jsou tedy navzájem svázány operátory $\operator{a}$ a $\conjugate{\operator{a}}$:
        \begin{subequations}
            \begin{empheq}[box=\fbox]{align}
                \ket{n-1}&\equiv\frac{1}{\sqrt{n}}\,\operator{a}\ket{n},\\
                \ket{n+1}&\equiv\frac{1}{\sqrt{n+1}}\,\conjugate{\operator{a}}\ket{n}.
            \end{empheq}
            \label{eq:AN}
        \end{subequations}
        
	\item
		$k$-násobným opakovaným působení operátoru $\operator{a}$ na stav $\ket{n}$ se dospěje k normovanému vektoru
		\begin{equation}
			\ket{n-k}\equiv\frac{1}{\sqrt{n(n-1)\dotsb(n-k)}}\,\operator{a}^{k}\ket{n},
		\end{equation}
		což je vlastní vektor operátoru $\operator{n}$ příslušející vlastní hodnotě $n-k$.
        \trick{Pozitivita operátoru však omezuje hodnoty $k$: Pro žádné $k$ není povoleno získat vektor, který by měl odpovídat záporné vlastní hodnotě.}
        Jelikož $k$ lze volit libovolně, musí spektrum operátoru bezpodmínečně obsahovat hodnotu $0$, tj. musí existovat vektor $\ket{0}$ takový, že
		\begin{equation}
			\operator{n}\ket{0}=0\ket{0}=0,
        \end{equation}        
        a tedy $\operator{a}\ket{0}=0$.       
        Další aplikace operátoru $\operator{a}$ dají identicky nulu.

		Spektrum operátoru $\operator{n}$ tedy tvoří všechna nezáporná celá čísla $n\in\mathbb{N}_{0}$.
		
	\item
		Všechny normalizované vlastní vektory $\ket{n}$ lze nagenerovat ze stavu $\ket{0}$ pomocí vícenásobného použití operátoru $\conjugate{\operator{a}}$ podle~\eqref{eq:AN}:
		\begin{equation}
			\important{
				\ket{n}=\frac{\left(\conjugate{\operator{a}}\right)^{n}}{\sqrt{n!}}\ket{0}.			
			}
			\label{eq:EigenstateNumberOperator}
		\end{equation}		
        
    \item
        Dimenze Hilbertova prostoru, na kterém působí operátory $\operator{a}$ a $\conjugate{\operator{a}}$, je nekonečná, přičemž za jeho bázi lze zvolit vektory $\ket{n}$. 
        V maticové realizaci lze tento vektor vyjádřit jako nekonečný sloupec
        \begin{equation}
            \ket{n}\equiv
            \begin{blockarray}{cc}  
                \begin{block}{(c)c}
                    0 & \\ 
                    \Vdots & \\ 
                    0 & \\ 
                    1 & \dotsb\ n\text{-tá pozice}\\ 
                    0 & \\ 
                    \Vdots & \\
                \end{block}
            \end{blockarray}                        
        \end{equation}
        Z algebraického vztahu mezi bázovými vektory~\eqref{eq:AN} vyplývá
        \begin{align}
            \label{eq:ANMatrix}
            \matrix{a}&=\makematrix{
                    0 & 1 & 0 & 0 & 0 & \\
                    0 & 0 & \sqrt{2} & 0 & 0 & \\
                    0 & 0 & 0 & \sqrt{3} & 0 & \dotsb \\ 
                    0 & 0 & 0 & 0 & \sqrt{4} & \\ 
                    0 & 0 & 0 & 0 & 0 & \\
                      &   &   & \vdots & & \ddots }, &
            \conjugate{\matrix{a}}&=\makematrix{
                    0 & 0 & 0 & 0 & 0 & \\
                    1 & 0 & 0 & 0 & 0 & \\
                    0 & \sqrt{2} & 0 & 0 & 0 & \dotsb \\ 
                    0 & 0 & \sqrt{3} & 0 & 0 & \\ 
                    0 & 0 & 0 & \sqrt{4} & 0 & \\
                    &   &   & \vdots & & \ddots }\equiv\transpose{\matrix{a}}.
        \end{align}
        Operátor $\operator{n}$ je v této bázi diagonální a jeho maticové vyjádření je
        \begin{equation}
            \matrix{n}=\makematrix{
                    0 & 0 & 0 & 0 & 0 & \\
                    0 & 1 & 0 & 0 & 0 & \\
                    0 & 0 & 2 & 0 & 0 & \dotsb \\
                    0 & 0 & 0 & 3 & 0 & \\
                    0 & 0 & 0 & 0 & 4 & \\
                    &   &   & \vdots & & \ddots }.
        \end{equation}
        K tomuto vztahu lze rovněž dospět pronásobením nekonečných matic~\eqref{eq:ANMatrix}, tj. $\matrix{n}=\conjugate{\matrix{a}}\matrix{a}$.
    \end{enumerate}
\end{solution}

\begin{note}
    Operátory $\operator{a}$, $\conjugate{\operator{a}}$ posouvají vlastní stav operátoru $\operator{n}$
    z vyšší vlastní hodnoty na nižší a naopak, proto se obvykle nazývají \emph{posunovací operátory}\index{operátor!posunovací}.
    Při využití těchto operátorů nad Fokovými prostory (kapitola~\ref{sec:ManyBody}), kde vytvářejí a ruší částice daných vlastností, se nazývají \emph{anihilační a kreační operátory}.
\end{note}
