\subsection{Vlastnosti grupy $\group{O}(n)$}
	Grupa $\group{O}(n)$ je grupou ortogonálních transformací, které udávají všechna natočení v $n$-rozměrném prostoru,
	\begin{equation}
		x'_{j}=\sum_{k=1}^{n}R_{jk}x_{k},
	\end{equation}
	kde $R_{jk}\in\mathbb{R}$ jsou elementy matice rotace $\matrix{R}$, $x_{j}\in\mathbb{R}$ jsou elementy vektoru $\vector{x}$.

	\begin{enumerate}
	\item 
		Dokažte, že při natočení vektoru se jeho délka nemění:
		\begin{equation}
			\abs{\vector{x}'}^{2}=\sum_{j=1}^{n}x_{j}^{'2}=\sum_{j=1}^{n}x_{j}^{2}=\abs{\vector{x}}^{2}.
		\end{equation}
		Na základě této vlastnosti ukažte, že matice $\matrix{R}$ musejí být ortogonální, tj. $\matrix{R}^{\ti{T}}=\matrix{R}^{-1}$
		a že $\det\matrix{R}=\pm1$. 

	\item 
		Ukažte, že grupa $\group{O}(n)$ je uzavřená, tj. že matice $\matrix{R}_{3}=\matrix{R}_{1}\matrix{R}_{2}$
		(součin dvou ortogonálních matic) je ortogonální.

	\item 
		Nalezněte počet nezávislých elementů ortogonální matice rozměru $n\times n$.
	\end{enumerate}

	Nadále uvažujte jen \emph{vlastní rotace}, které splňují $\det\matrix{R}=1$. 
	Ty tvoří grupu $\group{SO}(n)$. 
	Matice s $\det\matrix{R}=-1$ zahrnují totiž kromě otočení ještě prostorovou inverzi.

	\begin{enumerate}
	\setcounter{enumi}{3}
	\item 
		Ukažte, že generátory rotací v maticové realizaci $\matrix{L}^{(jk)}$ lze zapsat jako hermitovské matice
		\begin{equation}
			\matrix{L}^{(jk)}_{\alpha\beta}=\im\left(\delta_{j\beta}\delta_{k\alpha}-\delta_{j\alpha}\delta_{k\beta}\right)
		\end{equation}
		($\alpha,\beta=1,\dotsc,n$ určují složky matic $\matrix{L}^{(jk)}$),
		tj. ukažte, že matice 
		\begin{align}
			\matrix{R}=\exp\left(\im \sum_{j<k} a_{jk}\matrix{L}^{(jk)}\right)
		\end{align}
		jsou ortogonální a $a_{jk}$ jsou reálné parametry. 
		
		Přesvědčte se, že počet takto vytvořených generátorů je stejný jako počet nezávislých elementů ortogonálních matic rozměru $n$.
		To znamená, že každou ortogonální matici lze jednoznačně zadat pomocí parametrů $a_{jk}$.

		\item 
			Nalezněte strukturní koeficienty grupy $\group{SO}(n)$, tj. spočítejte komutátor
			\begin{equation}
				\commutator{\matrix{L}^{(jk)}}{\matrix{L}^{(lm)}}.
			\end{equation}
			Tento komutátor musí být stejný pro jakoukoliv (nejen maticovou) realizaci grupy.
			Nadále budeme tedy uvažovat, že generátory $\operator{L}_{jk}$ jsou operátory nad obecným Hilbertovým prostorem $\hilbert{H}$, 
			které splňují tyto komutační relace.

		\item 
			Ukažte, že lineární Casimirův operátor
			\begin{equation}
				\operator{C}_{1}[\group{SO}(n)]\equiv\sum_{j=1}^{n}\operator{L}_{jj}
			\end{equation}
			je identicky nulový.
			Nenulový je až kvadratický Casimirův operátor 
			\begin{equation}
				\operator{C}_{2}[\group{SO}(n)]\equiv\sum_{j,k=1}^{n}\operator{L}_{jk}\operator{L}_{kj}\,.
			\end{equation}			
			Přesvědčte se, že komutuje se všemi generátory $\operator{L}_{jk}$.

		\item 
			Ukažte, že operátory impulsmomentu v $n$-rozměrném prostoru, definované vztahem
			\begin{equation}
				\operator{L}_{jk}=\operator{x}_{j}\operator{p}_{k}-\operator{x}_{k}\operator{p}_{j}, \quad j,k=1,\dots,n,
			\end{equation}
			(mezi operátorem souřadnice a hybnosti platí standardní komutační relace $\commutator{\operator{x}_{j}}{\operator{p}_{k}}=\im\hbar\delta_{jk}$,	$\commutator{\operator{x}_{j}}{\operator{x}_{k}}=\commutator{\operator{p}_{j}}{\operator{p}_{k}}=0$) splňují (až na konstantu $\hbar$) stejné komutační relace jako výše vypočtené matice $\matrix{L}^{(jk)}$, a tvoří tedy generátory grupy $\group{SO}(n)$.
			Napište vyjádření operátorů $\operator{L}_{jk}$ v $x$-reprezentaci, tj. dosaďte $\operator{p}_{j}=-\im\hbar\frac{\partial}{\partial x_{j}}$.

		\item 
			V třírozměrném prostoru dokažte, že platí
			\begin{equation}
				\matrix{R}^{(12)}=\e^{\left(\im\phi \matrix{L}^{(12)}\right)}=\left(\begin{array}{ccc}\cos\phi & \sin\phi & 0 \\ -\sin\phi & \cos\phi & 0 \\ 0 & 0 & 1\end{array}\right)
			\end{equation}
			(matice $\matrix{L}^{(12)}$ je definována výše). 
			Jedná se tedy o matici rotace okolo osy $3$ o úhel $\phi$.
			K výpočtu můžete použít rozvoj exponenciály do Taylorovy řady.
	\end{enumerate}

	\emph{Poznámka:} 
		V třírozměrném prostoru platí mezi složkami vektoru impulsmomentu $\operator{L}_{j}$ a mezi výše definovanými operátory $\operator{L}_{kl}$ vztah
		\begin{equation}
		\operator{L}_{j}=\frac{1}{2}\epsilon_{jkl}\operator{L}_{kl}.
		\end{equation}
		Složky $\operator{L}_{j}$ splňují komutační relace 
		\begin{equation}
			\commutator{\operator{L}_{j}}{\operator{L}_{k}}=\im\hbar\epsilon_{jkl}\operator{L}_{l}
		\end{equation}
		a jsou jen jiným vyjádřením generujících operátorů grupy $\group{SO}(3)$.
