\sec{Komutátor}
Komutátor\index{komutátor} dvou operátorů $\operator{A}$ a $\operator{B}$ se zavádí jako bilineární zobrazení
\begin{equation}
    \commutator{\operator{A}}{\operator{B}}
        \equiv\operator{A}\operator{B}-\operator{B}\operator{A}
    \label{eq:Commutator}
\end{equation}
a splňuje následující dvě vlastnosti:
\begin{subequations}  
\begin{enumerate}
\item\emph{Antisymetrie}
    \begin{equation}
        \commutator{\operator{A}}{\operator{B}}
            =-\commutator{\operator{B}}{\operator{A}},
    \end{equation}
\item\emph{Jacobiho identita}\index{identita!Jacobiho}
    \begin{equation}
        \commutator{\operator{A}}{\commutator{\operator{B}}{\operator{C}}}
            +\commutator{\operator{B}}{\commutator{\operator{C}}{\operator{A}}}
            +\commutator{\operator{C}}{\commutator{\operator{A}}{\operator{B}}}
            =0.
    \end{equation}
\end{enumerate}
\end{subequations}

\begin{note}
    Skládání lineárních operátorů nad Hilbertovými prostory je z definice asociativní, tj. platí, že
    \begin{equation}
        \left(\operator{A}\operator{B}\right)\operator{C}
            =\operator{A}\left(\operator{B}\operator{C}\right).
    \end{equation}
    Existují však \emph{neasociativní algebry}\index{algebra!neasociativní}, kde míru neasociativity udává \emph{asociátor}\index{asociátor}
    \begin{equation}
        \associator{x}{y}{z}
            \equiv(xy)z-x(yz).
    \end{equation}
    Příkladem je algebra oktonionů\index{oktoniony} nebo algebry používající se v teoriích superstrun\index{teorie!superstrun} (například pro popis částice pohybující se v okolí magnetického monopólu).
\end{note}
