\sec{Funkce operátoru}
Funkci operátoru $\operator{A}$\index{funkce!operátoru} na základě funkce reálného argumentu $f(\xi)$ lze zavést dvěma způsoby:
\begin{enumerate}
\item
    \emph{Rozvojem do řady.}
    Za předpokladu, že existuje \emph{Maclaurinova řada}\index{řada!Maclaurinova} konvergující k zadané funkci, tj. rozvoj funkce $f(\xi)$ do mocninné řady v okolí bodu $\xi=0$,
    \begin{equation}
        f(\xi)=\sum_{n=0}^{\infty}a_{n}\xi^{n},\quad 
        a_{n}=\frac{1}{n!}\left.\frac{\d^{n}f(\xi)}{\d\xi^{n}}\right|_{\xi=0},
    \end{equation}
    pak se pod funkcí $f(\operator{A})$ rozumí operátor
    \begin{equation}
        f(\operator{A})
            =\sum_{n=0}^{\infty}a_{n}\operator{A}^{n}.
        \label{eq:OperatorFncArray}
    \end{equation}	
    
    \begin{example}\index{exponenciála operátoru}
        \begin{equation}
            \important{
                \e^{\operator{A}}
                    \equiv\sum_{n=0}^{\infty}\frac{\operator{A}^{n}}{n!}
            }
            \label{eq:OperatorExp}
        \end{equation}
    \end{example}
        
\item
    \emph{Sylvestrova formule.}\index{formule!Sylvestrova}
    Je-li operátor $\operator{A}$ samosdružený se spektrálním rozkladem
    \begin{equation}
        \operator{A}
            =\sum_{k}A_{k}\operator{P}_{k},
    \end{equation}
    kde $\operator{P}_{k}=\projector{A_{k}}$ je projektor na podprostor příslušející reálné nedegenerované vlastní hodnotě $A_{k}$, a je-li funkce $f(\xi)$ definovaná na množině, která zahrnuje všechny body $\xi=A_{k}$, funkci operátoru $\operator{A}$ lze vyjádřit jako součet
    \begin{equation}
        f(\operator{A})
            =\sum_{k}f(A_{k})\operator{P}_{k}.
        \label{eq:OperatorFncSylvester}
    \end{equation}		
\end{enumerate}
