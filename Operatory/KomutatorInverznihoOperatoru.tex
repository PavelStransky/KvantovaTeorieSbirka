\subsection{Komutátor inverzního operátoru}
Vyjádřete komutátor $\commutator{\operator{A}}{\operator{B}^{-1}}$ pomocí komutátoru $\commutator{\operator{A}}{\operator{B}}$.

\begin{solution}
	\trick{Každý operátor komutuje s operátorem identity}, tj.
	\begin{equation}
        \commutator{\operator{A}}{\operator{B}\operator{B}^{-1}}
			=\commutator{\operator{A}}{\operator{1}}=0.
	\end{equation}
	Levá strana rovnice se rozepíše podle vztahu~\eqref{eq:CommutatorA,BC} z předchozího příkladu,
	\begin{equation}
		\commutator{\operator{A}}{\operator{B}\operator{B}^{-1}}
			=\commutator{\operator{A}}{\operator{B}}\operator{B}^{-1}+\operator{B}\commutator{\operator{A}}{\operator{B}^{-1}},
	\end{equation}
	což vede k výsledku
	\begin{equation}
		\commutator{\operator{A}}{\operator{B}^{-1}}
			=-\operator{B}^{-1}\commutator{\operator{A}}{\operator{B}}\operator{B}^{-1}\,.
	\end{equation}	
	Dalším přirozeným krokem je ukázat, že
	\begin{equation}
		\commutator{\operator{A}^{-1}}{\operator{B}^{-1}}
			=\operator{A}^{-1}\operator{B}^{-1}\commutator{\operator{A}}{\operator{B}}\operator{B}^{-1}\operator{A}^{-1}
	\end{equation}
	(o platnosti tohoto vztahu se lze též přesvědčit prostým roznásobením).
\end{solution}
