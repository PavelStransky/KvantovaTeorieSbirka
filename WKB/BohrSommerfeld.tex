\sec{Bohrovo-Sommerfeldovo kvantování}\index{kvantování!Bohrovo-Sommerfeldovo}
Omezme se nyní na vázané stavy, tj. na stavy, jejichž vlnová funkce $\psi(x\pm\infty)=0$.
Vyskytují-li se na zadané energii jen dva body obratu, pak musí platit $\psi_{\mathrm{I}_{1}}(x)=\psi_{\mathrm{I}_{2}}(x)$, což je splněno, pokud
\begin{itemize}
\item
    $C_{1}=C_{2}$ a zároveň
    \begin{equation}
        \sin\left[\frac{1}{\hbar}\int_{x_{1}}^{x}p(x')\d x'+\frac{\pi}{4}\right]=\sin\left[\frac{1}{\hbar}\int_{x}^{x_{2}}p(x')\d x'+\frac{\pi}{4}\right],
    \end{equation}
    neboli
    \begin{equation}
        \frac{1}{\hbar}\int_{x_{1}}^{x}p(x')\d x'+\frac{\pi}{4}+n\pi=\frac{1}{\hbar}\int_{x}^{x_{2}}p(x')\d x'+\frac{\pi}{4},
    \end{equation}
    nebo vhodněji pokud	
\item
    $C_{1}=-C_{2}$ a zároveň
    \begin{equation}
        \sin\left[\frac{1}{\hbar}\int_{x_{1}}^{x}p(x')\d x'+\frac{\pi}{4}\right]=-\sin\left[\frac{1}{\hbar}\int_{x}^{x_{2}}p(x')\d x'+\frac{\pi}{4}\right],
    \end{equation}
    neboli
    \begin{equation}
        \frac{1}{\hbar}\int_{x_{1}}^{x}p(x')\d x'+\frac{\pi}{4}+n\pi=-\frac{1}{\hbar}\int_{x}^{x_{2}}p(x')\d x'-\frac{\pi}{4}.
    \end{equation}
\end{itemize}
Z poslední rovnosti plyne \emph{Bohrova-Sommerfeldova kvantovací podmínka}
\begin{equation}
    \label{eq:BohrSommerfeld}
    \important{\oint p(x')\d x'=2\int_{x_{1}}^{x_{2}}p(x')\d x'=2\pi\hbar\left(n+\frac{1}{2}\right)},
\end{equation}
kde $n\in\mathbb{N}_{0}$.	
Tento výraz platí jen pro \uv{měkké} body obratu.
Za každý \uv{tvrdý} bod obratu (odraz o nekonečně vysokou bariéru) je nutné do závorky přidat ještě $\frac{1}{4}$.
