WKB\footnote{
    Metodu rozpracovali fyzici Gregor Wentzel, Hendrik Kramers, Léon Brillouin (1926) a paralelně matematik Harold Jeffreys (1923), někdy se proto též nazývá JWKB nebo WKBJ metoda.
    Poprvé ji však popsali o století dříve Joseph Liouville a George Green (1837) a v některých zdrojích se tudíž uvádí jako LG metoda.
    Přínos autorů WKBJ byl v zavedení navazovacích podmínek v bodech obratu.
}\index{aproximace!WKB}
je kvaziklasická aproximace jednorozměrné Schrödingerovy rovnice
\begin{equation}
    -\frac{\hbar^{2}}{2M}\psi''(x)+V(x)\psi(x)=E\psi(x).
\end{equation}
Rovnice se přepíše do tvaru
\begin{equation}
    \label{eq:WKBSchrodingerp}
    \psi''(x)=-\frac{p^{2}}{\hbar^{2}}\psi(x),
\end{equation}
kde
\begin{equation}
    p=\pm\sqrt{2M\left[E-V(x)\right]}
\end{equation}
je klasická hybnost částice.
Vlnová funkce se následně hledá ve tvaru
\begin{equation}
    \psi(x)=A(x)\e^{\frac{\im}{\hbar}S(x)},
\end{equation}
kde obě funkce $A(x), S(x)$ jsou reálné; udávají tedy amplitudu a fázi.
Dosazení do Schrödingerovy rovnice~\eqref{eq:WKBSchrodingerp} vede na dvě rovnice (jednu pro reálnou a druhou pro imaginární část)
\begin{subequations}
    \begin{align}
        \hbar^{2}\frac{A''(x)}{A(x)}-\left[S'(x)\right]^{2}&=-p^{2},\\
        2A'(x)S'(x)+A(x)S''(x)&=0.
    \end{align}        
\end{subequations}
Řešení druhé rovnice je 
\begin{align}
    \left[A^{2}(x)S'(x)\right]'&=0 && \Longrightarrow & A(x)&=\frac{C}{S'(x)}.
\end{align}
V první rovnici se zanedbá člen $\hbar^{2}$ a navíc se předpokládá, že \uv{oscilace} amplitudy $A(x)$ jsou mnohem menší než amplituda sama.
Tím se rovnice zjednoduší a funkci $S(x)$ lze nalézt integrováním
\begin{align}
    \left[S'(x)\right]^{2}&=p^{2} && \Longrightarrow & S'(x)&=\pm\abs{p(x)} && \Longrightarrow & S(x)=\pm\int_{x_{0}}^{x}\abs{p(x')}\d x',
\end{align}
což je výraz pro klasickou akci.\index{akce!klasická}

WKB vlnová funkce má tedy tvar
\begin{equation}
    \psi(x)=\frac{1}{\sqrt{\abs{p(x)}}}\left[C\e^{\frac{\im}{\hbar}\int_{x_{0}}^{x}\abs{p(x')}\d x'}+D\e^{-\frac{\im}{\hbar}\int_{x_{0}}^{x}\abs{p(x')}\d x'}\right],
\end{equation}
kde $x_{0}$ je libovolný bod, od kterého se integruje, a $C,D$ jsou normalizační konstanty (změna integrační meze $x_0$ pouze změní celkovou fázi vlnové funkce).

V kinematicky nedostupné oblasti, ve které je $E<V(x)$, vlnová funkce přejde od oscilujícího řešení k~exponenciálně ubývajícím řešením
\begin{equation}
    \psi(x)=\frac{1}{\sqrt{\abs{p(x)}}}\left[E\e^{\frac{1}{\hbar}\int_{x_{0}}^{x}\abs{p(x')}\d x'}+F\e^{-\frac{1}{\hbar}\int_{x_{0}}^{x}\abs{p(x')}\d x'}\right].
\end{equation}