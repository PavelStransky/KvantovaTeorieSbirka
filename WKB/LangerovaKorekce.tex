\subsection{Coulombické pole}	
    WKB metodu lze aplikovat také na problémy se sféricky symetrickým polem.
    Schrödingerova rovnice pro radiální část vlnové funkce $R(r)$ obecného sféricky symetrického problému má tvar
    \begin{equation}
        \frac{1}{r^{2}}\frac{\d}{\d r}r^{2}\frac{\d}{\d r}R(r)+\frac{2m}{\hbar^{2}}\left(E-V_{\ti{ef}}(r)\right)R(r)
            =0,
    \end{equation}
    kde 
    \begin{equation}
    V_{\ti{ef}}(r)
        \equiv V(r)+\frac{\hbar^{2}l(l+1)}{2mr^{2}}
    \end{equation}
    je efektivní potenciál, zahrnující v sobě centrifugální člen.
    
    Zavedením substituce $R(r)=u(r)/r$ dostaneme rovnici
    \begin{equation}
        \frac{\d^{2}}{\d r^{2}}u(r)+k^{2}(r)u(r)
            =0,
    \end{equation}
    kde $k^{2}(r)=2m/\hbar^{2}(E-V_{\ti{ef}})$.
    
    WKB metoda pro sféricky symetrické potenciály dává dobré výsledky jedině v případě, 
    aplikujeme-li tzv. \emph{Langerovu korekci}\index{korekce!Langerova}, která spočívá v nahrazení 
    \begin{equation}
        \important{l(l+1)\rightarrow(l+\frac{1}{2})^{2}}
    \end{equation}
    (dá se odvodit z asymptotiky vlnových funkcí; původní práce Rudolpha E. Langera je v~\cite{Langer1937}).
    Vázané stavy lze pak nalézt z rovnice ekvivalentní Bohrově-Sommerfeldově kvantovací podmínce
    \begin{equation}
        \int_{r_{1}}^{r_{2}}k'(r)\d r
            =\left(n_{\ti{r}}+\frac{1}{2}\right)\pi,
    \end{equation}
    přičemž $k'(r)$ zahrnuje Langerovu korekci, $n_{\ti{r}}=0,1,\dots$ je radiální kvantové číslo a
    $r_{1,2}$ jsou body obratu klasické trajektorie s hybností $p'(r)=\hbar k'(r)$.
    
    Uvažujte konkrétní případ pohybu částice v Coulombickém poli
    \begin{equation}
        V(r)=-\frac{\gamma}{r}.
    \end{equation}
    kde $\gamma=e^{2}/(4\pi\epsilon_{0})$.
    
    \begin{enumerate}
        \item 
            Nalezněte body obratu $r_{1,2}$ trajektorie s energií $E$ (počítejte s Langerovou korekcí).
        
        \item 
            Pomocí WKB přiblížení nalezněte spektrum, tj. stavy s energií $E<0$.
        
        \item
            Porovnejte toto spektrum se spektrem získaným přesným řešením Schrödingerovy rovnice.
    \end{enumerate}

    \begin{solution}
        Řešení formou hry naleznete na následujících odkazech:
        \href{http://pavelstransky.cz/cviceniktvt/WKBLanger_1.1_linux.zip}{Linux},
        \href{http://pavelstransky.cz/cviceniktvt/WKBLanger_1.1_windows.zip}{Windows}.
    \end{solution}
