\subsection{Homogenní pole}
Částice o hmotnosti $M$ (hopík) skáče v homogenním (např. gravitačním) poli,
přičemž od podložky se odráží bez ztráty energie.
Potenciál se tedy dá vyjádřit jako
\begin{equation}
    V(z)=
    \begin{cases}
    mgz & z > 0 \\
    \infty & z < 0
    \end{cases}
\end{equation}

\begin{enumerate}
\item 
        Řešení pomocí WKB metody:
        \begin{itemize}
        \item 
            Nalezněte body obratu, má-li částice energii $E$.

        \item 
            Pomocí WKB přiblížení vypočítejte energetické spektrum.

        \item
            Nalezněte WKB vlnové funkce v klasicky dostupné i nedostupné oblasti.
            Vlnové funkce nemusíte normovat.
        \end{itemize}

    \item 
        Hledání základního stavu variační metodou:
        \begin{itemize}
        \item 
            Podle chování potenciálu navrhněte vhodnou testovací funkci s jedním parametrem (dodatečný parametr bude fixovat normalizaci).

        \item 
            Nalezněte optimální hodnotu parametru a jemu odpovídající přibližnou energii základního stavu.
        \end{itemize}

    \item 
        Srovnáním energií základního stavu získaných oběma metodami určete, která metoda dává základní stav přesněji.
\end{enumerate}	
