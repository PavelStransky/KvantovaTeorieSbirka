\subsection{Poissonovo rozdělení}
\label{sec:PoissonDistribution}
Poissonovo rozdělení\index{rozdělení!Poissonovo} udává počet $n$ výskytu jevů v určitém intervalu, pokud jsou jednotlivé jevy statisticky nezávislé.
Rozdělení je dáno předpisem
\begin{equation}
    \label{eq:PoissonDistribution}
    P_{n}=\frac{\lambda^{n}}{n!}\e^{-\lambda}\,,
\end{equation}			
kde $\lambda$ je parametr Poissonova rozdělení.

\begin{enumerate}
\item
    Ukažte, že rozdělení je normalizované.
    
\item
    Nalezněte střední hodnotu Poissonova rozdělení.
    
\item
    Nalezněte rozptyl Poissonova rozdělení.
\end{enumerate}

\begin{solution}
    \begin{enumerate}
	\item
		Normalizace
		\begin{equation}
			\sum_{n=0}^{\infty}P_{n}
				=\sum_{n=0}^{\infty}\frac{\lambda^{n}}{n!}
				=\e^{-\lambda}\e^{\lambda}=1\,.
		\end{equation}
		
	\item
		Střední hodnota
		\begin{equation}
			\mean{n}
				=\sum_{n=0}^{\infty}nP_{n}
				=\e^{-\lambda}\sum_{n=1}^{\infty}\frac{\lambda^{n}}{(n-1)!}
				=\e^{-\lambda}\lambda\e^{\lambda}=\lambda\,.
		\end{equation}
		
	\item
		Střední hodnota kvadrátu je
		\begin{align}
			\mean{n^{2}}
				&=\sum_{n=0}^{\infty}n^{2}P_{n}
				=\e^{-\lambda}\sum_{n=1}^{\infty}\frac{\lambda^{n}}{(n-1)!}
					\underbrace{n}_{n-1+1}\nonumber\\
				&=\e^{-\lambda}\left[\lambda^{2}\sum_{n=2}^{\infty}\frac{\lambda^{n-2}}{(n-2)!}
					+\lambda\sum_{n=1}^{\infty}\frac{\lambda^{n-1}}{(n-1)!}\right]\nonumber\\
				&=\lambda^{2}+\lambda\,,
		\end{align}
		takže rozptyl vychází
		\begin{equation}
			(\Delta n)^{2}=\mean{n^{2}}-\mean{n}^{2}=\lambda\,.
		\end{equation}
		Poissonovo rozdělení má tedy rozpyl i střední hodnotu stejnou a danou velikostí parametru $\lambda$.
	\end{enumerate}
\end{solution}
