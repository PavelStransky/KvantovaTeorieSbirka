\subsection{Střední hodnoty operátorů}
\label{sec:CoherentStatesExpectation}
\begin{enumerate}
\item 
    Nalezněte střední hodnotu energie harmonického oscilátoru ve stavu $\ket{z}$
    a porovnejte s výsledkem~\eqref{eq:HOmeanE}.
    
\item 
    Určete střední hodnoty operátorů polohy a hybnosti 
	\begin{subequations}
		\begin{align}
			\mean{\operator{x}}_{z}&\equiv\matrixelement{z}{\operator{x}}{z},\\
			\mean{\operator{p}}_{z}&\equiv\matrixelement{z}{\operator{p}}{z}
		\end{align}					
	\end{subequations}
    a vyjádřete pomocí nich číslo $z$.

\item 
    Určete střední kvadratickou odchylku operátorů souřadnice a hybnosti 
	\begin{subequations}
		\begin{align}
			\uncertainty{z}{x}&\equiv\matrixelement{z}{\left(\operator{x}-\mean{\operator{x}}_{z}\right)^{2}}{z},\\
			\uncertainty{z}{p}&\equiv\matrixelement{z}{\left(\operator{p}-\mean{\operator{p}}_{z}\right)^{2}}{z}
		\end{align}			
	\end{subequations}
    a pomocí nich ukažte, že koherentní stavy minimalizují relace neurčitosti.
\end{enumerate}

\begin{solution}
	\begin{enumerate}
	\item 
		Hamiltonián harmonického oscilátoru se vyjádří pomocí posunovacích operátorů~\eqref{eq:HOHamiltonianAA+} a využije se vztahy~\eqref{eq:HOCoherentStateA}:
		\begin{equation}
			\matrixelement{z}{\operator{H}}{z}
				=\matrixelement{z}{\hbar\Omega\left(\conjugate{\operator{a}}\operator{a}+\frac{1}{2}\right)}{z}
				=\hbar\Omega\left(\abs{z}^{2}+\frac{1}{2}\right).
		\end{equation}
		To se rovná dříve obdrženému výsledku~\eqref{eq:HOmeanE}.
		
	\item
		Přímočarý postup rozvinutím koherentních stavů do řady a využitím vztahu mezi souřadnicí a posunovacími operátory~\eqref{eq:ShiftOperatorToPX}\sfootnote{
			Místo explicitního výpočtu lze využít i \trick{jednodušší postup díky vztahu~\eqref{eq:HOCoherentStateA}}:
			\begin{equation}
				\label{eq:HOCoherentStateAZ}
				\matrixelement{z}{\operator{a}}{z}=z\,,
				\qquad\matrixelement{z}{\conjugate{\operator{a}}}{z}=\left(\matrixelement{z}{\operator{a}}{z}\right)^{*}=z^{*}\,.
			\end{equation}
		} se dostane
		\begin{align}
			\matrixelement{z}{\operator{x}}{z}
				&=\e^{-\abs{z}^{2}}\sum_{n,n'=0}^{\infty}\frac{z^{*n}}{\sqrt{n!}}
					\frac{z^{n'}}{\sqrt{n'!}}
					\underbrace{\matrixelement{n'}{\operator{x}}{n}}_{\sqrt{\frac{\hbar}{2M\Omega}}
					\left(\sqrt{n+1}\delta_{n',n+1}+\sqrt{n}\delta_{n',n-1}\right)}\nonumber\\
				&=\e^{-\abs{z}^{2}}\sqrt{\frac{\hbar}{2M\Omega}}
					\left[\sum_{n=0}^{\infty}\frac{z^{*n}z^{n+1}}{\sqrt{n!(n+1)!}}\sqrt{n+1}
					+\sum_{n=0}^{\infty}\frac{z^{*n}z^{n-1}}{\sqrt{n!(n-1)!}}\sqrt{n}\right]\nonumber\\
				&=\e^{-\abs{z}^{2}}\sqrt{\frac{\hbar}{2M\Omega}}
					\left[z\e^{\abs{z}^{2}}+z^{*}\e^{\abs{z}^{2}}\right]\nonumber\\
				&=\sqrt{\frac{2\hbar}{M\Omega}}\real{z}
		\end{align}
		a analogicky pro operátor hybnosti
		\begin{equation}
			\matrixelement{z}{\operator{p}}{z}=\sqrt{2\hbar M\Omega}\,\imaginary{z}.
		\end{equation}		
		Číslo $z$ lze tedy vyjádřit jako
		\begin{equation}
			\label{eq:HOCoherentStateZ}
			\boxed{
			z=\sqrt{\frac{M\Omega}{2\hbar}}
				\left(\mean{\operator{x}}_{z}+\frac{\im}{M\Omega}\mean{\operator{p}}_{z}\right)
			}.
		\end{equation}
		Reálná část čísla $z$ udává střední hodnotu \emph{souřadnice}, 
		imaginární část $z$ střední hodnotu \emph{hybnosti}.				
				
	\item
		Kvadrát operátoru souřadnice vyjádřený pomocí posunovacích operátorů~\eqref{eq:ShiftOperatorToPX} je
		\begin{align}
			\operator{x}^{2}
				&=\frac{\hbar}{2M\Omega}\left(\operator{a}^{2}+\operator{a}\conjugate{\operator{a}}+\conjugate{\operator{a}}\operator{a}+\operator{a}^{\dagger 2}\right)
					\nonumber\\
				&=\frac{\hbar}{2M\Omega}\left(\operator{a}^{2}+2\conjugate{\operator{a}}\operator{a}+1+\operator{a}^{\dagger 2}\right)
		\end{align}
		(využily se navíc komutační relace $\commutator{\operator{a}}{\conjugate{\operator{a}}}=1$).
		Na základě vztahů analogických k~\eqref{eq:HOCoherentStateAZ} je střední hodnota operátoru $\operator{x}^{2}$ rovna
		\begin{equation}
			\matrixelement{z}{\operator{x}^{2}}{z}=\frac{\hbar}{2M\Omega}\left(z^{2}+2\abs{z}^{2}+1+z^{*2}\right).
		\end{equation}
		Střední kvadratická odchylka tedy vychází
		\begin{equation}
			\uncertainty{z}{x}
				=\frac{\hbar}{2M\Omega}
					\left(z^{2}+2\abs{z}^{2}+1+z^{*2}-z^{2}-2\abs{z}^{2}-z^{*2}\right)
				=\frac{\hbar}{2M\Omega}
		\end{equation}
		a podobně pro operátor hybnosti
		\begin{equation}
			\uncertainty{z}{p}=\frac{\hbar M\Omega}{2}.
		\end{equation}
		Relace neurčitosti pro koherentní stav tedy znějí
		\begin{equation}
			\label{eq:HOCoherentStateUncertainty}
			\uncertainty{z}{x}\uncertainty{z}{p}=\frac{\hbar^{2}}{4},
		\end{equation}
		což je minimální hodnota, kterou může neurčitost mezi polohou a hybností nabývat.
	\end{enumerate}
\end{solution}
