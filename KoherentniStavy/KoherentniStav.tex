V této části se navazuje zejména na kapitolu~\ref{sec:HO} o algebraickém řešení harmonického oscilátoru.
Cílem je prozkoumat vlastnosti \emph{koherentního stavu}\index{stav!koherentní} jednorozměrného harmonického oscilátoru
\begin{equation}
	\label{eq:HOCoherentState}
	\important{\ket{z}=\e^{-\frac{|z|^{2}}{2}}\sum_{n=0}^{\infty}\frac{z^{n}}{\sqrt{n!}}\ket{n}}\,,
\end{equation}
kde $z\in\mathbb{C}$ je libovolné komplexní číslo a $\ket{n}$ je vlastní stav harmonického oscilátoru~\eqref{eq:NumberOperator} příslušející energii $E_{n}$, dané vztahem~\eqref{eq:HOEnergy}.

Níže uvedený postup je deduktivní a vychází z definice koherentního stavu~\eqref{eq:HOCoherentState}.
Dá se však postupovat i opačně: (i) buď hledat stavy harmonického oscilátoru, které minimalizují relace neurčitosti~\eqref{eq:HOCoherentStateUncertainty}, jak je popsáno v práci~\cite{Formanek2004} (strana 220), nebo hledat takové stavy, které co nejlépe aproximují pohyb klasické částice, tj. splňují vztah~\eqref{eq:HOCoherentStateClassicalMotion} pro střední hodnoty $\operator{x}$ a $\operator{p}$.
Tento postup je sledován například v učebnici~\cite{Manoukian2006}.