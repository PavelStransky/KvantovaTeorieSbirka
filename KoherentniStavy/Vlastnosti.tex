\subsection{Základní vlastnosti koherentních stavů}
\label{sec:HOCoherentState}
\begin{enumerate}
\item 
    Ukažte, že skalární součin dvou koherentních stavů~\eqref{eq:HOCoherentState} je
    \begin{equation}			
        \label{eq:HOCoherentStateBraket}
        \braket{z}{z'}=\e^{-\frac{\abs{z-z'}^{2}}{2}
            +\im\abs{z}\abs{z'}\sin\left(\phi'-\phi\right)}\,,
    \end{equation}
    kde
    \begin{equation}
        \label{eq:z}
        z=\abs{z}\e^{\im\phi}\,,\qquad\phi\in[0,2\pi)\,.
    \end{equation}
    Koherentní stavy jsou tedy normalizované, avšak nejsou navzájem ortogonální, z čehož vyplývá, že $\ket{z}$ nelze vzít za bázi Hilbertova prostoru harmonického oscilátoru (resp. někdy se říká, že báze pomocí koherentních stavů je přeurčená).

\item
    Ukažte, čemu se rovná pravděpodobnost nalezení stavu $\ket{z'}$, 
    pokud máme systém připravený ve stavu $\ket{z}$.

\item
    Ukažte, že je splněna relace uzavřenosti ve tvaru
    \begin{equation}
        \int\ket{z}\bra{z}\frac{\d z}{\pi}=1\,.
    \end{equation}

\item
    Ukažte, že rozdělení energií v koherentním stavu je Poissonovo, tj. že lze psát
    \begin{equation}
        P_{n}=\abs{\braket{n}{z}}^{2}=\frac{\lambda^{n}}{n!}\e^{-\lambda}\,.
    \end{equation}
    Nalezněte $\lambda$.

\item 
    Na základě vlastností Poissonova rozdělení nalezněte, 
    čemu se rovná střední hodnota energie harmonického oscilátoru
    v koherentním stavu $\mean{E}_{z}$.
\end{enumerate}
	
\begin{solution}
	\begin{enumerate}
	\item
		Na základě definice~\eqref{eq:HOCoherentState} platí:
		\begin{align}
			\braket{z}{z'}
				&=\e^{-\frac{\abs{z}^{2}+\abs{z'}^{2}}{2}}
					\sum_{n,n'=0}^{\infty}\frac{(z^{*})^{n}(z')^{n'}}{\sqrt{n!n'!}}
					\underbrace{\braket{n}{n'}}_{\delta_{nn'}}\nonumber\\
				&=\e^{-\frac{\abs{z}^{2}+\abs{z'}^{2}}{2}}
					\sum_{n=0}^{\infty}\frac{(z^{*}z')^{n}}{n!}\,.
		\end{align}
		Po zavedení
		\begin{equation}
			z^{*}=\abs{z}\e^{-\im\phi}\,,\qquad z'=\abs{z'}\e^{\im\phi'}			
		\end{equation}
		lze absolutní hodnotu rozdílu dvou komplexních čísel rozepsat jako
		\begin{align}
			\abs{z-z'}^{2}
				&=\left(z-z'\right)\left(z^{*}-z'^{*}\right)\nonumber\\
				&=\abs{z}^{2}+\abs{z'}^{2}-zz'^{*}-z^{*}z'\nonumber\\
				&=\abs{z}^{2}+\abs{z'}^{2}-\abs{z}\abs{z'}\e^{\im(\phi-\phi')}
					-\abs{z}\abs{z'}\e^{-\im(\phi-\phi')}\nonumber\\
				&=\abs{z}^{2}+\abs{z'}^{2}-2\abs{z}\abs{z'}\cos{(\phi-\phi')}\,.
		\end{align}		
		Pak
		\begin{align}
			\braket{z}{z'}
				&=\e^{-\frac{\abs{z}^{2}+\abs{z'}^{2}}{2}}\sum_{n=0}^{\infty}
					\frac{\abs{z}^{n}\abs{z'}^{n}}{n!}\e^{\im n(\phi'-\phi)}\nonumber\\
				&=\e^{-\frac{\abs{z}^{2}+\abs{z'}^{2}}{2}}\sum_{n=0}^{\infty}
					\frac{\left[\abs{z}\abs{z'}\e^{\im(\phi'-\phi)}\right]^{n}}{n!}\nonumber\\
				&=\e^{-\frac{\abs{z}^{2}+\abs{z'}^{2}}{2}}\e^{\abs{z}\abs{z'}
					\left[\cos{(\phi'-\phi)}+\im\sin{(\phi'-\phi)}\right]}\nonumber\\
				&=\e^{-\frac{\abs{z'-z}^{2}}{2}+\im\abs{z}\abs{z'}\sin{(\phi'-\phi)}}\,.
		\end{align}
		Je tedy $\braket{z}{z}=1$, avšak $\braket{z}{z'}\neq 0$ je obecně nenulové komplexní číslo.
		
	\item
		Hledaná pravděpodobnost je dána kvadrátem amplitudy~\eqref{eq:HOCoherentStateBraket}
		\begin{equation}
			p=\abss{\braket{z'}{z}}=\e^{-\abs{z'-z}}\,.
		\end{equation}
	
	\item
		Komplexní číslo $z$ se opět vyjádří pomocí velikosti a fáze~\eqref{eq:z}, 
		kde pro zjednodušení zápisu označíme $r\equiv\abs{z}$.
		Pak $\d z=r\d r\d\phi$ a integrál relací uzavřenosti dá
		\begin{align}
			\int\ket{z}\bra{z}\frac{\d z}{\pi}
				&=\int\e^{-r^{2}}\sum_{n,n'=0}^{\infty}\frac{r^{n}\e^{\im n\phi}}{\sqrt{n!}}
					\frac{r^{n'}\e^{-\im n'\phi}}{\sqrt{n'!}}\ket{n}\bra{n'}
					\frac{r}{\pi}\,\d r\d\phi\nonumber\\
				&=\sum_{n,n'=0}^{\infty}\frac{\ket{n}\bra{n'}}{\pi\sqrt{n!n'!}}
					\underbrace{\int_{0}^{\infty}\e^{-r^{2}}r^{n+n'+1}\d r}_{\rho\equiv r^{2}\,,
					\quad\d\rho=2r\d r}
					\underbrace{\int_{0}^{2\pi}\e^{\im(n-n')\phi}\d\phi}_{2\pi\delta_{nn'}}\nonumber\\
				&=\sum_{n=0}^{\infty}\frac{2\ket{n}\bra{n}}{n!}\frac{1}{2}
					\underbrace{\int_{0}^{\infty}\rho^{n}\e^{-n}\d\rho}_{\Gamma(n+1)=n!}\nonumber\\
				&=\sum_{n=0}^{\infty}\ket{n}\bra{n}=1\,.
		\end{align}
		
	\item
		Pravděpodobnost se získá dosazením do definičního vztahu~\eqref{eq:HOCoherentStateBraket}
		\begin{equation}
			P_{n}
				=\abs{\braket{n}{z}}^{2}
				=\abs{\e^{-\frac{\abs{z}^{2}}{2}}\frac{z^{n}}{\sqrt{n!}}}^{2}\nonumber\\
				=\e^{-\abs{z}^{2}}\frac{\abs{z}^{2n}}{n!}
		\end{equation}
		a srovnáním s Poissonovým rozdělením~\eqref{eq:PoissonDistribution}
		\begin{equation}
			\lambda=\abs{z}^{2}\,.
		\end{equation}			
		
	\item
		Energie vlastního stavu harmonického oscilátor $\ket{n}$ je dána vztahem~\eqref{eq:HOEnergy}.
		Po dosazení
		\begin{equation}
			\label{eq:HOmeanE}
			\mean{E}_{z}				
				=\sum_{n=0}^{\infty}\hbar\Omega\left(n+\frac{1}{2}\right)\e^{-\lambda}
					\frac{\lambda^{n}}{n!}
				=\hbar\Omega\left(\lambda+\frac{1}{2}\right)
				=\hbar\Omega\left(\abs{z}^{2}+\frac{1}{2}\right)\,.
		\end{equation}	
	\end{enumerate}
\end{solution}
