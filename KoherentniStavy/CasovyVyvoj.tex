\subsection{Časový vývoj}
\begin{enumerate}
\item 
    Nalezněte vyjádření koherentního stavu v čase $t$, tj. stav
    \begin{equation}
        \ket{z(t)}=\operator{U}(t)\ket{z},
    \end{equation}
    kde $\operator{U}(t)$ je operátor časového vývoje.

\item 
    Určete střední hodnoty operátorů polohy a hybnosti v čase $t$ 
    \begin{equation}
        \mean{\operator{x}(t)}_{z}\equiv\matrixelement{z(t)}{\operator{x}}{z(t)}\,,
        \qquad\mean{\operator{p}(t)}_{z}\equiv\matrixelement{z(t)}{\operator{p}}{z(t)}\,.
    \end{equation}
    Ukažte, že časový vývoj koherentního stavu lze znázornit jako elipsu v grafu, 
    ve kterém na osy vynášíme $\mean{\operator{x}(t)}_{z}$ proti $\mean{\operator{p}(t)}_{z}$.
    To je konzistentní s dynamikou klasické částice v potenciálu harmonického oscilátoru.

\item 
    Nalezněte podmínku pro hodnotu $z$, 
    za které bude elipsa v klasickém případě a v kvantovém případě pro střední hodnoty stejná.
\end{enumerate}

\begin{solution}
	\begin{enumerate}
	\item
		Evoluční operátor pro harmonický oscilátor zní
		\begin{equation}
			\operator{U}(t)
				=\e^{-\frac{\im}{\hbar}\operator{H}t}
				=\e^{-\im\Omega t\left(\conjugate{\operator{a}}\operator{a}+\frac{1}{2}\right)}\,.
		\end{equation}
		Jeho působení na koherentní stav dává
		\begin{align}
			\ket{z(t)}
				&=\e^{-\frac{\abs{z}^{2}}{2}}\sum_{n=0}^{\infty}\frac{z^{n}}{\sqrt{n!}}
					\e^{-\im\Omega t\left(n+\frac{1}{2}\right)}\ket{n}\nonumber\\
				&=\e^{-\frac{\abs{z}^{2}}{2}-\frac{\im\Omega t}{2}}\sum_{n=0}^{\infty}
					\frac{\left(z\e^{-\im\Omega t}\right)^{n}}{\sqrt{n!}}\ket{n}\nonumber\\
				&=\e^{-\frac{\im\Omega t}{2}}\ket{z\e^{-\im\Omega t}}\,.
		\end{align}
		Při časovém vývoj tedy (až na fázi) číslo $z$ periodicky osciluje s frekvencí $\Omega$,
		\begin{equation}
			z(t)=z\e^{-\im\Omega t}\,.
		\end{equation}
		Jak víme z předchozího příkladu~\ref{sec:CoherentStatesExpectation}, reálná a imaginární část čísla $z$ udávají střední hodnotu polohy a hybnosti~\eqref{eq:HOCoherentStateZ}.
		Tyto střední hodnoty se tedy budou měnit v čase, jak se také explicitně ukáže v následující části úlohy.
		
	\item
		Díky vyjádření komplexního čísla $z$ pomocí rozkladu na velikost a fázi~\eqref{eq:z} lze časový vývoj středních hodnot zapsat ve tvaru
		\begin{subequations}
			\begin{align}
				\mean{\operator{x}(t)}_{z}
					&=\sqrt{\frac{2\hbar}{M\Omega}}\real{z(t)}
					=\sqrt{\frac{2\hbar}{M\Omega}}\abs{z}\cos{\left[\phi_{0}-\Omega t\right]},\\
				\mean{\operator{p}(t)}_{z}
					&=\sqrt{2\hbar M\Omega}\imaginary{z(t)}
					=\sqrt{2\hbar M\Omega}\abs{z}\sin{\left[\phi_{0}-\Omega t\right]},
			\end{align}
			\label{eq:CoherentStatesExpectationTime}
		\end{subequations}
		kde fáze $\phi_{0}$ je fixována střední hodnotou operátorů souřadnice a hybnosti v čase $t=0$.
		Vztahy~\eqref{eq:CoherentStatesExpectationTime} se pak dají přepsat do tvaru
		\begin{equation}
			\frac{\mean{\operator{x}(t)}_{z}^{2}}{A_{q}^{2}}
				+\frac{\mean{\operator{p}(t)}_{z}^{2}}{B_{q}^{2}}=1,
		\end{equation}
		což je rovnice elipsy s poloosami
		\begin{subequations}
			\begin{align}
				A_{q}&=\sqrt{\frac{2\hbar}{M\Omega}}\abs{z},\\
				B_{q}&=\sqrt{2\hbar M\Omega}\abs{z}.
			\end{align}
		\end{subequations}
		
	\item
		V klasickém případě je energie harmonického oscilátoru (klasický Hamiltonián)
		\begin{equation}
			\label{eq:HOCoherentStateClassicalMotion}
			E=\frac{1}{2M}p^{2}+\frac{1}{2}M\Omega^{2}x^{2},
		\end{equation}
		takže
		\begin{equation}
			\frac{x^{2}}{A_{c}^{2}}+\frac{p^{2}}{B_{c}^{2}}=1,
		\end{equation}
		kde
		\begin{subequations}
			\begin{align}
				A_{c}&=\sqrt{\frac{2E}{M\Omega^{2}}},\\
				B_{c}&=\sqrt{2ME}.
			\end{align}
		\end{subequations}
		Střední hodnota energie v kvantovém případě popsaném koherentním stavem $\ket{z}$ je~\eqref{eq:HOmeanE}.
		Pro velké hodnoty $z$ lze zanedbat faktor $\frac{1}{2}$ a přibližně tedy platí
		\begin{equation}
			\mean{E}_{z}\approx\hbar\Omega\abs{z}^{2}\,.
		\end{equation}
		Čím větší je hodnota $z$, tím lépe bude koherentní stav odpovídat pohybu klasické částice o stejné energii.
	\end{enumerate}
\end{solution}    
