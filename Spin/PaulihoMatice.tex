\sec{Pauliho matice}\index{matice!Pauliho}
\begin{equation}
    \important{
        \matrix{\sigma}_{1}
            =\makematrix{0 & 1 \\ 1 & 0},\quad
        \matrix{\sigma}_{2}
            =\makematrix{0 & -\im \\ \im & 0},\quad
        \matrix{\sigma}_{3}
            =\makematrix{1 & 0 \\ 0 & -1}
    }
    \label{eq:Sigma}
\end{equation}
jsou unitární, hermitovské a unimodulární matice s nulovou stopou,
\begin{subequations}
    \begin{align}
        \matrix{\sigma}_{j}
            &=\conjugate{\matrix{\sigma}_{j}}
            =\matrix{\sigma}_{j}^{-1},\\
        \det\matrix{\sigma}_{j}
            &=-1,\\
        \trace\matrix{\sigma}_{j}&=0,
    \end{align}        
    \label{eq:SigmaDetTrace}
\end{subequations}
které splňují komutační relace
\begin{align}
    \commutator{\matrix{\sigma}_{j}}{\matrix{\sigma}_{k}}
        &=2\im\epsilon_{jkl}\matrix{\sigma}_{l}.		
    \label{eq:SigmaCommutator}
\end{align}
Z rovností~\eqref{eq:SigmaDetTrace} vyplývá, že všechny Pauliho matice mají dvě vlastní hodnoty $\lambda_{\pm}=\pm1$.
Příslušné vlastní vektory se běžně značí
\begin{subequations}
    \begin{align}
        \ket{x-}
            &\equiv\ket{\leftarrow}\equiv\frac{1}{\sqrt{2}}\makematrix{1\\-1},
        &\ket{x+}
            &\equiv\ket{\rightarrow}\equiv\frac{1}{\sqrt{2}}\makematrix{1\\1},\label{eq:SigmaX}\\
        \ket{y-}
            &\equiv\ket{\odot}\equiv\frac{1}{\sqrt{2}}\makematrix{1\\-\im},
        &\ket{y+}
            &\equiv\ket{\otimes}\equiv\frac{1}{\sqrt{2}}\makematrix{1\\\im},\label{eq:SigmaY}\\
        \ket{z-}
            &\equiv\ket{\downarrow}\equiv\ket{-}\equiv\makematrix{0\\1},
        &\ket{z+}
            &\equiv\ket{\uparrow}\equiv\ket{+}\equiv\makematrix{1\\0}\label{eq:SigmaZ}
    \end{align}
    \label{eq:SigmaXYZ}
\end{subequations}
(první sloupec odpovídá záporným vlastním hodnotám $\lambda_{-}=-1$, druhý sloupec záporným vlastním hodnotám $\lambda_{+}=+1$; $j$-tý řádek odpovídá matici $\matrix{\sigma}_{j}$).