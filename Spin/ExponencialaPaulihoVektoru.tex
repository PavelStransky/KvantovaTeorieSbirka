\subsection{Exponenciála Pauliho vektoru}
Dokažte, že
\begin{equation}
    \important{
        \e^{\im\alpha\left(\vector{n}\cdot\vector{\matrix{\sigma}}\right)}
            =\matrix{1}\cos{\alpha}+\im\left(\vector{n}\cdot\vector{\matrix{\sigma}}\right)\sin{\alpha},
    }
    \label{eq:SigmaExp}
\end{equation}
kde $\alpha$ je reálný parametr.

\begin{solution}
	Dokazovaný vztah~\eqref{eq:SigmaExp} nabízí \trick{rozvést exponenciálu} na levé straně \trick{do řady}~\eqref{eq:OperatorExp}.
	V řadě se budou kromě číselných koeficientů vyskytovat celočíselné mocniny spinové matice $\matrix{\sigma}_{\vector{n}}^{k}=\left(\vector{n}\cdot\vector{\matrix{\sigma}}\right)^k$, $k\in\mathbb{N}_{0}$.
	Druhá mocnina $k=2$ je rovna jednotkové matici,
	\begin{align}
		\left(\vector{n}\cdot\vector{\matrix{\sigma}}\right)^{2}
			&=\makematrix{\cos{\theta} & \e^{-\im\phi}\sin{\theta} 
					\\ \e^{\im\phi}\sin{\theta} & -\cos{\theta}}
				\makematrix{\cos{\theta} & \e^{-\im\phi}\sin{\theta} 
					\\ \e^{\im\phi}\sin{\theta} & -\cos{\theta}}\nonumber\\
			&=\makematrix{\cos^{2}\theta+\sin^{2}\theta & 0 \\ 0 & \cos^{2}\theta+\sin^{2}\theta}\nonumber\\
			&=\matrix{1},
	\end{align}
	z čehož indukcí vyplývá, že
	\begin{equation}
        \left(\vector{n}\cdot\vector{\matrix{\sigma}}\right)^{2n}
            =\matrix{1},\quad
        \left(\vector{n}\cdot\vector{\matrix{\sigma}}\right)^{2n+1}
            =\vector{n}\cdot\vector{\matrix{\sigma}}.
	\end{equation}
	Rozvoj exponenciály se rozpadne na sudé a liché členy (podobně jako v příkladu~\ref{sec:RotationOperator}),
	které se vysčítají na goniometrické funkce cosinus a sinus,
	\begin{align}
		\e^{\im\alpha\left(\vector{n}\cdot\vector{\matrix{\sigma}}\right)}
			&=\sum_{k=0}^{\infty}\frac{\im^{k}}{k!}
				\alpha^{k}\left(\vector{n}\cdot\vector{\matrix{\sigma}}\right)^{k}\nonumber\\
			&=\matrix{1}\sum_{k=0}^{\infty}\frac{\minus{k}}{(2k)!}\alpha^{2k}
				+\im\left(\vector{n}\cdot\vector{\matrix{\sigma}}\right)
					\sum_{k=0}^{\infty}\frac{\minus{k}}{(2k+1)!}\alpha^{2k+1}\nonumber\\
			&=\matrix{1}\cos{\alpha}+\im\left(\vector{n}\cdot\vector{\matrix{\sigma}}\right)\sin{\alpha}.
	\end{align}
    Vztah~\eqref{eq:SigmaExp} je tímto dokázán.
    
    \begin{note}
        Důkaz lze provést i na základě vyjádření funkce operátoru pomocí spektrálního rozkladu~\eqref{eq:OperatorFncSylvester}.
        To je v obecnosti provedeno v následujícím příkladu~\ref{sec:PauliFunction}.
    \end{note}
\end{solution}
