\subsection{Kvantový oblak}
\begin{enumerate}
\item
    Spočítejte vlastní hodnoty $E_{1,2,3}$ a vlastní vektory $\ket{E_{1,2,3}}$ systému popsaného Hamiltoniánem 
    \begin{equation}
        \matrix{H}
            =\matrix{H}_{0}+\matrix{V},\quad
        \mathrm{kde}\quad 
        \matrix{H}_{0}
            =\makematrix{e & 0 & 0 \\ 
                         0 & e & 0 \\ 
                         0 & 0 & e},\quad
        \matrix{V}
            =\makematrix{0 & v & v \\ 
                         v & 0 & v \\ 
                         v & v & 0}
    \end{equation}
    a načrtněte graf $E_{1,2,3}(v)$.
    
\item
    \emph{Zobecnění předchozího případu:}
    Určete vlastní hodnoty a normalizované ortogonální vlastní vektory Hamiltoniánu $\matrix{H}'$ obecné dimenze $N$,
    \begin{equation}
        \matrix{H}
            =\makematrix{e & v & v & v & \\ v & e & v & v & \\ v & v & e & v & \ldots 
                \\ v & v & v & e & \\ & & \vdots & & \ddots}.
    \end{equation}
    Tento Hamiltonián popisuje například částici na mřížce o $N$ pozicích, přičemž částice může přeskočit na libovolnou pozici mřížky a žádná pozice není upřednostněna.    
\end{enumerate}

