\sec{Kvaterniony}\index{kvaterniony}
Pauliho matice úzce souvisejí s kvaterniony $q=ai+bj+ck+d\in\mathbb{H}$: imaginární jednotky $i,j,k$ kvaternionů splňující
\begin{subequations}
    \begin{align}
        i^2=j^2=k^2=ijk=-1,\\
        ij=k,\quad
        jk=i,\quad
        ki=j
    \end{align}        
\end{subequations}
lze namapovat na Pauliho matice vztahy
\begin{align}
    i&\leftrightarrow-\im\sigma_{1}=\makematrix{0 & -\im \\ -\im & 0},
    &j&\leftrightarrow-\im\sigma_{2}=\makematrix{0 & -1 \\ 1 & 0},
    &k&\leftrightarrow-\im\sigma_{3}=\makematrix{-\im & 0 \\ 0 & \im}.
\end{align}
Kvaternion se pak dá zapsat jako komplexní matice
\begin{equation}
    q=\makematrix{d-\im c & -b-\im a \\ b-\im a & d+\im c}.
\end{equation}
Algebru kvaternionů lze tedy převést na algebru komplexních matic $2\times2$, což souvisí s isomorfismem Lieových algeber $\algebra{u}(1,\mathbb{H})\simeq\algebra{su}(2,\mathbb{C})$.\index{algebra!Lieova}