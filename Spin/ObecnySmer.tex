\subsection{Spin mířící do obecného směru}\label{sec:Pauli}
Na jednotkové kouli (tzv. \emph{Blochově sféře})\index{sféra!Blochova} je zadán jednotkový vektor
\begin{equation}
	\vector{n}
		=\left(n_{1},n_{2},n_{3}\right)
		=\left(\sin\theta\cos\phi,\sin\theta\sin\phi,\cos\theta\right),\qquad
		\abs{\vector{n}}=1,
		\theta\in\left[0,\pi\right),
		\phi\in\left[0,2\pi\right)
\end{equation}			
mířící do obecného směru daného sférickými úhly $(\theta,\phi)$.
\begin{enumerate}
\item
    V souřadnicích $(n_{1},n_{2},n_{3})$ a $(\theta,\phi)$ vyjádřete matici $\matrix{\sigma}_{\vector{n}}\equiv\vector{n}\cdot\vector{\matrix{\sigma}}$, kde $\vector{\matrix{\sigma}}=(\matrix{\sigma}_{1},\matrix{\sigma}_{2},\matrix{\sigma}_{3})$ je tzv. \emph{Pauliho vektor}.\index{vektor!Pauliho}

\item
    Určete vlastní hodnoty a vlastní vektory matice $\matrix{\sigma}_{\vector{n}}$.

\item
    Napište maticové vyjádření projektorů $\operator{P}_{\vector{n}\pm}$\index{projektor} na podprostory odpovídající vlastním hodnotám $\sigma_{\vector{n}\pm}$ matice $\matrix{\sigma}_{\vector{n}}$ a ověřte, že se skutečně jedná o projektory.\index{projektor}

\item
    Dokažte, že platí
    \begin{equation}
        \important{
            \matrix{P}_{\vector{n}\pm}
                =\frac{1}{2}\left(\matrix{1}\pm\matrix{\sigma}_{\vector{n}}\right)
        }.
        \label{eq:SigmaProjector}
    \end{equation}		
\end{enumerate}
	
\begin{solution}
	\begin{enumerate}
	\item
		Spinová matice vyjádřená v kartézských souřadnicích:
		\begin{equation}
			\matrix{\sigma}_{\vector{n}}
				=n_{1}\matrix{\sigma}_{1}+n_{2}\matrix{\sigma}_{2}+n_{3}\matrix{\sigma}_{3}
				=\makematrix{n_{3} & n_{1}-\im n_{2} \\ n_{1}+\im n_{2} & -n_{3}}
		\end{equation}
		a ve sférických souřadnicích:
		\begin{equation}
            \matrix{\sigma}_{\vector{n}}
                =\makematrix{\cos{\theta} & \e^{-\im\phi}\sin{\theta} 
					\\ \e^{\im\phi}\sin{\theta} & -\cos{\theta}}.
            \label{eq:SigmaSpherical}
        \end{equation}			
    
	\item
		Vlastní hodnoty spinové matice získané diagonalizací:
		\begin{gather}			
            \matrix{\sigma}_{\vector{n}}\ket{\vector{n}\pm}
                =\sigma_{\vector{n}\pm}\ket{\vector{n}\pm}\nonumber\\
			\det\makematrix{\cos{\theta}-\sigma_{\vector{n}\pm} & \e^{-\im\phi}\sin{\theta} 
                \\ \e^{\im\phi}\sin{\theta} & -\cos{\theta}-\sigma_{\vector{n}\pm}}
                =0\nonumber\\
            -\left(\cos{\theta}-\sigma_{\vector{n}\pm}\right)\left(\cos{\theta}+\sigma_{\vector{n}\pm}\right)-\sin^{2}\theta
                =0\nonumber\\
            \sigma_{\vector{n}\pm}^{2}-\cos^{2}\theta-\sin^{2}\theta
                =0\nonumber\\
            \sigma_{\vector{n}\pm}
                =\pm1.
		\end{gather}
		Vlastní vektor příslušející $\lambda=1$:
		\begin{gather}
			\makematrix{\cos{\theta} & \e^{-\im\phi}\sin{\theta} 
				    \\ \e^{\im\phi}\sin{\theta} & -\cos{\theta}}
                    \makematrix{x \\ y}
                =+\makematrix{x \\ y}\nonumber\\
            x\e^{\im\phi}\sin{\theta}
                =y\left(\cos{\theta}+1\right)\nonumber\\
            x=y\frac{\cos{\theta+1}}{\sin{\theta}}\e^{-\im\phi}
		\end{gather}
		a jeho normalizace
		\begin{align}
			1&=xx^{*}+yy^{*}\nonumber\\
			 &=\abs{y}^{2}\left[\frac{\cos^{2}\theta+2\cos{\theta}+1}{\sin^{2}\theta}+1\right]\nonumber\\
			 &=\abs{y}^{2}\frac{2\cos{\theta}+2}{\left(1-\cos^{2}\theta\right)}\nonumber\\
			 &=\abs{y}^{2}\frac{2}{1-\cos{\theta}}=\nonumber\\
			 &=\abs{y}^{2}\frac{1}{\sin^{2}\frac{\theta}{2}}.
		\end{align}
		Fázi lze volit libovolně.
		Konkrétní volba s reálnou $y$-ovou složkou odpovídá
        \begin{subequations}
            \begin{align}
                y&=\sin{\frac{\theta}{2}}\,,\\
                x&=\sin{\frac{\theta}{2}}\cotg{\frac{\theta}{2}}\e^{-\im\phi}
                    =\e^{-\im\phi}\cos{\frac{\theta}{2}},
            \end{align}
        \end{subequations}
		takže vlastní vektor odpovídající vlastní hodnotě $\lambda=1$ je
		\begin{equation}
    		\important{
	    		\ket{\vector{n}+}
		    		=\makematrix{\e^{-\im\phi}\cos{\frac{\theta}{2}} \\ \sin{\frac{\theta}{2}}}
		    }.
		\end{equation}
		Obdobně se získá i vlastní vektor k vlastní hodnotě $\lambda=-1$:
		\begin{equation}
    		\important{
                \ket{\vector{n}-}
                    =\makematrix{-\e^{-\im\phi}\sin{\frac{\theta}{2}} \\ \cos{\frac{\theta}{2}}}
		    }.
        \end{equation}
        		
	\item
		Z definice projektoru plyne po dosazení vlastních vektorů
		\begin{align}
			\operator{P}_{\vector{n}+}
				&\equiv\ket{\vector{n}+}\bra{\vector{n}+}\nonumber\\
				&=\makematrix{\e^{-\im\phi}\cos{\frac{\theta}{2}} \\ \sin{\frac{\theta}{2}}}
					\makematrix{\e^{\im\phi}\cos{\frac{\theta}{2}} \\ \sin{\frac{\theta}{2}}}\nonumber\\
				&=\makematrix{\cos^{2}\frac{\theta}{2} & 
						\e^{-\im\phi}\sin{\frac{\theta}{2}}\cos{\frac{\theta}{2}} \\
						\e^{\im\phi}\sin{\frac{\theta}{2}}\cos{\frac{\theta}{2}} &
						\sin^{2}\frac{\theta}{2}}\nonumber\\
				&=\boxed{\frac{1}{2}
					\makematrix{1+\cos{\theta} & \e^{-\im\phi}\sin{\theta} \\
							 \e^{\im\phi}\sin{\theta} & 1-\cos{\theta}}}
			\label{eq:SigmaProjectorPlus}
		\end{align}
		\begin{equation}
			\important{
				\operator{P}_{\vector{n}-}
					=\frac{1}{2}\makematrix{1-\cos{\theta} & -\e^{-\im\phi}\sin{\theta} \\
							   -\e^{\im\phi}\sin{\theta} & 1+\cos{\theta}}
		    }
            \label{eq:SigmaProjectorMinus}
    	\end{equation}		
		Skutečnost, že se jedná o projektory, se dokáže ze \trick{vztahu úplnosti}
		\begin{equation}
			\operator{P}_{\vector{n}+}+\operator{P}_{\vector{n}-}
				=\frac{1}{2}\makematrix{2 & 0 \\ 0 & 2}
				=\matrix{1}
		\end{equation}
		a z vlastnosti \trick{kvadrátu projektoru}
        \begin{subequations}
            \begin{gather}
                \operator{P}_{\vector{n}+}^{2}
                    =\frac{1}{4}
                        \makematrix{1+\cos{\theta} & \e^{-\im\phi}\sin{\theta} \\
                                \e^{\im\phi}\sin{\theta} & 1-\cos{\theta}}
                        \makematrix{1+\cos{\theta} & \e^{-\im\phi}\sin{\theta} \\
                                \e^{\im\phi}\sin{\theta} & 1-\cos{\theta}}
                    =\operator{P}_{\vector{n}+},\\
                \operator{P}_{\vector{n}-}^{2}
                    =\operator{P}_{\vector{n}-}.
            \end{gather}		
        \end{subequations}

    \item
		Vztah~\eqref{eq:SigmaProjector} mezi projektory a spinovou maticí vyplývá z porovnání projektorů~\eqref{eq:SigmaProjectorPlus} a~\eqref{eq:SigmaProjectorMinus} s explicitním vyjádřením matice~\eqref{eq:SigmaSpherical}.		
	\end{enumerate}
\end{solution}