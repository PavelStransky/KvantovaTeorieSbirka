\subsection{Funkce Pauliho vektoru}\index{funkce operátoru}\label{sec:PauliFunction}
Dokažte, že pro funkci definovanou v bodech $\pm\alpha$ platí
\begin{equation}
    \important{
        f\left(\alpha\,\vector{n}\cdot\vector{\matrix{\sigma}}\right)=\frac{f(\alpha)+f(-\alpha)}{2}
            +\frac{f(\alpha)-f(-\alpha)}{2}\vector{n}\cdot\vector{\matrix{\sigma}}
    }.
\end{equation}

\begin{note}
	Pro $f\equiv\exp$ dostaneme výsledek z předchozího příkladu.
\end{note}

\begin{solution}	
	V příkladu~\ref{sec:Pauli} bylo ukázáno, že vlastní hodnoty operátoru 
	$\vector{n}\cdot\vector{\matrix{\sigma}}$ jsou $\pm1$, z čehož vyplývá, že spektrální rozklad operátoru 
	v argumentu funkce je
	\begin{equation}
		\alpha\,\vector{n}\cdot\vector{\matrix{\sigma}}
			=\alpha\operator{P}_{\vector{n}+}-\alpha\operator{P}_{\vector{n}-},
	\end{equation}
	kde $\operator{P}_{\vector{n}\pm}$ jsou projekční operátory na charakteristické podprostory
	příslušných vlastních hodnot $\sigma_{\vector{n}\pm}=\pm1$.
	Podle definice funkce operátoru pomocí spektrálního rozkladu~\eqref{eq:OperatorFncSylvester}\index{formule!Sylvestrova} platí
	\begin{equation}
		f\left(\alpha\,\vector{n}\cdot\vector{\matrix{\sigma}}\right)
			=f(\alpha)\operator{P}_{\vector{n}+}+f(-\alpha)\operator{P}_{\vector{n}-}.
	\end{equation}
	Dosazení vyjádření projekčních operátorů pomocí spinové matice $\matrix{\sigma}_{\vector{n}}=\vector{n}\cdot\vector{\matrix{\sigma}}$
	dané vztahem~\eqref{eq:SigmaProjector} vede na
	\begin{align}
		f\left(\alpha\,\vector{n}\cdot\vector{\matrix{\sigma}}\right)
			&=f(\alpha)\left(\frac{1}{2}+\vector{n}\cdot\vector{\matrix{\sigma}}\right)
				+f(-\alpha)\left(\frac{1}{2}-\vector{n}\cdot\vector{\matrix{\sigma}}\right)\nonumber\\
			&=\frac{f(\alpha)+f(-\alpha)}{2}
			+\frac{f(\alpha)-f(-\alpha)}{2}\vector{n}\cdot\vector{\matrix{\sigma}}.
	\end{align}
	Tím je důkaz hotov.
\end{solution}
