\subsection{Fotoelektrický jev --- plné řešení}\label{sec:Photoeffect}
	Atom vodíku popsaný Hamiltoniánem
	\begin{equation}
		\operator{H}_{0}
			=\frac{1}{2m}\vectoroperator{p}^{2}-\frac{\gamma}{\operator{r}},
	\end{equation}
	je vystaven elektromagnetickému vlnění s vektorovým potenciálem
	\begin{equation}\label{eq:PhotoeffectA}
		\vector{A}(\vectoroperator{r},t)
			=2A_{0}\vector{\epsilon}\cos\left(\vector{\kappa}\cdot\vectoroperator{r}-\omega t\right)
	\end{equation}
	(jednotkový vektor $\vector{\epsilon}$ určuje polarizaci vln, $\vector{\kappa}=\vector{n}\omega/c$ je vlnový vektor určující směr postupu vlny)	a skalárním potenciálem
	\begin{equation}
		\Phi(\vectoroperator{r},t)
			=0.
	\end{equation}

	\begin{enumerate}
		\item Nalezněte interakční Hamiltonián.
		
		\item Nalezněte hustotu pravděpodobnosti vztaženou na jednotku času (rychlost přechodu) jevu, kdy kdy atom vodíku nacházející se v základním stavu emituje elektron do oblasti prostorového úhlu $(\Omega,\Omega+\d\Omega)$ (fotoelektrický jev). 
		
		\item Určete diferenciální účinný průřez výše uvedeného jevu.
	\end{enumerate}

\begin{solution}
	\begin{enumerate}
	\item
		\emph{Interakční Hamiltonián}
	
		Hamiltonián atomu vodíku, popisující interakci jeho elektronu s elektromagnetickým polem, je
		\begin{equation}
            \label{eq:PhotoeffectH}
			\operator{H}_{\ti{(H\leftrightarrow EM)}}
				=\frac{1}{2m}\left(\vectoroperator{p}^{2}-e\vector{A}(\vectoroperator{r},t)\right)^{2}
					+e\,\Phi(\vectoroperator{r})-\frac{\gamma}{\operator{r}}.
		\end{equation}
		Ve speciální (Coulombické) kalibraci
		\begin{equation}
			\nabla\cdot\vector{A}
				=0\,,
				\qquad\Phi
				=0\,.
		\end{equation}
		lze Hamiltonián~\eqref{eq:PhotoeffectH} přepsat do tvaru
		\begin{align}
			\operator{H}_{\ti{(H\leftrightarrow EM)}}(t)
				&=\operator{H}_{0}-\frac{e}{m}\,\vector{A}(\vectoroperator{r},t)\cdot\vectoroperator{p}
					+\frac{e^{2}}{m}\,\vector{A}(\vectoroperator{r},t)\cdot\vector{A}(\vectoroperator{r},t)\approx\nonumber\\
				&\approx\operator{H}_{0}-\frac{e}{m}\,\vector{A}(\vectoroperator{r},t)\cdot\vectoroperator{p},
		\end{align}
		kde se (jako obvykle) zanedbal člen úměrný $\abs{\vector{A}(\vectoroperator{r})}^{2}$.
		Interakční část Hamiltoniánu $\operator{H}_{\ti{I}}(t)=-\frac{e}{m}\,\vector{A}(\vectoroperator{r},t)\cdot\vectoroperator{p}$ po dosazení vektorového potenciálu~\eqref{eq:PhotoeffectA} má tvar
		\begin{equation}\label{eq:PhotoeffectHI}
			\operator{H}_{\ti{I}}(t)
				=-\frac{eA_{0}}{m}
					\left[\e^{\im\left(\vector{\kappa}\cdot\vectoroperator{r}-\omega t\right)}
						+\e^{-\im\left(\vector{\kappa}\cdot\vectoroperator{r}-\omega t\right)}\right]
					\vector{\epsilon}\cdot\vectoroperator{p}\,.
		\end{equation}
		
		Nás bude zajímat excitace, stačí tedy brát pouze část
		\begin{align}
		\operator{H}_{\ti{I}}(t)
			&=\operator{h}\e^{-\im\omega t},\\
		\operator{h}
			&=-\frac{eA_{0}}{m}\e^{\im\vector{\kappa}\cdot\vector{r}}\vector{\epsilon}\cdot\vectoroperator{p}.
		\end{align}
	
	\item
		\emph{Rychlost přechodu}
		
		Vlnová funkce základního stavu atomu vodíku je\footnote{
			Jedná se o radiální i úhlovou část vlnové funkce, viz~\eqref{eq:HydrogenR10} a s ní související poznámka.
		}
		\begin{equation}
			\psi_{i}(\vector{r})
				=R_{10}(r)Y_{00}(\theta,\phi)
				=\frac{1}{\sqrt{\pi a_{0}^{3}}}\e^{-\frac{r}{a_{0}}},
		\end{equation}
		kde $a_{0}$ je Bohrův poloměr.
		Vlnová funkce konečného stavu volného elektronu je ovlivněna Coulombickým polem jádra.
        Toto pole je však rychle odstíněno látkou, která se v okolí jádra vyskytuje.
        Lze tedy uvažovat, že elektron je volný, a jeho vlnová funkce funkce má tudíž tvar rovinné vlny
		\begin{equation}
			\psi_{f}(\vector{r})\equiv\braket{\vector{r}}{\vector{k}}
				=\frac{1}{\sqrt{(2\pi\hbar)^{3}}}\e^{\im \vector{k}\cdot\vector{r}}
		\end{equation}
		kde $\vector{k}$ je vlnový vektor elektronu s energií $E_{e}$,
		\begin{equation}
			E_{e}=\frac{\hbar^{2}k^{2}}{2m}.
		\end{equation}
		
		Výpočet přechodu mezi spojitou a diskrétní částí spektra se zjednoduší tím, že budeme považovat elektron nikoliv za zcela volný, ale za uzavřený v krabici (nekonečně hluboké potenciálové jámě) o objemu $V$. 
		Předpokládá se, že krabice je tak velká, že neovlivní příliš spektrum atomu vodíku (přesněji stačí, aby neovlivnila základní stav, se kterým počítáme).
		Nakonec se provede limita $V\rightarrow\infty$.
		
		Vlnová funkce elektronu v krabici zní
		\begin{equation}
			\psi'_{f}(\vector{r})
				=\frac{1}{\sqrt{V}}\e^{\im \vector{k}\cdot\vector{r}},
		\end{equation}
		kde $\vector{k}$ je důsledkem konečných rozměrů krabice kvantovaná veličina,
		\begin{align}
			E_{\vector{n}}&=\frac{\hbar^{2}\vector{k}^{2}}{2m}\,,&
			\vector{k}&=\frac{2\pi}{L}\vector{n}\,,&
			L&=\sqrt[3]{V}\,,&
			\vector{n}&=\left(n_{1},n_{2},n_{3}\right)\,.
		\end{align}
		Pokud je však objem $V$ dostatečně velký, lze s $\vector{k}$ nadále počítat jako se spojitou veličinou.
		
		Při výpočtu hustoty hladin volného elektronu se vyjde ze vztahů \eqref{eq:PhaseSpaceVolume} a \eqref{eq:LevelDensity}.
		Objem energetické nadplochy fázového prostoru klasicky se pohybujícího volného elektronu v krabici je podle nich
		\begin{align}
		\Omega_{\ti{PS}}(E)
			&=\int_{V}\d^{3}\vector{x}\int\delta\left(E-\frac{1}{2m}\vector{p}^{2}\right)
				\d^{3}\vector{p}=\nonumber\\
			&=V\int_{\Omega}\d\Omega\int_{0}^{\infty}\delta\left(E-\frac{1}{2m}p^{2}\right)
				p^{2}\d p.
		\end{align}
		
		\uv{Diferenciální} Hustota hladin pro elektron letící do směru daného elementem prostorového úhlu $\d\Omega$ je
		\begin{align}
			\frac{\d\rho(E)}{\d\Omega}
				&=\frac{1}{(2\pi\hbar)^{3}}\frac{\d\Omega_{\ti{PS}}(E)}{\d\Omega}=\nonumber\\
				&=\frac{V}{(2\pi\hbar)^{3}}\int_{0}^{\infty}\delta\left(E-\frac{1}{2m}p^{2}\right)
					p^{2}\d p=\nonumber\\
				&=\frac{V}{(2\pi\hbar)^{3}}\frac{2m}{2\sqrt{2mE}}\int_{0}^{\infty}
					\delta\left(p-\sqrt{2mE}\right)p^{2}\d p=\nonumber\\
				&=\frac{V}{(2\pi\hbar)^{3}}\frac{2m}{2\sqrt{2mE}}\,2mE=\nonumber\\
				&=\frac{V}{(2\pi\hbar)^{3}}\,{m}\sqrt{2mE},
		\end{align}
		nebo v závislosti na veličině $k$
		\begin{equation}
			\frac{\d\rho(k)}{\d\Omega}=\frac{V}{(2\pi\hbar)^{3}}\,\hbar km.
		\end{equation}
		
		K výpočtu pravděpodobnosti, resp. rychlosti přechodu se využije Fermiho zlaté pravidlo~\eqref{eq:FermiGoldenRuleHarmonic}.
		V něm se objevuje maticový element
		\begin{align}\label{eq:Photoeffecthfi}
			h_{fi}
				&=\matrixelement{f}{\operator{h}}{i}=\nonumber\\
				&=-\frac{eA_{0}}{m}
					\int\psi_{f}^{'*}(\vector{r})\e^{\im\vector{\kappa}\cdot\vector{r}}
						\vector{\epsilon}\cdot\vector{p}\,
					\psi_{i}(\vector{r})\d^{3}\vector{r}=\nonumber\\
				&=\frac{\im\hbar eA_{0}}{m\sqrt{\pi a_{0}^{3}V}}\,\vector{\epsilon}\cdot
					\int\e^{\im(\vector{\kappa}-\vector{k})\cdot
					\vector{r}}\nabla\e^{-\frac{r}{a_{0}}}\d^{3}\vector{r}=\nonumber\\
				&=-\frac{\im\hbar eA_{0}}{ma_{0}\sqrt{\pi a_{0}^{3}V}}\,\vector{\epsilon}\cdot
					\int\e^{\im\vector{q}\cdot\vector{r}}\frac{\vector{r}}{r}
					\e^{-\frac{r}{a_{0}}}\d^{3}\vector{r}=\nonumber\\
				&=-\frac{\im\hbar eA_{0}}{ma_{0}\sqrt{\pi a_{0}^{3}V}}\,\vector{\epsilon}\cdot
					\vector{I}(\vector{q}),
		\end{align}
		kde $\vector{q}\equiv\vector{\kappa}-\vector{k}$.
		
        \trick{
		Integrál $\vector{I}(\vector{q})$ se vypočítá následující úvahou.
        Výsledkem integrace musí být vektor.
		Jediný vektor, na kterém integrand integrálu závisí, je $\vector{q}$.
		To znamená, že integrál musí být možné vyjádřit jako
		\begin{equation}
			\vector{I}(\vector{q})=\vector{q}I(\vector{q}),
		\end{equation}
		kde $I(\vector{q})$ je již skalární funkce.}
		Počítejme tedy počítat výraz
		\begin{align}
		q^{2}I(\vector{q})
			&=\int\e^{\im\vector{q}\cdot\vector{r}}\frac{\vector{q}\cdot\vector{r}}{r}
				\e^{-\frac{r}{a_{0}}}\d^{3}\vector{r}=\nonumber\\
			&=\equationcomment{\text{Sférické souřadnice }(r,\theta,\phi) 
				\\ \text{osa }z\text{ paralelní s vektorem }\vector{q}}=\nonumber\\
			&=\int_{0}^{\infty}r\e^{-\frac{r}{a_{0}}}\d r
				\int_{0}^{\pi}\e^{\im qr\cos{\theta}}qr\cos{\theta}\sin{\theta}\,\d\theta
				\int_{0}^{2\pi}\d\phi=\nonumber\\
			&=\equationcomment{u=\cos{\theta} & \d u=-\sin{\theta}\,\d\theta}=\nonumber\\
			&=2\pi q\int_{0}^{\infty}r\e^{-\frac{r}{a_{0}}}\d r
				\int_{-1}^{1}\e^{\im qru}u\,\d u\overset{\text{Per partes}}{=}\nonumber\\
			&=2\pi q\int_{0}^{\infty}r\e^{-\frac{r}{a_{0}}}\d r
				\left\{\left[\frac{1}{\im qr}\e^{\im qru}u\right]_{-1}^{1}
					-\frac{1}{\im qr}\int_{-1}^{1}\e^{\im qru}\d u\right\}=\nonumber\\
			&=-2\pi q\im\int_{0}^{\infty}r\e^{-\frac{r}{a_{0}}}\d r
				\left\{\frac{1}{qr}\left(\e^{\im qr}+\e^{-\im qr}\right)
					+\frac{\im}{(qr)^{2}}\left(\e^{\im qr}-\e^{-\im qr}\right)\right\}=\nonumber\\
			&=-2\pi\im\int_{0}^{\infty}r\left\{\e^{-r\left(\frac{1}{a_{0}}+\im q\right)}
				+\e^{-r\left(\frac{1}{a_{0}}-\im q\right)}\right\}\d r-\nonumber\\
			&\qquad\qquad-\frac{2\pi}{q}\int_{0}^{\infty}
				\left\{\e^{-r\left(\frac{1}{a_{0}}+\im q\right)}
				-\e^{-r\left(\frac{1}{a_{0}}-\im q\right)}\right\}\d r=\nonumber\\
			&=-2\pi\im\,J_{1}-\frac{2\pi}{q}\,J_{2}.
		\end{align}
		Platí
        \begin{subequations}
            \begin{align}
                \int_{0}^{\infty}\e^{-\alpha r}\d r
                    &=\frac{1}{\alpha},\\
                \int_{0}^{\infty}r\e^{-\alpha r}\d r
                    &=\frac{1}{\alpha}\int_{0}^{\infty}\e^{-\alpha r}\d r=\frac{1}{\alpha^{2}}
            \end{align}                
        \end{subequations}
		(pro $\alpha>0$), takže
		\begin{align}
			J_{1}
				&=\frac{1}{\left(\frac{1}{a_{0}}+\im q\right)^{2}}
					+\frac{1}{\left(\frac{1}{a_{0}}-\im q\right)^{2}}
				=a_{0}^{2}\frac{\left(1-\im qa_{0}\right)^{2}+\left(1+\im qa_{0}\right)^{2}}
						{\left(1+q^{2}a_{0}^{2}\right)^{2}}=\nonumber\\
				&=2a_{0}^{2}\frac{1-q^{2}a_{0}^{2}}{\left(1+q^{2}a_{0}^{2}\right)^{2}},\\
			J_{2}
				&=\frac{1}{\frac{1}{a_{0}}+\im q}-\frac{1}{\frac{1}{a_{0}}-\im q}
					=a_{0}\frac{1-\im qa_{0}-1-\im qa_{0}}{1+q^{2}a_{0}^{2}}=\nonumber\\
				&=-\frac{2\im qa_{0}^{2}}{1+q^{2}a_{0}^{2}},
		\end{align}
		a po dosazení 
		\begin{align}
			q^{2}I(\vector{q})
				&=-4\pi\im a_{0}^{2}
					\left[\frac{1-a_{0}^{2}q^{2}}{\left(1+q^{2}a_{0}^{2}\right)^{2}}
						-\frac{1}{1+q^{2}a_{0}^{2}}\right]=\nonumber\\
				&=-4\pi\im a_{0}^{2}\frac{1-a_{0}^{2}q^{2}-1-a_{0}^{2}q^{2}}
					{\left(1+q^{2}a_{0}^{2}\right)^{2}}=\nonumber\\
				&=\frac{8\im\pi a_{0}^{4}q^{2}}{\left(1+q^{2}a_{0}^{2}\right)^{2}}.
		\end{align}
		Hledaný integrál má tedy tvar
		\begin{equation}
			\vector{I}(\vector{q})
				=\frac{8\im\pi a_{0}^{4}}{\left(1+q^{2}a_{0}^{2}\right)^{2}}\,\vector{q}.
		\end{equation}
		Maticový element je
		\begin{align}
			h_{fi}
				&=\frac{\im\hbar eA_{0}}{ma_{0}\sqrt{\pi a_{0}^{3}V}}
					\frac{8\im\pi a_{0}^{4}}{\left(1+q^{2}a_{0}^{2}\right)^{2}}\,\vector{\epsilon}\cdot(\vector{\kappa}-\vector{k})\nonumber\\
				&=-\frac{\im\hbar eA_{0}}{ma_{0}\sqrt{\pi a_{0}^{3}V}}
					\frac{8\im\pi a_{0}^{4}}{\left(1+q^{2}a_{0}^{2}\right)^{2}}\,\vector{\epsilon}\cdot\vector{k}\,,
		\end{align}
		neboť $\vector{\epsilon}\cdot\vector{\kappa}=0$, což plyne z vlastností Coulombické kalibrace.
							
		Nyní již máme v rukou vše, co potřebujeme k použití Fermiho zlatého pravidla~\eqref{eq:FermiGoldenRuleHarmonic}.
		Po dosazení do něj vychází
		\begin{align}\label{eq:PhotoefectTransitionRate}
			\frac{\d w_{i\rightarrow f}}{\d\Omega}
				&=\frac{2\pi}{\hbar}\abs{h_{fi}}^{2}\frac{\d\rho}{\d\Omega}=\nonumber\\
				&=\frac{2\pi}{\hbar}
					\abs{\frac{\im\hbar eA_{0}}{ma_{0}\sqrt{\pi a_{0}^{3}V}}
					\frac{8\im\pi a_{0}^{4}}{\left(1+q^{2}a_{0}^{2}\right)^{2}}\,
						\vector{\epsilon}\cdot\vector{k}}^{2}
					\frac{V}{(2\pi\hbar)^{3}}\,\hbar km=\nonumber\\
				&=\frac{16}{\pi\hbar}\frac{(eA_{0})^{2}}{m}
					\frac{\left(\vector{\epsilon}\cdot\vector{k}\right)^{2}ka_{0}^{3}}
						{\left(1+q^{2}a_{0}^{2}\right)^{4}}
		\end{align}
		
	\item
		\emph{Účinný přůřez}
	
		Účinný průřez procesu je definován jako počet procesů $i\rightarrow f$ za jednotku času dělenou celkovým tokem částic.
		V našem případě je to absorbovaná energie za jednotku času dělená tokem energie dopadajícího elektromagnetického záření.
		
		Absorbovaná energie za jednotku času je dána součinem rychlosti přechodu~\eqref{eq:PhotoefectTransitionRate} a energie, která se absorbuje a která je rovna $\hbar\omega$:
		\begin{equation}
			\mathcal{U}_{i\rightarrow f}
				=\hbar\omega\frac{\d w_{i\rightarrow f}}{\d\Omega}
		\end{equation}					
		Tok energie $\Phi$ je součin rychlosti přenosu energie $c$ a střední hustoty energie
		\begin{equation}
			\langle w\rangle=\frac{\epsilon_{0}}{2}\left(\left\langle\vector{E}^{2}\right\rangle
				+c^{2}\left\langle\vector{B}^{2}\right\rangle\right)\,,
		\end{equation}
		kde vektory elektrické intenzity a magnetické indukce jsou
        \begin{subequations}
            \begin{align}
                \vector{E}
                    &=-\partialderivative{\vector{A}}{t}=-2A_{0}\omega\vector{\epsilon}
                        \sin\left(\vector{\kappa}\cdot\vector{r}-\omega t\right)\,,\\
                \vector{B}
                    &=\nabla\times\vector{A}=-2A_{0}\vector{\epsilon}\times\vector{\kappa}
                        \sin\left(\vector{\kappa}\cdot\vector{r}-\omega t\right)\,,
            \end{align}                
        \end{subequations}
		takže
        \begin{subequations}
            \begin{align}
                \left\langle\vector{E}^{2}\right\rangle
                    &=4\abs{A_{0}}^{2}\omega^{2}\left\langle\sin^{2}\left(\vector{\kappa}\cdot\vector{r}-\omega t\right)\right\rangle=2\abs{A_{0}}^{2}\omega^{2},\\
                \left\langle\vector{B}^{2}\right\rangle
                    &=2\abs{A_{0}}^{2}\kappa^{2}=2\abs{A_{0}}^{2}\frac{\omega^{2}}{c^{2}}.
            \end{align}                
        \end{subequations}
		To dává
        \begin{subequations}
            \begin{align}
                \left\langle w\right\rangle&=2\epsilon_{0}\abs{A_{0}}^{2}\omega^{2},\\
                \Phi=c\left\langle w\right\rangle&=2c\epsilon_{0}\abs{A_{0}}^{2}\omega^{2}
            \end{align}							                
        \end{subequations}
		a hledaný účinný průřez je po dosazení
		\begin{align}
			\frac{\d\sigma_{i\rightarrow f}}{\d\Omega}
				&=\frac{\mathcal{U}_{i\rightarrow f}}{\Phi}
				 =\frac{\hbar\omega}{2c\epsilon_{0}\omega^{2}\abs{A_{0}}^{2}}
					\frac{\d w_{i\rightarrow f}}{\d\Omega}
				=\frac{32\gamma}{mc\omega}
					\frac{\left(\vector{\epsilon}\cdot\vector{k}\right)^{2}ka_{0}^{3}}
						{\left(1+q^{2}a_{0}^{2}\right)^{4}}
				=\frac{32\alpha\hbar}{m\omega}
					\frac{\left(\vector{\epsilon}\cdot\vector{k}\right)^{2}ka_{0}^{3}}
						{\left(1+q^{2}a_{0}^{2}\right)^{4}},
            \label{eq:PhotoeffectCrossSection}
		\end{align}
		kde $\alpha=\gamma/(\hbar c)$ je konstanta jemné struktury.
	\end{enumerate}

	Zaveďme ještě speciální souřadnou soustavu tak, aby vektor polarizace $\vector{\epsilon}$ mířil do směru osy $x$ a vlnový vektor dopadající vlny $\vector{\kappa}$ do směru osy $z$.
	Ve sférických souřadnicích pak bude
    \begin{subequations}
        \begin{align}
            \vector{\epsilon}\cdot\vector{k}
                &=k\sin{\theta}\cos{\phi},\\
            q^{2}
                &=k^{2}-2\vector{k}\cdot\vector{\kappa}+\kappa^{2}
                =k^{2}-2k\frac{\omega}{c}\cos{\theta}+\left(\frac{\omega}{c}\right)^{2}.
        \end{align}            
    \end{subequations}
	
	Při výpočtu se potichu předpokládalo, že Coulombické pole neovlivní pohyb vyraženého elektronu a ten se tudíž pohybuje jako volný.
	To však platí pouze v případě, že $k\gg\abs{E_{0}}$, kde $E_{0}$ je energie základního stavu atomu.
	Tuto aproximaci lze rozvést ještě dál.
	Energie, kterou získá vylétávající elektron, je díky tomuto přiblížení rovna přibližně energii dopadajících fotonů:
	\begin{equation}
		k=\frac{\sqrt{2mE_{e}}}{\hbar}
		 =\sqrt{\frac{2m\omega}{\hbar}}.
	\end{equation}
	Jelikož $\kappa=\omega/c$, platí
	\begin{equation}
		\frac{\kappa}{k}
			=k\frac{\kappa}{k^{2}}
			=\frac{\hbar k}{2mc}
			=\frac{p}{2mc}
			=\frac{v}{2c},
	\end{equation}
	a
	\begin{equation}
		q^{2}\approx k^{2}\left(1-\frac{v}{c}\cos{\theta}\right),
	\end{equation}
	kde $v$ je rychlost vylétnuvšího elektronu.
	Tento výraz navíc pomůže zjednodušit jmenovatel diferenciálního účinného průřezu~\eqref{eq:PhotoeffectCrossSection}
	\begin{align}
		1+q^{2}a_{0}^{2}
			&\approx1+k^{2}a_{0}^{2}\left(1-\frac{v}{c}\cos{\theta}\right)\approx\nonumber\\
			&\approx k^{2}a_{0}^{2}\left(1-\frac{v}{c}\cos{\theta}\right).
	\end{align}
	Diferenciální účinný průřez pak bude v této aproximaci
	\begin{equation}
		\frac{\d\sigma_{i\rightarrow f}}{\d\Omega}
			=\frac{32\alpha\hbar}{m\omega\left(ka_{0}\right)^{5}}
				\frac{\sin^{2}\theta\cos^{2}\phi}{\left(1-\frac{v}{c}\cos{\theta}\right)^{4}}.
	\end{equation}
	Tato funkce nabývá maxima pro $\phi=0$, tj. v rovině polarizace dopadající elektromagnetické vlny, a pro $\theta$ dané rovnicí
	\begin{align}
		\frac{\d}{\d\theta}\frac{\sin^{2}\theta}{\left(1-\frac{v}{c}\cos{\theta}\right)^{4}}
			&=0\nonumber\\
		2\sin{\theta}\cos{\theta}\left(1-\frac{v}{c}\cos{\theta}\right)^{4}
			-4\frac{v}{c}\sin^{2}\theta\sin{\theta}\left(1-\frac{v}{c}\cos{\theta}\right)^{3}
			&=0\nonumber\\
		\cos{\theta}-\frac{v}{c}\cos^{2}\theta-2\frac{v}{c}\sin^{2}\theta
			&=0
	\end{align}
	tj.
	\begin{equation}
		\cos{\theta}
			=\frac{-1\pm\sqrt{1+8\left(\frac{v}{c}\right)^{2}}}{2\frac{v}{c}}
			\approx\frac{c}{2v}\left[-1\pm1\pm4\left(\frac{v}{c}\right)^{2}\right]
			\approx\begin{cases}
				{\displaystyle -\frac{c}{v}} & \\ {\displaystyle 2\frac{v}{c}} &
			\end{cases}.
	\end{equation}
	První řešení nevyhovuje, neboť cosinus by musel být menší než -1, což pro reálné úhly $\theta$ nelze splnit.
    Maximální pravděpodobnost emise při fotoefektu tedy vychází do směru
	\begin{equation}
        \important{
            \begin{aligned}
                \theta&=\frac{\pi}{2}-2\frac{v}{c}\,, & \phi&=0
            \end{aligned}
        }.
	\end{equation}
	
\begin{note}
	Integrál \eqref{eq:Photoeffecthfi} lze vypočítat také tak, že se operátor $\nabla$ posouvá vlevo.		
	Posunutí skrz člen $\e^{\im\vector{\kappa}\cdot\vector{r}}$	lze provést přímo díky Coulombické kalibraci (směr šíření elektromagnetické vlny je kolmý na polarizaci).
	Posunutí skrz člen $\e^{-\im\vector{k}\cdot\vector{r}}$ se provede pomocí integrace \emph{per partes}:
	povrchový příspěvek je $0$ a gradient po zapůsobení na tento člen dá pouze faktor $-\im\vector{k}$,	který je možné vytknout před integrál.
	Nakonec tedy stačí integrovat pouze
	\begin{equation}
		h_{fi}
			=-\frac{\hbar eA_{0}}{mca_{0}\sqrt{\pi a_{0}^{3}V}}\,\vector{\epsilon}\cdot\vector{k}
				\int\e^{\im\vector{q}\cdot\vector{r}}\e^{-\frac{r}{a_{0}}}\d^{3}\vector{r},
	\end{equation}
	což je vlastně Fourierova transformace vlnové funkce základního stavu atomu vodíku.
\end{note}

\end{solution}
