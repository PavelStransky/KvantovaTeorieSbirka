Jsou zadány dva nezávislé operátory momentu hybnosti $\vectoroperator{J}^{(1)}$, $\vectoroperator{J}^{(2)}$, $\commutator{\vectoroperator{J}^{(1)}}{\vectoroperator{J}^{(2)}}=0$, které působí na Hilbertových prostorech $\hilbert{H}^{(1)}$, $\hilbert{H}^{(2)}$.
Operátor celkového impulsmomentu\footnote{
    Formálně by se mělo správně psát
    \begin{equation}
        \vectoroperator{J}=\vectoroperator{J}^{(1)}\otimes\vectoroperator{1}^{(2)}+\vectoroperator{1}^{(1)}\otimes\vectoroperator{J}^{(2)}\,,
    \end{equation}
    což ve složkách znamená
    \begin{equation}
        \operator{J}_{j}=\operator{J}_{j}^{(1)}\otimes\operator{1}^{(2)}+\operator{1}^{(1)}\otimes\operator{J}_{j}^{(2)}\,,
        \qquad j=1,2,3\,,
    \end{equation}
    ve shodě s již dříve použitým operátorem dvou spinů~\eqref{eq:TwoSpins}.
}
\begin{equation}
    \vectoroperator{J}=\vectoroperator{J}^{(1)}+\vectoroperator{J}^{(2)}
\end{equation}
pak působí na Hilbertově prostoru $\hilbert{H}=\hilbert{H}^{(1)}\otimes\hilbert{H}^{(2)}$.
Mezi jednotlivými operátory a jejich složkami platí komutační relace
\begin{subequations}
    \begin{align}
        \commutator{\operator{J}_{j}}{\operator{J}_{k}}
            &=\im\epsilon_{jkl}\operator{J}_{l}\,,\\
        \commutator{\operator{J}_{j}}{\vectoroperator{J}^{2}}
            &=0\,,\\
        \commutator{\vectoroperator{J}^{2}}{\vectoroperator{J}^{(1)2}}
            &=\commutator{\vectoroperator{J}^{2}}{\vectoroperator{J}^{(2)2}}=0\,,\\
        \commutator{\operator{J}_{j}}{\vectoroperator{J}^{(1)2}}
            &=\commutator{\operator{J}_{j}}{\vectoroperator{J}^{(2)2}}=0\,.
    \end{align}
\end{subequations}
Z toho vyplývá, že na prostoru $\hilbert{H}$ lze volit za úplnou množinu komutujících operátorů jednu z následujících dvou množin operátorů se svými bázemi:
\begin{equation}
    \begin{array}{lcl}
        \vectoroperator{J}^{(1)2},\operator{J}^{(1)}_{3},\vectoroperator{J}^{(2)2},\operator{J}^{(2)}_{3} 
            & \longrightarrow 
            & \{\ket{j_{1}\,m_{1}}\otimes\ket{j_{2}\,m_{2}}\}\\
        \vectoroperator{J}^{(1)2},\vectoroperator{J}^{(2)2},\vectoroperator{J}^{2},\operator{J}_{3} 
            & \longrightarrow 
            & \{\ket{j_{1}\,j_{2}\,j\,m}\}
    \end{array}
\end{equation}
(dále budeme užívat zjednodušené značení $\ket{j_{1}\,m_{1}}\otimes\ket{j_{2}\,m_{2}}
\equiv\ket{j_{1}\,m_{1}}\ket{j_{2}\,m_{2}}$).
Platí tedy
\begin{subequations}
    \begin{align}
        \vectoroperator{J}^{(1)2}\ket{j_{1}l_{2}lm}
            &=j_{1}(j_{1}+1)\ket{j_{1}\,j_{2}\,j\,m}\,,\\
        \vectoroperator{J}^{(2)2}\ket{j_{1}\,j_{2}\,j\,m}
            &=j_{2}(j_{2}+1)\ket{j_{1}\,j_{2}\,j\,m}\,,\\
        \vectoroperator{J}^{2}\ket{j_{1}\,j_{2}\,j\,m}
            &=j(j+1)\ket{j_{1}\,j_{2}\,j\,m}\,,\\
        \operator{J}_{3}\ket{j_{1}\,j_{2}\,j\,m}
            &=m\ket{j_{1}\,j_{2}\,j\,m}\,,
    \end{align}
\end{subequations}
přičemž kvantová čísla musejí splňovat
\begin{equation}
    \label{eq:AngularMomentumSelectionRules}
    \important{\abs{j_{1}-j_{2}}\leq j\leq j_{1}+j_{2}\,,\qquad m_{1}+m_{2}=m}.
\end{equation}
Mezi oběma bázemi platí vztah
\begin{equation}
    \important{
        \ket{j_{1}\,j_{2}\,j\,m}
            =\sum_{m_{1}\,m_{2}}\clebsch{j_{1}}{m_{1}}{j_{2}}{m_{2}}{j}{m}
                \ket{j_{1}\,m_{1}}\ket{j_{2}\,m_{2}}
    },
\end{equation}
kde $\clebsch{j_{1}}{m_{1}}{j_{2}}{m_{2}}{j}{m}$ 
jsou \emph{Clebsch-Gordanovy koeficienty}\footnote{
    Jiné způsoby zápisu Clebsch-Gordanových koeficientů používané v literatuře jsou
    \begin{equation}
            C^{jm}_{j_{1}\,m_{1}\,j_{2}\,m_{2}}
            =\braket{j_{1}\,m_{1}\,j_{2}\,m_{2}}{j\,m}
            =\left(j_{1}\,j_{2}\,j|m_{1}\,m_{2}\,m\right)
    \end{equation}
}.
