\subsection[Explicitní výpočet C-G koeficientů]{Explicitní výpočet Clebsch-Gordanových koeficientů}
Explicitním výpočtem pomocí posunovacích operátorů $\operator{J}_{\pm}=\operator{J}_{1}\pm\im\operator{J}_{2}$ 
nalezněte Clebsch-Gordanovy koeficienty pro skládání impulsmomentů $j_{1}=j_{2}=1$.

\begin{solution}
	Budeme užívat zkrácený zápis
	\begin{subequations}
		\begin{align}
			\ket{j_{1}\,j_{2}\,j\,m}
				=\ket{1\,1\,\,j\,m}
				&\rightarrow\ket{j\,m}\\
			\ket{j_{1,2}\,m_{1,2}}
				=\ket{1\,m_{1,2}}
				&\rightarrow\ket{m_{1,2}}
		\end{align}
	\end{subequations}
	Na základě trojúhelníkové nerovnosti~\eqref{eq:AngularMomentumSelectionRules} může celkový moment hybnosti nabývat pouze hodnot $j\in\{0,1,2\}$.

	\begin{itemize}
	\item 
		Začíná se obvykle s vektory s nejvyšší váhou:
		\begin{equation}
			\label{eq:AngularMomentum22}
			\ket{2\,2}
				=\ket{1}\ket{1}
		\end{equation}
		(fázi můžeme volit obecně libovolně, jednička je v tzv. 
		\emph{Condon-Shortleyově fázové konvenci}), takže
		\begin{equation}
			\clebsch{1}{1}{1}{1}{2}{2}
				=1.
		\end{equation}
	
	\item
		K výpočtu dalších Clebsch-Gordanových koeficientů v podprostoru $j=2$ se využívá postupného působení posunovacími operátory $\operator{J}_{\pm}$, které splňují
		\begin{equation}
			\label{eq:AngularMomentumShiftOperator}
			\important{
			\begin{aligned}
				\operator{J}_{\pm}\ket{j\,m}
					&=\alpha^{(\pm)}(j,m)\ket{j\,m\pm1}\,,\\
				\alpha^{(\pm)}(j,m)
					&=\sqrt{j(j+1)-m(m\pm1)}
			\end{aligned}
			}
		\end{equation}
		(analogické vztahy platí pro jednotlivé impulsmomenty $\vectoroperator{J}^{(1,2)}$, přičemž $\operator{J}_{\pm}=\operator{J}^{(1)}_{\pm}+\operator{J}^{(2)}_{\pm}$).
		Působení operátoru $\operator{J}_{-}$ na obě strany rovnosti~\eqref{eq:AngularMomentum22} dá
		\begin{subequations}
			\begin{align}
				\operator{J}_{-}\ket{2\,2}
					&=2\ket{2\,1}\,,\\
				\operator{J}_{-}\ket{1}\ket{1}
					&=\operator{J}^{(1)}_{-}\ket{1}\ket{1}+\operator{J}^{(2)}_{-}\ket{1}\ket{1}\nonumber\\
					&=\sqrt{2}\left(\ket{0}\ket{1}+\ket{1}\ket{0}\right),
			\end{align}
		\end{subequations}		
		z čehož vyplývá, že
		\begin{equation}
			\ket{2\,1}
				=\frac{1}{\sqrt{2}}\left(\ket{1}\ket{0}+\ket{0}\ket{1}\right),
		\end{equation}
		a tedy
		\begin{equation}
			\clebsch{1}{0}{1}{1}{2}{1}
				=\clebsch{1}{1}{1}{0}{2}{1}
				=\frac{1}{\sqrt{2}}.
		\end{equation}
	
	\item
		Jelikož musí platit $m=m_{1}+m_{2}$, viz~\eqref{eq:AngularMomentumSelectionRules}, jsou ostatní Clebsch-Gordanovy koeficienty s $j=2$ a $m=1$ nulové:
		\begin{equation}
			\clebsch{1}{1}{1}{1}{2}{1}
				=\clebsch{1}{0}{1}{0}{2}{1}
				=\clebsch{1}{-1}{1}{0}{2}{1}
				=\clebsch{1}{0}{1}{-1}{2}{1}
				=\clebsch{1}{-1}{1}{-1}{2}{1}=0.
		\end{equation}
	
	\item
		Další působení posunovacího operátoru vede na
		\begin{subequations}
			\begin{align}
				\operator{J}_{-}\ket{2\,1}
					&=\sqrt{6}\ket{2\,0}\,,\\
				\operator{J}_{-}\frac{1}{\sqrt{2}}\left(\ket{0}\ket{1}+\ket{1}\ket{0}\right)
					&=\frac{1}{\sqrt{2}}\left(\sqrt{2}\ket{-1}\ket{1}+\sqrt{2}\ket{0}\ket{0}
						+\sqrt{2}\ket{0}\ket{0}+\sqrt{2}\ket{1}\ket{-1}\right)=\nonumber\\
					&=\ket{-1}\ket{1}+2\ket{0}\ket{0}+\ket{1}\ket{-1}\,,
			\end{align}
		\end{subequations}
		takže
		\begin{equation}
			\ket{2\,0}
				=\frac{1}{\sqrt{6}}\left(\ket{-1}\ket{1}+2\ket{0}\ket{0}+\ket{1}\ket{-1}\right)\,.
		\end{equation}
		Odpovídající Clebsch-Gordanovy koeficienty jsou
		\begin{subequations}
			\begin{align}
				\clebsch{1}{-1}{1}{1}{2}{0}
					=\clebsch{1}{1}{1}{-1}{2}{0}
					&=\frac{1}{\sqrt{6}}\\
				\clebsch{1}{0}{1}{0}{2}{0}
					&=\frac{2}{\sqrt{6}}.
			\end{align}
		\end{subequations}
		Všechny ostatní koeficienty s $l=2$, $m=0$ jsou nulové.
	
	\item
		Opakování postupu působení operátorem $\operator{J}_{-}$ dá 
		\begin{subequations}
			\begin{align}
				\ket{2\,-1}
					&=\frac{1}{\sqrt{2}}\left(\ket{0}\ket{-1}+\ket{-1}\ket{0}\right),\\
				\ket{2\,-2}
					&=\ket{-1}\ket{-1}
			\end{align}
		\end{subequations}
		a příslušné Clebsch-Gordanovy koeficienty jsou
		\begin{subequations}
			\begin{align}
				\clebsch{1}{0}{1}{-1}{2}{-1}
					=\clebsch{1}{-1}{1}{0}{2}{-1}
					&=\frac{1}{\sqrt{2}}\\
				\clebsch{1}{-1}{1}{-1}{2}{-2}
					&=1.
			\end{align}
		\end{subequations}
	
	\item
		V dalším kroku se přejde do podprostoru $j=1$. 
		Vektor s nejvyšší vahou $\ket{1\,1}$ lze zkonstruovat pouze ze dvou vektorů báze nesložených momentů hybnosti,
		\begin{equation}
			\ket{1\,1}
				=c_{1}\ket{0}\ket{1}+c_{2}\ket{1}\ket{0}.
		\end{equation}
		Tento vektor musí být kolmý na $\ket{2\,1}$, 
		\begin{equation}
			\braket{2\,1}{1\,1}=0,			
		\end{equation}
		z čehož plyne rovnice
		\begin{equation}
		\frac{c_{1}}{\sqrt{2}}+\frac{c_{2}}{\sqrt{2}}=0.
		\end{equation}
		Koeficienty $c_{1}$, $c_{2}$ jsou navíc vázany normalizační podmínkou $\abss{c_{1}}+\abss{c_{2}}=1$.
		Podle Condon-Shortleyovy fázové konvence se koeficienty $c_{1}$, $c_{2}$ volí reálné, a navíc koeficient u $\ket{1}\ket{0}$ kladný.
		To vede na jednoznačné vyjádření
		\begin{equation}
			\ket{1\,1}
				=\frac{1}{\sqrt{2}}\left(\ket{1}\ket{0}-\ket{0}\ket{1}\right),
		\end{equation}
		z čehož lze získat Clebsch-Gordanovy koeficienty
		\begin{subequations}
			\begin{align}
				\clebsch{1}{1}{1}{0}{1}{1}
					&=\frac{1}{\sqrt{2}},\\
				\clebsch{1}{0}{1}{1}{1}{1}
					&=-\frac{1}{\sqrt{2}}.
			\end{align}
		\end{subequations}
	
	\item
		Nyní lze opět působit operátorem $\operator{J}_{-}$, což dá
		\begin{subequations}
			\begin{align}
				\ket{1\,0}
					&=\frac{1}{\sqrt{2}}\left(\ket{1}\ket{-1}-\ket{-1}\ket{1}\right),\\
				\ket{1\,-1}
					&=\frac{1}{\sqrt{2}}\left(\ket{0}\ket{-1}-\ket{-1}\ket{0}\right),
			\end{align}
		\end{subequations}
		takže
		\begin{subequations}
			\begin{align}
				\clebsch{1}{1}{1}{-1}{1}{0}
					&=\frac{1}{\sqrt{2}}\,,\\
				\clebsch{1}{-1}{1}{1}{1}{0}
					&=-\frac{1}{\sqrt{2}}\,,\\
				\clebsch{1}{0}{1}{-1}{1}{-1}
					&=\frac{1}{\sqrt{2}}\,,\\
				\clebsch{1}{-1}{1}{0}{1}{-1}
					&=-\frac{1}{\sqrt{2}}\,.
			\end{align}
		\end{subequations}
	
	\item
		Zbývá určit poslední stav, který leží v jednorozměrném podprostoru $j=0$:
		\begin{equation}
			\ket{0\,0}
				=d_{1}\ket{1}\ket{-1}+d_{2}\ket{0}\ket{0}+d_{3}\ket{-1}\ket{1}.
		\end{equation}
		Podmínky ortogonality
		\begin{equation}
			\braket{2\,0}{0\,0}
				=\braket{1\,0}{0\,0}=0
		\end{equation}
		vedou na soustavu rovnic
		\begin{subequations}
			\begin{align}
				\frac{d_{1}}{\sqrt{6}}+\frac{2d_{2}}{\sqrt{6}}+\frac{d_{3}}{\sqrt{6}}
					&=0,\\
				\frac{d_{1}}{\sqrt{2}}-\frac{d_{3}}{\sqrt{2}}
					&=0,
			\end{align}
		\end{subequations}
		z které vyplývají vztahy mezi koeficienty
		\begin{equation}
			d_{1}
				=d_{3}
				=-d_{2}.
		\end{equation}
		S uvážením Condon-Shortleyovy fázové konvence je tedy 
		\begin{equation}
			\ket{0\,0}
				=\frac{1}{\sqrt{3}}\left(\ket{1}\ket{-1}-\ket{0}\ket{0}+\ket{-1}\ket{1}\right)
		\end{equation}
		a
		\begin{subequations}
			\begin{align}
			\clebsch{1}{1}{1}{-1}{0}{0}
				=\clebsch{1}{-1}{1}{1}{0}{0}
				&=\frac{1}{\sqrt{3}}\\
			\clebsch{1}{0}{1}{0}{0}{0}
				&=-\frac{1}{\sqrt{3}}.
			\end{align}
		\end{subequations}
	\end{itemize}

	\begin{table}[!htbp]
		\centering
		\begin{tabular}{|rr||r|rr|rrr|rr|r|}
			\hline
			& $j$ & 2  & 2  & 1  & 2 & 1 & 0 & 2  & 1  & 2 \\
			& $m$ & +2 & +1 & +1 & 0 & 0 & 0 & -1 & -1 & -2\\
			$m_{1}$ & $m_{2}$ & & & & & & & & &\\
			\hline\hline
			+1 & +1 & 1 & & & & & & & &\\
			\hline
			+1 & 0  & & $\frac{1}{\sqrt{2}}$ & $\frac{1}{\sqrt{2}}$ & & & & & &\\
			0 & +1  & & $\frac{1}{\sqrt{2}}$ & $-\frac{1}{\sqrt{2}}$ & & & & & &\\
			\hline
			+1 & -1  & & & & $\frac{1}{\sqrt{6}}$ & $\frac{1}{\sqrt{2}}$ & 
				$\frac{1}{\sqrt{3}}$ & & &\\
			0 & 0  & & & & $\frac{2}{\sqrt{6}}$ & 0 & $-\frac{1}{\sqrt{3}}$ & & &\\
			-1 & +1  & & & & $\frac{1}{\sqrt{6}}$ & $-\frac{1}{\sqrt{2}}$ & 
				$\frac{1}{\sqrt{3}}$ & & &\\
			\hline
			0 & -1  & & & & & & & $\frac{1}{\sqrt{2}}$ & $\frac{1}{\sqrt{2}}$ & \\
			-1 & 0  & & & & & & & $\frac{1}{\sqrt{2}}$ & $-\frac{1}{\sqrt{2}}$ & \\
			\hline
			-1 & -1 & & & & & & & & & 1\\
			\hline
		\end{tabular}
		\scaption{
			\protect\small Clebsch-Gordanovy koeficienty pro impulsmomenty $j_{1}=j_{2}=1$. 
			Pokud v tabulce není uvedeno žádné číslo, je příslušný C-G koeficient nulový.
		}
		\label{tab:AngularMomentumCoupling11}
	\end{table}

	Všechny vypočítané Clebsch-Gordanovy koeficienty pro impulsmomenty $j_{1}=j_{2}=1$ jsou přehledně uvedeny v tabulce~\ref{tab:AngularMomentumCoupling11}.

\sec{Shrnutí}

	Obecný postup výpočtu Clebsch-Gordanových koeficientů je tedy následující:
	\begin{enumerate}
		\item
			Začne se s vektorem s nejvyšší váhou
			\begin{equation}
				\ket{j_{1},j_{2},j=j_{1}+j_{2},m=j_{1}+j_{2}}
					=\ket{j_{1}\,j_{1}}\ket{j_{2}\,m_{2}}
			\end{equation}
		
		\item
			Opakovaně se zapůsobí posunovacím operátorem $\operator{J}_{-}$ na obě strany rovnice.
			Tím se naleznou všechny vektory $\ket{j_{1},j_{2},j=j_{1}+j_{2},m}$, $m\in\left\{-(j_{1}+j_{2}),\dotsc,j_{1}+j_{2}\right\}$ z podprostoru $j=j_{1}+j_{2}$.
		
		\item
			Vektor z podprostoru s o jedničku nižším $j=j_{1}+j_{2}-1$ a s odpovídajícím nejvyšším možným $m=j_{1}+j_{2}-1$ se určí z podmínky kolmosti na již vypočtený vektor z prostoru $j=j_{1}+j_{2}$,
			\begin{equation}
				\braket{j_{1},j_{2},j=j_{1}+j_{2}-1,m=j_{1}+j_{2}-1}
					{j_{1},j_{2},j=j_{1}+j_{2},m=j_{1}+j_{2}-1}=0.
			\end{equation}
			V Condon-Shortleyově fázová konvenci je pak koeficient u členu s nejvyšším $m_{1}$ kladný a reálný.
		
		\item
			Body 2 a 3 se opakují do té doby, než se dospěje do podprostoru s nejnižším možným $j=\abs{j_{1}-j_{2}}$.
	\end{enumerate}	

\end{solution}
