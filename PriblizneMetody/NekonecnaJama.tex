\subsection[Nekonečně hluboká jáma]{Aproximace základního stavu nekonečně hluboké potenciálové jámy}
Pomocí variačního principu nalezněte nejlepší aproximaci základního stavu nekonečně hluboké potenciálové jámy pološířky $a$
\begin{equation}
    V(x)=
    \begin{cases}	
        0 & \abs{x}<a \\ \infty & \abs{x}>a
    \end{cases}
\end{equation}
s testovací funkcí
\begin{equation}
    \theta_{\lambda}(x)=\braket{x}{\theta(\lambda)}=a^{\lambda}-\abs{x}^{\lambda}
\end{equation}
a srovnejte s přesným řešením 
\begin{subequations}
    \begin{align}
        E_{1}&=\frac{\hbar^{2}}{2m}\frac{\pi^{2}}{4a^{2}},\\
        \phi_{1}(x)&=\frac{1}{\sqrt{a}}\,\cos{\frac{\pi x}{2a}}.
    \end{align}
\end{subequations}

\begin{solution}
	K řešení se využije vztah~\eqref{eq:Ritz}, kde se minimalizace bude provádět přes jediný parametr $\lambda$.
	Výpočet spočívá ve dvou krocích:
	\begin{enumerate}
	\item
		\emph{Výpočet střední hodnoty Hamiltoniánu pro vlnovou funkci $\theta_{\lambda}(x)$:}
		\begin{align}
			\bar{H}(\lambda)
				&\equiv\frac{\matrixelement{\theta(\lambda)}{\operator{H}}{\theta(\lambda)}}{\braket{\theta(\lambda)}{\theta(\lambda)}}
					=\frac{-\frac{\hbar^{2}}{2m}\int_{-a}^{a}\theta^{*}_{\lambda}(x)\frac{\d^{2}}{\d x^{2}}\theta_{\lambda}(x)\d x}
					{\int_{-a}^{a}\abs{\theta_{\lambda}(x)}^{2}\d x}=\nonumber\\
				&=\equationcomment{\text{v čitateli i jmenovateli integrujeme sudé funkce} \\ \text{-- stačí počítat na intervalu }(0; a)}=\nonumber\\
				&=-\frac{\hbar^{2}}{2m}\frac{\int_{0}^{a}\left(a^{\lambda}-x^{\lambda}\right)
					\frac{\d^{2}}{\d x^{2}}\left(a^{\lambda}-x^{\lambda}\right)\d x}
					{\int_{0}^{a}\left(a^{\lambda}-x^{\lambda}\right)^{2}\d x}=\nonumber\\
				&=\frac{\hbar^{2}}{2m}\,\lambda(\lambda-1)
					\frac{\int_{0}^{a}\left(a^{\lambda}-x^{\lambda}\right)x^{\lambda-2}\d x}
					{\int_{0}^{a}\left(a^{2\lambda}-2x^{\lambda}a^{\lambda}+x^{2\lambda}\right)\d x}=\nonumber\\
				&=\frac{\hbar^{2}}{2m}\,\lambda(\lambda-1)
					\frac{\left[\frac{1}{\lambda-1}a^{\lambda}x^{\lambda-1}-\frac{1}{2\lambda-1}x^{2\lambda-1}\right]_{0}^{\lambda}}
					{\left[a^{2\lambda}x-\frac{2}{\lambda+1}a^{\lambda}x^{\lambda+1}+\frac{1}{2\lambda+1}x^{2\lambda+1}\right]}=\nonumber\\
				&=\frac{\hbar^{2}}{2ma^{2}}\,\lambda(\lambda-1)
					\frac{\frac{1}{\lambda-1}-\frac{1}{2\lambda-1}}{1-\frac{2}{\lambda+1}+\frac{1}{2\lambda+1}}=\nonumber\\
				&=\frac{\hbar^{2}}{2ma^{2}}\,\lambda(\lambda-1)
					\frac{\frac{2\lambda-1-\lambda+1}{(\lambda-1)(2\lambda-1)}}
					{\frac{(\lambda+1)(2\lambda+1)-2(2\lambda+1)+\lambda+1}{(\lambda+1)(2\lambda+1)}}=\nonumber\\
				&=\frac{\hbar^{2}}{4ma^{2}}
					\frac{(\lambda+1)(2\lambda+1)}{(2\lambda-1)}.
		\end{align}

	\item
		\emph{Výpočet minima funkce $\bar{H}(\lambda)$:}
		\begin{align}
			\frac{\partial{\bar{H}(\lambda)}}{\partial\lambda}&=0\nonumber\\
			(2\lambda+2+2\lambda+1)(2\lambda-1)-2(\lambda+1)(2\lambda+1)&=0\nonumber\\
			4\lambda^{2}-4\lambda-5&=0
		\end{align}
		Minimum je dáno kladným kořenem
		\begin{equation}
			\lambda_{\ti{min}}=\frac{1+\sqrt{6}}{2}\approx1,\!723.
		\end{equation}
		Po dosazení vychází
		\begin{align}
			E_{\ti{min}}\equiv\bar{H}(\lambda_{\ti{min}})&=\frac{\hbar^{2}}{4ma^{2}}\frac{2\sqrt{6}+5}{2}=\nonumber\\
				&=\frac{2\sqrt{6}+5}{\pi^{2}}\,E_{0}\approx\nonumber\\
				&\approx1,\!00298 E_{0}
		\end{align}
	\end{enumerate}

	Za pozornost stojí, že i velice jednoduchá testovací funkce závislá jen na jednom jediném parametru
	dává velice přesný odhad energie základního stavu.
\end{solution}
