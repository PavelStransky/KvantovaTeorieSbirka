\subsection{Van der Waalsova interakce}
\label{sec:VanDerWaals}
Uvažujte dva atomy vodíku, přičemž vektor vzájemné polohy jejich jader $\vector{R}$ míří od prvního atomu k druhému, polohy elektronů vůči příslušným atomům jsou udány vektory 
$\vector{r}_{1}$, $\vector{r}_{2}$.

Pro dostatečně velkou vzájemnou vzdálenost atomů vůči vzdálenostem jejich elektronů a při hrubé aproximaci $E_{n\geq2}^{(0)}\approx0$ (to značí, že všechny energie jednotlivých atomů vodíku kromě základních stavů berte jako nulové)
nalezněte opravu k energii základního stavu systému a rozhodněte, zda uvažovaná interakce bude přitažlivá či odpudivá.

Výpočet provádějte v adiabatické aproximaci, tj. předpokládejte, že atomy se vůči sobě nepohybují.

\begin{solution}
	Neporušený Hamiltonián je součtem Hamiltoniánů dvou neinteragujících atomů vodíku, jehož spektrum (vlastní energie a vlastní funkce) je známé.
	Oprava (porucha) pak bude dána interakcemi konstituentů jednoho atomu s konstituenty atomu druhého:
    \begin{subequations}
        \begin{align}
            \operator{H}&=\operator{H}_{0}+\operator{H}_{\ti{I}},\\
            \operator{H}_{0}&=\frac{\vectoroperator{p}_{1}^{2}}{2m}+\frac{\vectoroperator{p}_{2}^{2}}{2m}-\frac{\gamma}{\operator{r}_{1}}-\frac{\gamma}{\operator{r}_{2}},\\
            \operator{H}_{\ti{I}}&=\frac{\gamma}{\operator{R}}+\frac{\gamma}{\operator{r}}-\frac{\gamma}{\abs{\vectoroperator{R}+\vectoroperator{r}_{2}}}-\frac{\gamma}{\abs{\vectoroperator{R}-\vectoroperator{r}_{1}}}.
    	\end{align}                
    \end{subequations}
	V interakčním Hamiltoniánu souvisí jednotlivé členy postupně s interakcí kladně nabitých jader, interakcí elektronů ($\operator{r}=\abs{\vectoroperator{R}+\vectoroperator{r}_{2}-\vectoroperator{r}_{1}}$),
	interakcí prvního jádra s elektronem druhého atomu a interakcí druhého jádra s elektronem prvního atomu.

	Za předpokladu, že rozměry atomů jsou mnohem menší než jejich vzájemná vzdálenost, lze vzít jen \trick{nejnižší členy multipólového rozvoje}
	\begin{align}
		\frac{1}{\abs{\vector{R}-\vector{r}}}
		&=\frac{1}{R}-r_{i}\frac{\partial}{\partial R_{i}}\frac{1}{R}
		+\frac{1}{2}r_{i}r_{j}\frac{\partial^{2}}{\partial R_{i}\partial R_{j}}\frac{1}{R}-\dotsb=\nonumber\\
		&=\frac{1}{R}+\frac{R_{i}r_{i}}{R^{3}}
		+\frac{1}{2R^{3}}\left(3\frac{R_{i}R_{j}}{R^{2}}-\delta_{ij}\right)r_{i}r_{j}+\dotsb=\nonumber\\
		&=\frac{1}{R}+\frac{\vector{R}\cdot\vector{r}}{R^{3}}
		+\frac{1}{2R^{3}}\left(3\frac{(\vector{R}\cdot\vector{r})^{2}}{R^{2}}-r^{2}\right)+\dotsb.
	\end{align}
	Další člen multipólového rozvoje je 
	$\frac{\vector{R}\cdot\vector{r}}{2R^{5}}\left(5\frac{\left(\vector{R}\cdot\vector{r}\right)^{2}}{R^{2}}-3r^{2}\right)$.
	Jednotlivé řády multipólového rozvoje $H_{\ti{I}}$ jsou:
    \begin{subequations}
        \begin{align}
            \operator{H}_{\ti{I}}^{\hi{0}}&=0,\\
            \operator{H}_{\ti{I}}^{\hi{1}}
            &=\frac{\gamma}{\operator{R}^{3}}\left[\vectoroperator{R}\cdot\left(\vectoroperator{r}_{1}-\vectoroperator{r}_{2}\right)
            +\vectoroperator{R}\cdot\vectoroperator{r}_{2}-\vectoroperator{R}\cdot\vectoroperator{r}_{1}\right]=0,\\
            \operator{H}_{\ti{I}}^{\hi{2}}
            &=\frac{\gamma}{2\operator{R}^{3}}\left[3\frac{\left(\vectoroperator{R}\cdot\left(\vectoroperator{r}_{1}-\vectoroperator{r}_{2}\right)\right)^{2}}{\operator{R}^{2}}
            -\left(\vectoroperator{r}_{1}-\vectoroperator{r}_{2}\right)^{2}-\right.\nonumber\\
            & \qquad\left.-3\frac{\left(\vectoroperator{R}\cdot\vectoroperator{r}_{2}\right)^{2}}{\operator{R}^{2}}+\operator{r}_{2}^{2}
            -3\frac{\left(\vectoroperator{R}\cdot\vectoroperator{r}_{1}\right)^{2}}{\operator{R}^{2}}+\operator{r}_{1}^{2}\right]=\nonumber\\
            &=\frac{\gamma}{\operator{R}^{3}}\left[\vectoroperator{r}_{1}\cdot\vectoroperator{r}_{2}
            -3\frac{\left(\vectoroperator{R}\cdot\vectoroperator{r}_{1}\right)\left(\vectoroperator{R}\cdot\vectoroperator{r}_{2}\right)}{\operator{R}^{2}}\right].
        \end{align}
    \end{subequations}
	Nadále se omezíme na přiblížení $\operator{H}_{\ti{I}}\approx\operator{H}_{\ti{I}}^{(2)}$\sfootnote{
		To je vlastně interakční energie dvou dipólových momentů $\vectoroperator{d}_{1,2}=-e\,\vectoroperator{r}_{1,2}$:
		\begin{equation}
			\operator{H}_{\ti{I}}^{(2)}=\frac{1}{4\pi\epsilon_{0}}\frac{1}{R^{3}}\left[\vectoroperator{d}_{1}\cdot\vectoroperator{d}_{2}
		-3\frac{\left(\vectoroperator{R}\cdot\vectoroperator{d}_{1}\right)\left(\vectoroperator{R}\cdot\vectoroperator{d}_{2}\right)}{\operator{R}^{2}}\right].
		\end{equation}
	}.

	Ve \trick{speciálně zvolené souřadné soustavě}, ve které osa $z$ směřuje ve směru spojnice jader atomů od prvního jádra ke druhému, je
	\begin{align}
		\operator{H}_{\ti{I}}
		&=\frac{\gamma}{\operator{R}^{3}}\left[\operator{x}_{1}\operator{x}_{2}+\operator{y}_{1}\operator{y}_{2}+\operator{z}_{1}\operator{z}_{2}
		-3\frac{(\operator{R}\operator{z}_{1})(\operator{R}\operator{z}_{2})}{\operator{R}^{2}}\right]=\nonumber\\
		&=\frac{\gamma}{\operator{R}^{3}}\left[\operator{x}_{1}\operator{x}_{2}+\operator{y}_{1}\operator{y}_{2}-2\operator{z}_{1}\operator{z}_{2}\right].
	\end{align}
	přičemž $\vectoroperator{r}_{1}=(\operator{x}_{1},\operator{y}_{1},\operator{z}_{1})$ jsou složky vektoru $\vectoroperator{r}_{1}$; analogicky pro vektor $\vectoroperator{r}_{2}$.

	Neporušený základní stav dvou volných atomů vodíku je dán vlnovou funkcí
	\begin{equation}
	\ket{\phi_{1}}=\ket{n=1\,l=0\,m=0}_{1}\ket{n=1\,l=0\,m=0}_{2}\equiv\ket{1}\ket{2}
	\end{equation}
	(při použití zjednodušeného označení $\ket{1,2}\equiv\ket{n=1\,l=0\,m=0}_{1,2}$).
	Atomy jsou nerozlišitelné, vlnový vektor tudíž musí být symetrický nebo antisymetrický vůči záměně částic.\sfootnote{Pro úplnou analýzu je potřeba zahrnout i spinový stav elektronů; pro účely této úlohy stačí uvažovat, že spin elektronů se složí na antisymetrický singletní stav, aby celková vlnová funkce byla antisymetrická.}
	To je splněno.

	\emph{1. oprava k energii} je dle poruchové teorie
	\begin{align}
		E_{11}^{\hi{1}}&=\matrixelement{\phi_{1}}{\operator{H}_{\ti{I}}}{\phi_{1}}=\nonumber\\
		&=\frac{\gamma}{\operator{R}^{3}}\Big[\matrixelement{1}{\operator{x}_{1}}{1}\matrixelement{2}{\operator{x}_{2}}{2}+\matrixelement{1}{\operator{y}_{1}}{1}\matrixelement{2}{\operator{y}_{2}}{2}
		-2\matrixelement{1}{\operator{z}_{1}}{1}\matrixelement{2}{\operator{z}_{2}}{2}\Big].
	\end{align}
	K určení maticových elementů lze využít výběrová pravidla podle Wigner-Eckartova teorému~\eqref{eq:WignerEckartSelectionRules}.
	Komponenty vektorových operátorů $\vectoroperator{r}_{1,2}$ se vyjádří pomocí komponent tenzorových operátorů 1. řádu, viz~\eqref{eq:VectorTensor}). Je tedy $\lambda=1$ a jednotlivé komponenty jsou odlišeny indexem $\mu$.
	Výběrová pravidla pak dávají $J=j\pm1$ a $M=m+\mu$, kde v našem případě $J\equiv l=0$, $j\equiv l=0$, $M=m=0$.
	To není splněno pro žádnou ze složek operátorů $\vectoroperator{r}_{1,2}$, takže všechny maticové elementy na pravé straně výrazu pro 1. opravu jsou nulové.\sfootnote{
		To úzce souvisí s tím, že dipólový moment atomů v základním stavu je nulový.
	}

	\emph{2. oprava k energii} základního stavu dává
	\begin{align}
        E_{11}^{\hi{2}}
            &=\sum_{\substack{n_{1}\neq1 \\ n_{2}\neq1 \\ l_{1}\,m_{1}\,l_{2}\,m_{2}}}
                \frac{\abs{\bra{2}\bra{1}\operator{H}_{\ti{I}}\ket{n_{1}\,l_{1}\,m_{1}}\ket{n_{2}\,l_{2}\,m_{2}}}^{2}}
                {2E_{1}^{\hi{0}}-E_{n_{1}}^{\hi{0}}-E_{n_{2}}^{\hi{0}}}\approx\nonumber\\
            &\approx\frac{1}{2E_{1}^{\hi{0}}}\sum\bra{2}\bra{1}\operator{H}_{\ti{I}}\ket{n_{1}\,l_{1}\,m_{1}}\ket{n_{2}\,l_{2}\,m_{2}}
                \bra{n_{1}\,l_{1}\,m_{1}}\bra{n_{2}\,l_{2}\,m_{2}}\operator{H}_{\ti{I}}\ket{1}\ket{2}=\nonumber\\
            &=\frac{1}{2E_{1}^{\hi{0}}}\bra{2}\bra{1}\operator{H}_{\ti{I}}\left(\operator{1}-
                \ket{1}\ket{2}\bra{2}\bra{1}\right)\operator{H}_{\ti{I}}\ket{1}\ket{2}=\nonumber\\
            &=\frac{1}{2E_{1}^{\hi{0}}}\bra{2}\bra{1}\operator{H}_{\ti{I}}^{2}\ket{1}\ket{2},
	\end{align}
	kde se provedla \trick{hrubá aproximace $E_{n\geq2}^{\hi{0}}\approx0$}, využily se relace úplnosti a znalost nulovosti maticových elementů $\bra{2}\bra{1}\operator{H}_{\ti{I}}\ket{1}\ket{2}$.
	Při formálně zcela správném řešení je nutné uvažovat nerozlišitelnost částit, avšak výsledek bude stejný (díky užití relací úplnosti).

	Kvadrát Hamiltoniánu poruchy je
	\begin{equation}
		\operator{H}_{\ti{I}}^{2}=\frac{\gamma^{2}}{\operator{R}^{6}}\left[\operator{x}_{1}^{2}\operator{x}_{2}^{2}+\operator{y}_{1}^{2}\operator{y}_{2}^{2}+4\operator{z}_{1}^{2}\operator{z}_{2}^{2}
			+2\operator{x}_{1}\operator{x}_{2}\operator{y}_{1}\operator{y}_{2}-4\operator{x}_{1}\operator{x}_{2}\operator{z}_{1}\operator{z}_{2}-4\operator{y}_{1}\operator{y}_{2}\operator{z}_{1}\operator{z}_{2}\right].
	\end{equation}
	Maticový element pro smíšené členy (poslední tři členy v závorce) se vynuluje díky symetrii základního stavu.\sfootnote{
		Toto lze opět dokázat pomocí Wigner-Eckartova teorému.
		Na základě příkladu~\ref{sec:DyadicProduct} lze dyadický součin dvou vektorových operátorů $\vectoroperator{R}$, $\vectoroperator{S}$ vyjádřit pomocí tenzorových operátorů nultého, prvního a druhého řádu.
		Speciálně pro $\vectoroperator{R}=\vectoroperator{S}$ a pro smíšené složky $R_{j}$, $R_{k}$, $j\neq k$ platí
		\begin{align}
			\operator{R}_{1}\operator{R}_{2}&=\frac{\tensoroperatorcomponent{T}{2}{2}-\tensoroperatorcomponent{T}{2}{-2}}{2\im},
			&\operator{R}_{2}\operator{R}_{3}&=-\frac{\tensoroperatorcomponent{T}{2}{1}+\tensoroperatorcomponent{T}{2}{-1}}{2\im},
			&\operator{R}_{1}\operator{R}_{3}&=-\frac{\tensoroperatorcomponent{T}{2}{1}-\tensoroperatorcomponent{T}{2}{-1}}{2},
		\end{align}
		dají se tedy vyjádřit pomocí tenzorového operátoru řádu $\lambda=2$ s projekcí $\mu\neq0$.
		Po nahrazení $(\operator{R}_{1},\operator{R}_{2},\operator{R}_{3})=(\operator{x}_{1,2},\operator{y}_{1,2},\operator{z}_{1,2})$ se dospějena základě výběrových pravidel Wigner-Eckartova teorému k závěru, že libovolný maticový element $\matrixelement{100}{R_{j}R_{k}}{100}=0$, $j\neq k$.
	}
	Ze symetrie také vyplývá
	\begin{equation}
		\matrixelement{1}{\operator{x}_{1}^{2}}{1}=\matrixelement{1}{\operator{y}_{1}^{2}}{1}=\matrixelement{1}{\operator{z}_{1}^{2}}{1}=\frac{1}{3}\matrixelement{1}{\operator{r}_{1}^{2}}{1},
	\end{equation}
	takže druhou opravu k energii lze nakonec vyjádřit jako
	\begin{align}
		E_{11}^{\hi{2}}
			&=\frac{\gamma^{2}}{2E_{1}^{\hi{0}}R^{6}}\left[
				\matrixelement{1}{\operator{x}_{1}^{2}}{1}\matrixelement{2}{\operator{x}_{2}^{2}}{2}+
				\matrixelement{1}{\operator{y}_{1}^{2}}{1}\matrixelement{2}{\operator{y}_{2}^{2}}{2}+
				4\matrixelement{1}{\operator{z}_{1}^{2}}{1}\matrixelement{2}{\operator{z}_{2}^{2}}{2}
				\right]=\nonumber\\
			&=\frac{\gamma^{2}}{2E_{1}^{\hi{0}}R^{6}}\,\frac{6}{9}\matrixelement{1}{\operator{r}_{1}^{2}}{1}\matrixelement{2}{\operator{r}_{2}^{2}}{2}
            \label{eq:VanDerWaalsPerturbation2}
	\end{align}

	Zbytek úlohy se dořeší v $x$-reprezentaci.
	Energetické hladiny atomu vodíku jsou
	\begin{equation}
		E_{n}^{\hi{0}}=-\frac{\gamma}{2a_{0}}\frac{1}{n^{2}}
	\end{equation}
	a radiální část vlnové funkce základního stavu zní\sfootnote{
		Celá vlnová funkce základního stavu je $\braket{r,\theta,\phi}{100}=R_{10}(r)Y_{00}(\theta,\phi)$, 
		kde $Y_{00}(\theta,\phi)=1/\sqrt{4\pi}$. 
		Úhlovou a radiální část lze od sebe odseparovat a zde se počítá maticový element operátoru, který na úhlovou část nepůsobí, proto stačí uvažovat pouze radiální část.
	}
	\begin{equation}
		\label{eq:HydrogenR10}
		R_{10}(r)=\braket{r}{100}=\frac{2}{a_{0}^{\frac{3}{2}}}\e^{-\frac{r}{a_{0}}},
	\end{equation}
	kde $a_{0}=\hbar^{2}/\gamma m$ je Bohrův poloměr.
	Maticový element je dán integrálem
	\begin{align}
		\matrixelement{100}{\operator{r}^{2}}{100}
		&=\frac{4}{a_{0}^{3}}\int_{0}^{\infty}\e^{-\frac{r}{a_{0}}}\,r^{2}\e^{-\frac{r}{a_{0}}}\,r^{2}\d r=\nonumber\\
		&=\frac{4}{a_{0}^{3}}\int_{0}^{\infty}r^{4}\e^{-\frac{2r}{a_{0}}}\d r=\nonumber\\
		&=-\frac{4}{a_{0}^{3}}\,4\frac{a_{0}}{2}\int_{0}^{\infty}r^{3}\e^{-\frac{2r}{a_{0}}}\d r=\dotsb=\nonumber\\
		&=-\frac{4}{a_{0}^{3}}\,4!\left(\frac{a_{0}}{2}\right)^{5}\left[\e^{-\frac{2r}{a_{0}}}\right]_{0}^{\infty}=\nonumber\\
		&=\frac{4}{a_{0}^{3}}\,24\left(\frac{a_{0}}{2}\right)^{5}=\nonumber\\
		&=3a_{0}^{2},
	\end{align}
	který po dosazení do vztahu pro 2. opravu energie~\eqref{eq:VanDerWaalsPerturbation2} dá konečný výsledek
	\begin{equation}
		E_{11}^{\hi{2}}=\frac{3\gamma^{2}a_{0}^{4}}{E_{1}^{\hi{0}}R^{6}}=-\frac{6\gamma a_{0}^{5}}{R^{6}}.
	\end{equation}

	Oprava je záporná, lze z ní tedy usuzovat na přitažlivost sil mezi atomy a na její rychlý pokles s narůstající vzdáleností.

	Atomy nemusí být nutně vodíkové, výsledek platí i pro jiné atomy nebo molekuly,
	pouze musí dostatečně přesně platit, že na tento systém lze nahlížet jako na soustavu kladně nabitého centra 
	(jádro + elektrony z vnitřních slupek) a okolo obíhající valenční elektron.
	Pak vidíme, že Van der Waalsova síla je tím větší, čím jsou větší rozměry atomů.

	Zatímco pro základní stav je 1. oprava poruchové teorie k nulová, pro excitované stavy již tomu tak být nemusí.
	To znamená, že atomy v excitovaných stavech se budou ovlivňovat silněji na velkých vzdálenostech, velikost opravy bude klesat jen jako $\sim1/R^{3}$.
	Navíc excitované stavy mohou být degenerované a je nutné použít degenerovanou poruchovou teorii.

	Ačkoliv jsou jednotlivé dipólové momenty atomů v základním stavu nulové (dipólový moment $\equiv$ střední hodnota operátoru dipólového momentu),
	jsou Van der Waalsovy síly projevem dipól-dipólové interakce.
	Je to důsledek toho, že záladní stav není vlastním stavem dipólového operátoru $\vectoroperator{d}$.
		
	Detaily ohledně Van der Waalsovy síly naleznete v přehledovém článku~\cite{Margenau1939} či v učebnici~\cite{Formanek2004}, kapitola 10.10.3.
\end{solution}